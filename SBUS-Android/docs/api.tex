\documentclass[12pt,a4paper,twoside]{article}
\usepackage{fancyhdr,graphicx,latexsym}

\setlength{\parindent}{0cm}
\setlength{\parskip}{2ex plus1ex minus 0.5ex}

\addtolength{\evensidemargin}{-2.5cm}
\addtolength{\oddsidemargin}{-0.5cm}
\addtolength{\textwidth}{3cm}

\addtolength{\headheight}{0.2cm}
\addtolength{\topmargin}{-1cm}
\addtolength{\textheight}{2.5cm}

\renewcommand{\_}{\texttt{\symbol{95}}}
\addtolength{\fboxsep}{0.1cm}
\newcommand{\param}[1]{\textit{\textrm{\textmd{#1}}}}
\newcommand{\codebar}{\rule{\textwidth}{0.3mm}}
% \newcommand{\spc}{\hspace{0.5mm}$\sqcup$\hspace{0.6mm}}
\newcommand{\spc}{\hspace{0.5mm}$\Box$\hspace{0.5mm}}
\newcommand{\todo}[1]{\textbf{TODO: #1}}

\newlength{\codelen}
\newcommand{\code}[1]
{\begin{center}\fbox{\parbox{16cm}{\texttt{#1}}}\end{center}}

\fancyhead{}
\fancyhead[RO,LE]{\thepage}
\fancyhead[LO,RE]{SBUS API}
\fancyfoot{}
\pagestyle{empty}

\newenvironment{bulletlist}
{
	\begin{itemize}
	\setlength{\itemsep}{0ex}
	\setlength{\parsep}{0ex}
}
{
	\end{itemize}
}

\newenvironment{alphalist}
{
	\begin{enumerate}
	\setlength{\itemsep}{0ex}
	\setlength{\parsep}{0ex}
	\renewcommand{\labelenumi}{(\alph{enumi})}
}
{
	\end{enumerate}
}

\newenvironment{numericlist}
{
	\begin{enumerate}
	\setlength{\itemsep}{0ex}
	\setlength{\parsep}{0ex}
}
{
	\end{enumerate}
}

\begin{document}

% \sfseries
\centerline{\textbf{\LARGE SBUS API}}
\begin{center} \large
David Ingram\\
TIME-EACM Project\\
University of Cambridge Computer Lab\\
8th November 2008\\
\end{center}

{ \parskip 1mm plus 1pt \tableofcontents }
\pagestyle{fancy}

\section{General}

The \verb^snode^ object is the in-memory representation
for all SBUS message content (an \verb^smessage^ includes
an \verb^snode^ and some metadata, such as the message source).

\verb^snode^s are obtained in one of three ways: by using a
set of ``packing'' functions to convert data from your programming
language's native types, by importing from some XML text, or by receiving
them from SBUS across the network.

Given an \verb^snode^, you can use a set of ``extraction''
functions to convert to your programming language's natively
typed variables, or export it to an XML string, or send it
across the network using SBUS.

The packing and extracting operations are
performed manually using C++ functions. The recommended approach is to
consolidate these calls into specific marshalling methods you define
for your application's objects. It would also be possible to write
a stub generator to automatically create the packing and extracting
code, but such a thing is not shipped with SBUS.

It is only permissable to extract data from an \verb^snode^ which has
been \textit{validated}. This is the process of having it checked
against a schema written in the LITMUS language, which describes
the valid message types.
As well as ensuring the message conforms to the schema, the validation
procedure performs the additional function of establishing the exact
types for the data, for example deciding whether a C++
\verb^int^ represents an integer or a boolean, and whether a string
is really a string or actually should be interpreted as a timestamp.
Finally it sets up some state for efficient extraction.

Validation is normally done for you, whenever you use SBUS to send an
\verb^snode^ across the network. In fact SBUS will validate the
message once at the sender before transmitting it, and again at the
receiver before handing it over to the other application.

Internally the validated \verb^snode^ is marshalled to a binary
representation for transmission across the network, and then
unmarshalled back to an \verb^snode^ at the other end.
The schema is essential at the unmarshalling stage in order to
make sense of the message sent over the wire, since the default
binary encoding does not transmit verbose type information.

In a few special cases (for example if you import an XML message
from a file, and then wish to inspect it locally with the extraction
functions), an application might wish to validate a message itself.
This service is only available to C++ programs; see Section \ref{schemas}
on schemas for details.

\subsection{Memory management}

The C++ API adopts a convention for memory management, as follows.

Buffers and objects passed \textit{to} SBUS functions must be allocated and
deallocated by the caller. The caller may deallocate them immediately
after making the call, because the API takes copies of all arguments.

Buffers and objects returned by SBUS functions are
allocated by SBUS, but must be deallocated using the \verb^delete^
operator by the caller when they are no longer required.

There are currently a small number of exceptions to the memory
management rule for efficiency reasons (subject to review), as follows:

\begin{bulletlist}
\item Element names returned to you by \verb^snode::get_name()^
      are shallow copies; do not delete them.
\item The \verb^snode^ functions \verb^extract_txt()^, \verb^extract_bin()^
      and \verb^extract_value()^ return shallow copies. Do not delete
		the data returned.
\end{bulletlist}

\section{Packing and extracting}

\subsection{Packing functions}

The packing functions are used to programmatically build messages as
\verb^snode^s in memory. The first set of pack functions simply wrap primitive
types with an \verb^snode^. The second set pack multiple existing
\verb^snode^s into compound data structures. You must pack fields
in the same order they are specified in the schema.

The packing functions allow you to optionally specify element
names, if you wish to make your code clearer and introduce additional
field name checks. It is also acceptable to omit them (NULL is passed
instead by the default arguments). If you do pass element names
you must not delete these strings whilst the \verb^snode^ is in use,
because shallow copies are taken for speed. Generally this isn't
a problem as the element names you specify are usually string constants
in your program.

To simplify memory management, when you delete an \verb^snode^ containing
references to other subnodes (inside structures, lists, arrays etc)
the destructor also recursively destroys those. This means that you must
not manually delete intermediate \verb^snode^s created during the
packing process. The benefit is that you do not have to name or
keep track of them, so pack functions can be chained, for example:
\verb^sn = pack(pack(5), pack("foo"));^

\begin{verbatim}
snode *pack(int n, const char *name = NULL); // int, flg
snode *pack(const char *s, const char *name = NULL); // txt, value
snode *pack(const char *data, int len, const char *name = NULL); // bin
snode *pack(double x, const char *name = NULL);
snode *pack(sdatetime *time, const char *name = NULL);
snode *pack(slocation *loc, const char *name = NULL);
snode *pack(SNull, const char *name = NULL);

int select = get_id(const char *value);

// Struct, List, Seq, Array:
snode *mklist(const char * name = NULL);
snode *snode::append(snode *n);
snode *pack(snode **array, int n, const char *name = NULL);
snode *pack(snode *n1, ..., snode *n6, const char *name = NULL);
// Variations of the above for from 2 to 6 nodes
\end{verbatim}

\verb^SNull^ is a predefined constant of type \verb^const char *^ which
has the value \verb^NULL^. It must be used to pack
optional elements when they are not present in this particular message.
(Consequently optional strings can be packed directly, if you use
\verb^NULL^ to indicate the default value).

Some of the packing functions are used for more than one datatype,
e.g. the first is used for integers and booleans
since these are both represented by \verb^int^s in C++. This isn't
a problem because at validation time the schema will differentiate
between them and assign the more specific correct type.

There is no explicit mention of choice elements in the
packing functions; the user simply packs the alternative they have chosen for
this particular message. SBUS tries to decide which alternative of a choice
you are using by looking at the types of the elements. It is
simplistic and does not look ahead at the types of further elements when it
does so, hence it can fail in cases where a human examining the whole message
could differentiate. If there is an ambiguity it can't resolve then
element names must be provided for those elements (this is the only
case where element names are not optional).

\subsection{Extraction functions}

The extraction functions are the logical counterpart of the packing
functions. The programmer is assumed to know what types to expect
(they are working from the same schema, after all). There are
extract functions for each primitive type to unwrap an \verb^snode^,
converting its contents back to a native value.

If the \verb^snode^ you hold is a compound type, you have three options for
extracting its contents. Firstly, you can call \verb^extract_item()^ specifying
either the index or field name of the child member required. This returns
another \verb^snode^ which can then itself be extracted. Secondly, if the
primitive field you wish to retrieve is only nested inside one level, you can
access it directly by specifying an optional \verb^item^ index to the relevant
typed extraction function (e.g. \verb^sn->extract_dbl(3)^ means extract the
floating point value which is the fourth sub-element inside the structure or
list \verb^sn^). Finally, if the \verb^snode^ is a structure rather than
a list, you can do the same but specify the child by name
instead (there is a separate set of extraction functions which accept a string
argument for this purpose).

\verb^class snode^\\
\verb^{^\\
\verb^   int count(); // Returns number of children, if a compound type^\\
\verb^   snode *extract_item(int item);^\\
\verb^   snode *extract_item(const char *name);^\\
\verb^   snode **extract_array();^\\
\verb^   const char *get_name(); // Shallow copy^\\
\verb^   const char *get_name(int item); // Shallow copy^\\
\verb^   ^\\
\verb^   int extract_int(^\textit{args}\verb^);^\\
\verb^   int extract_flg(^\textit{args}\verb^);^\\
\verb^   double extract_dbl(^\textit{args}\verb^);^\\
\verb^   const char *extract_txt(^\textit{args}\verb^); // Shallow copy^\\
\verb^   void *extract_bin(^\textit{args}\verb^); // Shallow copy^\\
\verb^   int num_bytes(^\textit{args}\verb^);^\\
\verb^   sdatetime *extract_clk(^\textit{args}\verb^);^\\
\verb^   slocation *extract_loc(^\textit{args}\verb^);^\\
\verb^   const char *extract_value(^\textit{args}\verb^); // Shallow copy^\\
\verb^   int extract_enum(^\textit{args}
	\verb^); // Only defined on values which have been validated^\\
\verb^}^\\

\textit{args} may be (i) nothing, (ii) \verb^int item^, or (iii)
\verb^const char *name^.\\
For example, you may call:\\
\verb^extract_int()^, \verb^extract_int(int item)^,
or \verb^extract_int(const char *name)^.\\
The value \verb^item^ specifies the child index -- 0, 1, 2 etc.

\section{Component initialisation}

\verb^scomponent^ represents one component (with a connection to the component
metadata description normally via the RDC, or from a local file).

The following sections deal with the API calls in roughly the order
they are likely to be used within a program.

\subsection{Establishing a component instance}

\begin{verbatim}
scomponent *com;

com = new scomponent(const char *cpt_name, const char *instance_name =
      NULL, CptUniqueness unique = UniqueMultiple);

enum CptUniqueness { UniqueMultiple, UniqueSingle, UniqueReplace };
\end{verbatim}

Creating the component object also starts the wrapper process running.
The component is not active, however -- the wrapper waits for further
configuration.

The \verb^instance_name^ may be omitted, in which case it is set to
the same as \verb^cpt_name^. Instance names are passed to the RDC
and can be used to specify individual replicas of a component for
mapping.

Note: The uniqueness flag is not implemented at present.

\subsection{Adding endpoints}

\begin{verbatim}
sendpoint *ep;

ep  = com->add_endpoint(const char *name, EndpointType type,
      const char *msg_hash, const char *reply_hash = NULL);

enum EndpointType
{ EndpointServer, EndpointClient, EndpointSource, EndpointSink };
\end{verbatim}

LITMUS hash codes such as \verb^msg_hash^ and \verb^reply_hash^
represent the schema for a message or endpoint.
They are passed to the API as strings containing a 12 digit
hexadecimal number, e.g. ``\verb^AABBCCDDEEFF^''.

There is another version of \verb^add_endpoint()^ which accepts
\verb^HashCode^ objects instead of hex strings, which is sometimes
more convenient:

\begin{verbatim}
ep  = com->add_endpoint(const char *name, EndpointType type,
      HashCode *msg_hc, HashCode *reply_hc = NULL);
\end{verbatim}

Repeat calls to \verb^add_endpoint()^ as many times as necessary
to declare all the endpoints supported by the component.
You may call \verb^add_endpoint()^ before or after starting the
component.
Starting a component serves the dual purpose of attaching it to
its component metadata and starting the wrapper.
The endpoints added before the \verb^start()^ call
are checked against those listed in the component metadata,
those added afterwards are not.

\subsection{Adding RDCs}

\begin{verbatim}
com->add_rdc(const char *address);
com->add_rdc(const char *address);
...
\end{verbatim}

The RDCs you add are used to register the component with, and also to
lookup other components for mapping. These are in addition to those
set by the \verb^SBUS_RDC_PATH^ environment variable and to the
default location on the localhost.

Lookups will try all the RDCs sequentially until one succeeds.
The component will try to register itself with \textit{all}
the RDCs you have added (some may not exist, which is not an error).
The exception is addresses which begin with an exclaimation mark '!'
as a special flag, which indicates that they are for lookup only,
not registration.
In all cases the default RDC port number is used if none is supplied.

\subsection{Logging}

There are four types of log output: errors, warnings, messages and
debugging. Errors are only generated in fatal situations (the
component always shuts down after emitting an error). Each of
these four types of logging may be directed to the component's
log file and/or to stdout. This is controlled by the log levels:
0 = nothing, 1 = errors, 2 = warnings, 4 = messages, 8 = debugging.
You can add these values together to select a mixture of message types.
The initial log level for the log file is read from the environment
variable \verb^SBUS_LOG_LEVEL^, and for stdout from \verb^SBUS_ECHO_LEVEL^.
If these are not set then everything is logged to file, and everything apart
from debugging is sent to stdout (so the default is highly verbose).
A component may change its own log levels, by calling the following
static API function:

\begin{verbatim}
scomponent::set_log_level(int log, int echo);
\end{verbatim}

\textbf{Note} that \verb^scomponent::set_log_level()^ must be called
\textit{before} \verb^scomponent::start()^. The API does not currently
provide a way to change log levels after that point.
Also, if you wish to inhibit the initial message from the library saying
that the wrapper has started, call \verb^set_log_level^ before
\verb^new scomponent^ creates the component.
Log levels \textit{can} be changed remotely at any time using the built-in
endpoint for this purpose.

Both \verb^log^ and \verb^echo^ are formed by adding together
constants representing the desired types of information:
\verb^LogErrors^, \verb^LogWarnings^, \verb^LogMessages^ and
\verb^LogDebugging^.

The log files are stored in a directory specified by the \verb^SBUS_DIR^
environment variable: \verb^$SBUS_DIR/log/^.
If this environment variable is not set then the
directory \verb^$HOME/.sbus/log/^ is used by default.
Log files are named \verb^cptname-instance-X.log^ or \verb^cptname-X.log^
(the latter if no instance name is set).
Here \verb^X^ stands for either \verb^W^ or \verb^L^, indicating whether
the message arises from the wrapper or the library respectively.
Component log files overwrite previous files with the same name.
If you simultaneously run two components on the same machine with
identical component and instance names, the contents of the log
file are undefined.

\subsection{Starting the component}

\begin{verbatim}
com->start(const char *metadata_filename, int port = -1);
\end{verbatim}

This instructs the wrapper to go ahead and validate the component's
endpoints, and then (if the schemas match the metadata) to register
with the RDCs if applicable and to begin listening for external
connections from other components.

The \verb^LITMUS_PATH^ environment variable is used to locate the
metadata file.

\subsection{\texttt{scomponent} method summary}

\begin{verbatim}
class scomponent
{
   scomponent(const char *cpt_name, const char *instance_name = NULL,
         CptUniqueness unique = UniqueMultiple);
   ~scomponent();
      
   sendpoint *add_endpoint(const char *name, EndpointType type,
         const char *msg_hash, const char *reply_hash = NULL);
   sendpoint *add_endpoint(const char *name, EndpointType type,
         HashCode *msg_hc, HashCode *reply_hc = NULL);
   void add_rdc(const char *address);   
   void start(const char *metadata_filename, int port = -1);

   int count_endpoints();
   sendpoint *get_endpoint(int index);
   sendpoint *get_endpoint(const char *name);  
   sendpoint *fd_to_endpoint(int fd);
}
\end{verbatim}

\subsection{Mapping an endpoint}

Mapped \verb^sendpoint^s correspond with an open TCP-IP connection
from the wrapper to another component's wrapper.

An endpoint of type client may be mapped to at most one server.
The \verb^map()^ call \textit{remaps} endpoints of type client
if they are already mapped.
An endpoint of type server may be mapped
to any number of clients. Source endpoints may be mapped to any number
of sink endpoints, and a sink endpoint may be mapped to any number of
sources.

It is also possible to use an \textit{unmapped} endpoint, and no
errors result (obviously no data will be returned, and RPCs will
block indefinitely).
Many components are written to start running without specifying any
maps; the user can then link them into the component graph using dynamic
reconfiguration tools. This is generally more flexible than hardcoding
\verb^map()^ calls into the application.

It is recommended that all components should be capable of starting
unmapped (perhaps based on a command-line option).
Before you use \verb^map()^, consider whether it would be more
powerful to leave mapping policy open to the RDCs remapping engine
or other external agents.
Nevertheless, there are times when a component needs to set
up maps itself, and to do this they use the following API:

\begin{verbatim}
class sendpoint
{
   // Mapping: 
   char *map(const char *address, const char *endpoint, int sflags = 0,
           const char *pub_key = NULL);
   // If endpoint is NULL, assumes same name as this end
   // Caller must delete the string returned by map()
   void unmap();
   int ismapped();
   ...
}
\end{verbatim}

The \verb^map()^ function returns a string containing the address of the
other component if successful, otherwise NULL. If non-NULL, this
string must be deleted by the caller.

The address string is formatted as described in the ``Component
addresses'' section of the user guide. It can take one of two
main forms.
The first form specifies an explicit network address, and does not
use a RDC.
The address is then a string of the form \verb^hostname:port^
(or just \verb^:port^, which is equivalent to \verb^localhost:port^).
The \verb^hostname^ part may be a DNS name or an IP address in
dotted decimal notation.
% and an optional public key if
% identity must be confirmed.

The second, considerably more flexible form of mapping
consults the RDC and tries to satisfy the map constraints
specified by the address string.
This is the normal approach, since usually the current network address
of the component required is not known.
See the user guide for the list of valid constraints (each of which
begins with a plus sign in the string).

The \verb^endpoint^ parameter may be NULL, in which case it is
assumed that you wish to map to an endpoint on the target component
with the same name as the local endpoint being mapped (it is common
for client/server and source/sink pairs of endpoints which are
intended to be connected to have the same name).

Normally endpoints are mapped once and then only remapped occasionally,
e.g. by external agencies to achieve migration. A P2P overlay network
is an exception -- in this case remapping will be very frequent
and would use the explicit address method rather than an RDC, since
components may be randomly accessed potentially worldwide.

\subsubsection{Mapping flags}

Mapping flags are not implemented yet, so currently \verb^sflags^
should be set to zero.

% \begin{verbatim}
% // Values for sflags:
% const int UnavailableSilent = 1 << 0;
% const int DisconnectSilent  = 1 << 1;
% const int AttemptFailover   = 1 << 2;
% const int AttemptReconnect  = 1 << 3;
% const int AllowMigration    = 1 << 4;
% const int AwaitTransfer     = 1 << 5;
% \end{verbatim}

\subsection{Subscriptions}

To keep the interface straightforward, the \verb^map()^ call does not
include parameters for subscriptions or topics; these are
specified separately by the \verb^subscribe()^ function.

You can change the subscription for currently mapped endpoints,
or specify the subscription for new connections. If you are
mapping an endpoint and need a subscription set up immediately
(at connection time), you should call \verb^subscribe()^ first
(with \verb^peer^ set to NULL), and then \verb^map()^.

\begin{verbatim}
class sendpoint
{
   ...
   void subscribe(const char *topic, const char *subscription,
      const char *peer);
}
\end{verbatim}

In \verb^subscribe()^ calls, either \verb^topic^ or \verb^subscription^
(or both) may be NULL. If both are NULL, then all messages from this
endpoint will be returned. If neither are NULL, then both must match
for an event to be delivered. The subscription is stated in the
content-matching language also used by the prebuilt Broker component
and described in the SBUS Overview document.
The topic is a straightforward comparison on the topic name used to emit
events by the component. Content and topic based subscriptions are evaluated
automatically for each message and each subscriber by the emitting
component's wrapper process.

The \verb^peer^ parameter may be the name (component or instance)
of a currently mapped peer whose subscription should be altered.
It may also be the special string \verb^"*"^, in which case
the subscription in force for all currently mapped peers is changed.
Finally, if \verb^peer^ is NULL then the default subscription
used for future maps to this endpoint is altered.

\section{Data transfer}

\subsection{Data transfer methods}

\begin{verbatim}
class sendpoint
{
   int fd;
   ...	
   smessage *rcv();
   void reply(smessage *query, snode *result, int exception = 0,
      HashCode *hc = NULL);
   void emit(snode *msg, const char *topic = NULL, HashCode *hc = NULL);
   smessage *rpc(snode *query, HashCode *hc = NULL);
}
\end{verbatim}

Client endpoints call \verb^rpc()^ and server endpoints
call \verb^rcv()^ and \verb^reply()^.
Source endpoints call \verb^emit()^ and sink endpoints call \verb^rcv()^.

The optional \verb^HashCode^ arguments to \verb^emit()^, \verb^rpc()^
and \verb^reply()^ are used by polymorphic endpoints to indicate the
type of the message being sent. Normal, non-polymorphic endpoints
should omit \verb^HashCode^, and the predefined type for the endpoint
will be used instead.

\verb^rpc()^ returns NULL if the endpoint is unmapped, or mapped to
more than one peer, or if it becomes unmapped during the call due to
the server disconnecting.

If \verb^exception^ is \verb^1^ then the result contains error information
and isn't a normally formatted \verb^snode^.

Methods which return data to the caller, i.e. \verb^rcv()^, \verb^rpc()^,
return an \verb^smessage^ structure, defined below. The payload
is stored in the \verb^tree^ field; the other members are useful
metadata. \verb^seq^ is used for matching up RPC replies.

\begin{verbatim}
class smessage
{
   char *source_cpt, *source_inst, *source_ep;
   int source_clone;
   int seq;
   HashCode *hc;  
   snode *tree;
   void *state; // for conversations only
   char *topic; // NULL if not applicable
}
\end{verbatim}

\subsection{Handling client requests}

The methods in \verb^sendpoint^ are called by the component's
business logic.
You are free to write single- or multi-threaded components.
If a component is single-threaded and has multiple services it
will be necessary to \verb^select()^ on the file descriptors
for each one to see which to process next. The wrapper only
delivers incoming messages once they have been completely
received, so there is no need for
a component to implement timeouts whilst reading a message.

For sink endpoints, the component
just has to call the \verb^rcv()^ method.
This blocks until a message is available; if this is undesirable
the component should \verb^select()^ on \verb^fd^ first.

For an RPC the component must \verb^rcv()^ and then \verb^reply()^
passing back the query so the result can be matched up with it.

For source endpoints, the wrapper automatically tracks subscribers, so the
component just has to call \verb^emit()^, optionally supplying a topic.

For conversations, the component calls \verb^rcv()^, \verb^reply()^,
\verb^rcv()^, \verb^reply()^ and so on.
Replies also include the last message received.
The state held by this side concerning the conversation can be saved
in the \verb^smessage::state^ variable -- this value, if set, is
returned by all subsequent \verb^rcv()^'s from the same conversation.

All out-of-band messages such as connection closed and flow control
are currently dealt with by the wrapper, and not passed up to the
component through the \texttt{sendpoint} interface.

\section{Other features of \texttt{scomponent}}

\subsection{Declaring schemas}

\begin{verbatim}
class scomponent
{
   ...
   HashCode *declare_schema(const char *schema);
   HashCode *load_schema(const char *file);
}
\end{verbatim}
%  void add_certificate(const char *cert);

The \verb^declare_schema^ and \verb^load_schema^ functions add
a new schema (read from a string or a file respectively) to the wrapper's
schema cache whilst the component is running.
This is useful in the special case of polymorphic
source endpoints, since these may need to define new types
before emitting them.
Once declared, a schema is available to all
endpoints.
The HashCode returned by these functions must be deleted by the caller.

It is not an error if the schema is already in the cache; in this
case nothing happens, but the HashCode is still returned. This makes
these functions a convenient way to look up HashCodes for schemas,
without needing to link with \verb^libsbuswrap^ in order to parse the
schema in the application process.

\verb^load_schema^ uses the \verb^LITMUS_PATH^ environment variable
to locate the file containing the schema.

% The \verb^cert^ is a role certificate (there may be several)
% which is later presented to the server. Currently certificates are
% unimplemented.

\subsection{Obtaining wrapper state}

The following methods of \verb^scomponent^ are valid after
\verb^start()^ has returned.

\begin{verbatim}
class scomponent
{
   ...
   int count_builtins();
   sbuiltin *get_builtin(int index); // No need to delete sbuiltin
   int get_listen_port();
   const char *get_address();        // No need to delete string
   snode *get_status();              // Caller should delete snode after use
   const char *get_schema(HashCode *hc); // Caller should delete string
}

struct sbuiltin
{
   char *name;
   EndpointType type;
   HashCode *msg_hc, *reply_hc;
}
\end{verbatim}

\verb^get_address()^ returns the canonical address string for this
component (formed from the IP address of the first non-loopback interface,
together with the listening port).

\verb^get_status()^ returns an \verb^snode^ containing the current
component status structure (this is exactly the same thing as that
returned to third parties by the builtin endpoint of the same name).

\verb^get_schema()^ interrogates the wrapper's schema cache, and
returns the full text for the schema represented by the supplied
hash code. This is useful if a component has polymorphic sink
endpoints and hence may receive messages it doesn't know the schema
for. If the hash code presented does not appear in the schema
cache, the string \verb^"?"^ is returned.

\section{Shutting down}

\subsection{Disconnecting from another component}

This closes the TCP connection between the components.

\begin{verbatim}
ep->unmap();
\end{verbatim}

\subsection{Terminating your component}

This is done by deleting the \verb^scomponent^ object:

\begin{verbatim}
delete com;
\end{verbatim}

This deregisters with the RDC, terminates the wrapper process and frees
resources.

\section{Thread safety}

\textbf{Each endpoint must be accessed by only one thread}. Multiple threads
are fine provided each is responsible for different endpoints.
A single thread may use many endpoints, but it must do so exclusively.

In a multi-threaded component, endpoints of type Server, Source or Sink
can generally be handled conveniently by their own thread. Multiple
threads often need to employ the same Client endpoint, however.
To make this possible without breaking the rule of one thread per
endpoint, a \verb^clone^ method is provided to duplicate endpoints.
This is merely a convenience method which simply calls
\verb^add_endpoint()^ for you to generate another endpoint of exactly
the same type.

\begin{verbatim}
sendpoint *epc;

epc = com->clone(ep);
\end{verbatim}

Only the thread which starts the component may safely call \verb^clone()^.

\section{Import and Export}

\begin{verbatim}
snode *sn = snode::import(const char *s, const char **err);
snode *sn = snode::import_file(const char *path, const char **err);

char *s = sn->toxml(int pretty);
\end{verbatim}

The \verb^import()^ routines convert XML text in a string or from a file
into a parsed \verb^snode^. In case of error they return NULL
and place an explanation of the error in \verb^*err^.

\verb^toxml()^ exports an \verb^snode^ as XML text. If \verb^pretty^ is
\verb^1^ the output will contain newlines and indentation; if
\verb^pretty^ is \verb^0^ it will all be on one line.

\section{Schemas}
\label{schemas}

Schemas are not normally manipulated directly by applications; only
the wrapper does this. C/C++ programs may however call schema functions
if they wish (for example, clearly administrative programs such
as \verb^analysecpt^ need to do this). This section describes the
behind-the-scenes facilities available for such tools.

Schemas are read and parsed like this:

\begin{verbatim}
schema = Schema::create(const char *s, const char **err);
schema = Schema::load(const char *pathname, const char **err);
\end{verbatim}

The first form reads the schema text from a string in memory, the
second form from a file on disk. If the schema is invalid (and also
if the file can't be read, for \verb^load^), these functions return
NULL and place an error message in \verb^err^.

\textit{Technical note:} Schemas are pre-scanned, to build the tree and to extract
element names into a symbol table. Symbol table indexes appear in
\verb^snode^'s so that they can be looked up by element name if desired,
without too much overhead if element names are not used.

\subsection{Validation}

\begin{verbatim}
int validate(snode *node, Schema *schema, const char **err);
\end{verbatim}

\verb^validate()^ checks that an \verb^snode^ (typically just imported)
conforms to a particular schema. It returns 1 if it is valid,
otherwise it returns 0 and places a pointer to a string containing an
error message into err. \verb^validate()^ also makes types more specific,
suitable for unambiguous marshalling.

\section{Simplified \texttt{select()} wrapper}

This is an optional utility provided by the library. It wraps
the system \verb^select()^ call in a nicer interface. You can use
either \verb^multiplex^ or the native \verb^select()^ to manage
your endpoint file descriptors in components.

Adding an endpoint to the multiplex, e.g. \verb^multi->add(ep)^,
is similar to\\
\verb^multi->add(ep->fd, MULTI_READ)^, but also
enables checking of the \verb^ep->message_waiting()^ method. This
is necessary if your endpoint might receive a message whilst
it is being mapped or unmapped (in such cases the message is
automatically enqueued for later delivery, after the map call returns,
but ordinary \verb^select()^ will not notice a message has arrived
because it has been read from the file descriptor already).

\begin{verbatim}
class multiplex
{
   enum multi_mode { MULTI_READ, MULTI_WRITE };

   multiplex();
   ~multiplex();

   void add(int fd, multi_mode mode = MULTI_READ);
   void add(sendpoint *ep); // Also enables check for ep->message_waiting()
   void remove(int fd, multi_mode mode = MULTI_READ);
   void remove(sendpoint *ep, int silent = 0);
   void clear();

   // These functions return a FD, or -1 if none ready before timeout:
   int poll();
   int wait();
   int pause(int us);

   multi_mode last_mode();
}
\end{verbatim}

\end{document}
