\documentclass[12pt,a4paper,twoside]{article}
\usepackage{fancyhdr,graphicx,latexsym}

\setlength{\parindent}{0cm}
\setlength{\parskip}{2ex plus1ex minus 0.5ex}

\addtolength{\evensidemargin}{-2.5cm}
\addtolength{\oddsidemargin}{-0.5cm}
\addtolength{\textwidth}{3cm}

\addtolength{\headheight}{0.2cm}
\addtolength{\topmargin}{-1cm}
\addtolength{\textheight}{2.5cm}

\renewcommand{\_}{\texttt{\symbol{95}}}
\addtolength{\fboxsep}{0.1cm}
\newcommand{\param}[1]{\textit{\textrm{\textmd{#1}}}}
\newcommand{\codebar}{\rule{\textwidth}{0.3mm}}
% \newcommand{\spc}{\hspace{0.5mm}$\sqcup$\hspace{0.6mm}}
\newcommand{\spc}{\hspace{0.5mm}$\Box$\hspace{0.5mm}}

\newlength{\codelen}
\newcommand{\code}[1]
{\begin{center}\fbox{\parbox{16cm}{\texttt{#1}}}\end{center}}

\fancyhead{}
\fancyhead[RO,LE]{\thepage}
\fancyhead[LO,RE]{PIRATES Data Representation}
\fancyfoot{}
\pagestyle{empty}

\newenvironment{bulletlist}
{
	\begin{itemize}
	\setlength{\itemsep}{0ex}
	\setlength{\parsep}{0ex}
}
{
	\end{itemize}
}

\newenvironment{alphalist}
{
	\begin{enumerate}
	\setlength{\itemsep}{0ex}
	\setlength{\parsep}{0ex}
	\renewcommand{\labelenumi}{(\alph{enumi})}
}
{
	\end{enumerate}
}

\newenvironment{numericlist}
{
	\begin{enumerate}
	\setlength{\itemsep}{0ex}
	\setlength{\parsep}{0ex}
}
{
	\end{enumerate}
}

\begin{document}

% \sfseries
\centerline{\textbf{\LARGE PIRATES Data Representation}}
\begin{center} \large
David Ingram\\
TIME-EACM Project\\
University of Cambridge Computer Lab\\
13th August 2009\\
\end{center}

{ \parskip 1mm plus 1pt \tableofcontents }
\pagestyle{fancy}

\section{LITMUS schemas}

PIRATES message schemas describe the structure and types of permitted messages.
Each endpoint has a schema. The schemas are written in a language called
LITMUS (Language of Interface Types for Messages in Underlying Streams).

The messages themselves (which conform to a LITMUS schema) may be encoded
either in XML or a compact binary form for transmission. Neither of these
include type information because that is contained in the schema, however each
message header incorporates a value called the \textit{LITMUS code}, which is
a hash of the schema used for checking purposes. This makes messages
self-identifying but not self-descriptive (the schema has to be looked up from
elsewhere). Messages encoded as XML obviously also include some redundant field
name information, since this is used to name the tags which wrap data.

%For interoperability purposes there is also a mapping between LITMUS
%and the RELAX-NG schema language. There is no reverse mapping from RELAX-NG
%to LITMUS because RELAX-NG is too general and permits structures which
%do not make sense to PIRATES.

\subsection{LITMUS tests}

The hash value of the schema is the LITMUS code. The check that this matches
the expected schema's code is the LITMUS test. Note that schemas are normalised
before the LITMUS number is calculated, so that whitespace will not be
included. Importantly, the names of fields are included though.

\subsection{Primitive types}

\begin{tabular}{ll}
\texttt{int name} & Integer\\
\texttt{dbl name} & Double-precision floating point\\
\texttt{flg name} & Flag (boolean)\\
\texttt{txt name} & Text string\\
\texttt{bin name} & Binary data\\
\texttt{clk name} & Date and time (clock)\\
\texttt{loc name} & Location\\
\end{tabular}

\verb^name^ stands for an arbitrary word used as a label.
LITMUS understands both \verb^type name^ (C style syntax, as
shown above), and also the reverse, \verb^name type^.

\subsection{Composite types}

This section defines the syntax for each composite type, and then
gives specific examples of each. An occurrence of \verb^elt^ can be replaced
with any of the primitive or composite types, giving rise to
arbitrarily nested content. Many of the constructions here are
stated as \verb^name composite-type^; in these cases LITMUS also accepts
\verb^composite-type name^.

\begin{description}
\item[Structures:]
	\verb^name { elt1 elt2 ... eltN }^\\
   \verb^car^\\
   \verb^{^\\
   \verb^   loc pos^\\
   \verb^   dbl speed^\\
   \verb^   int passengers^\\
   \verb^}^\\
	\\
   \verb^journey^\\
   \verb^{^\\
   \verb^   loc from^\\
   \verb^   loc to^\\
   \verb^   stats^\\
   \verb^   {^\\
   \verb^      clk elapsed^\\
   \verb^      int max_speed^\\
	\verb^      flg delayed^\\
   \verb^   }^\\
   \verb^}^\\
\item[Lists:]~\\
	\begin{tabular}{ll}
	\verb^name ( elt )^ & -- List, zero or more occurrences of \verb^elt^\\
	\verb^name (+ elt )^ & -- List, one or more occurrences of \verb^elt^\\
	\verb^name (N elt )^ & -- Array, N occurrences of \verb^elt^\\
	\\
	\end{tabular}\\
   \verb^crew^\\
   \verb^(2^\\
   \verb^   person^\\
   \verb^   {^\\
   \verb^      txt name^\\
   \verb^      int age^\\
   \verb^   }^\\
   \verb^)^\\
	\\
   \verb^sightings^\\
   \verb^(^\\
   \verb^   clk when^\\
   \verb^)^\\
\item[Optional parts:]
	\verb^[ elt ]^\\
	\verb^[ int height_limit ]^
\item[Choices:]
	\verb^< elt1 elt2 ... eltN >^\\
   \verb^<^\\
   \verb^   txt address^\\
   \verb^   loc gps^\\
   \verb^   int zip_code^\\
   \verb^>^\\
	There is one important restriction to note: options \textbf{may not}
	appear directly within a choice, or vice versa.
\item[Enumerations:]
	\verb^name < #value1 #value2 ... #valueN >^\\
	\verb^transport < #land #sea #air >^
\item[Type definition:]
	\verb^@elt^\\
   \verb^@circle^\\
   \verb^{^\\
   \verb^   loc centre^\\
   \verb^   dbl radius^\\
   \verb^}^\\
	\\
	The top-level name inside \verb^elt^ becomes the type's label.\\
	Type definitions must occur in the global scope.
	You must not define a type with top-level optional or choice
	content (i.e. \verb^@[^ and \verb^@<^ are illegal)\\
	Note: when creating types which are aliases for a primitive type,
	you may find that the \verb^name int^ form is more natural than
	\verb^int name^, e.g. \verb^@weight int^\\
	Type definitions may occur in any order and you may reference a
	type before its definition, as long as it is defined eventually.\\
	There is one important exception to this freedom to reorder things;
	the main top-level type defined by the schema \textit{must} be the
	first one in the file (since that is the only way the parser can
	reliably tell it apart from any other type definitions).
\item[Type reference:]
	\verb&^label name&\\
	\verb&^circle boundary&\\
	\verb&^circle -&\\
\item[Imports:]
	\verb^@"filename"^\\
	\verb^@"cpt_status.idl"^\\
	\\
	Imports all type definitions (user-defined types and the top-level one)
	from the file specified. All the types imported can then be used via
	type references. The file imported must contain only a LITMUS schema
	(directly importing from schemas embedded within component metadata
	\verb^.cpt^ files is not supported).\\
	Import statements may occur anywhere in the file, including above
	the main type definition, and take effect globally (i.e. you can
	refer to types from the imported file before the import statement,
	if you wish).
\end{description}

\subsection{Syntax}

\subsubsection*{Names}

Names may contain the characters \verb^a..z^, \verb^A..Z^,
\verb^0..9^ as well as hyphen and underscore. They must start
with one of the characters \verb^a..z^ or \verb^A..Z^.

\subsubsection*{Unnamed elements}

Anywhere that a \verb^name^ is required, you can substitute the special
name `\verb^-^'. The element name will then be set to
the name of its type instead. This is mainly useful when a structure
contains only a single reference to a type (making this a unique name
within that scope), particularly a
user-defined type which has a suitably descriptive name itself.
Composite type names are \verb^struct^, \verb^list^ and \verb^array^.

\subsubsection*{Whitespace}

Whitespace is used to separate words, but otherwise ignored.
Newlines are whitespace but have no further meaning, apart from
terminating comments.
No commas, semi-colons or other separators are used.

\subsubsection*{Comments}

A comment is introduced with the asterisk character, and extends
to the end of the line (other candidates were \verb^/, %, #^).

\subsubsection*{Multiple declarations}

Plus signs can be used to declare several elements of the same
type without repeating the type specifier. For example:\\
\\
\verb^flg priority + secret^\\
\\
\verb^pilot + copilot^\\
\verb^{^\\
\verb^   txt name^\\
\verb^   bin photo^\\
\verb^}^\\
\\
Plus signs are just a shorthand for the same element repeated with
different names (e.g. \verb^flg priority flg secret^) and do not
automatically create a structure, so they must be used inside another
structure (delimited by braces) to be correct syntax.

\subsection{Special schemas}

There are several special schemas, listed in the table below.
Instead of following the normal LITMUS rules for schemas, these
are all written with just a single special character. They also
have a specially defined corresponding message hash code
(since hashing a one-character schema in the normal way would
not lead to a particularly suitable code).

\begin{center}
\begin{tabular}{|lll|}
\hline
Type & Schema text string & Hash code \rule[-2mm]{0mm}{6.5mm}\\
\hline
Empty message & \verb^0^ & \verb^000000000000^ \rule[0mm]{0mm}{4.5mm}\\
No message    & \verb^!^ & \verb^EEEEEEEEEEEE^\\
Polymorphic   & \verb^*^ & \verb^FFFFFFFFFFFF^ \rule[-2mm]{0mm}{2mm}\\
\hline
\end{tabular}
\end{center}

The first represents messages which are always empty.
This is most often useful as the message schema for an RPC which
takes no arguments (hence the reply type will be specified by a normal
schema, but query messages will be empty). Note that the schema
for an empty message must be quoted (e.g. \verb^"0"^) in XML-formatted
component metadata (otherwise it will be interpreted as an integer,
rather than a string containing just the zero digit).

The second special schema represents no message at all. It is used
as the reply type for endpoints which do not have replies, i.e.
sources and sinks. It may seem strange to have a hash code for a
message type which will never be sent, but doing so keeps the
framework uniform and avoids some special case logic. Notice the
distinction between not sending a message, and sending an empty
message as in the previous case (an empty message serves the purpose
of indicating that something has happened, or that a reply is
needed).

The third special schema matches any kind of message. For example
it is used by the standard event broker component, so that it can
send and receive messages of any type.

\newpage
\section{XML message encoding}

The XML encoding is designed so that the type of all the primitive fields
can be inferred automatically, even without a schema. Of course it is
not possible to distinguish some of the complex types, such as a
structure from a list, without the schema though.

\subsection{Primitive types}

These examples show how the primitive types are represented
in the XML encoding:

\begin{tabular}{l|l}
\verb^int name^ & \verb^<name>123</name>^\\
\verb^dbl name^ & \verb^<name>4.56</name>^, \verb^<name>3e-10</name>^\\
\verb^flg name^ & \verb^<name>true</name>^, \verb^<name>false</name>^\\
\verb^txt name^ & \verb^<name>Foo bar</name>^, \verb^<name>"123 bar"</name>^\\
\verb^bin name^ & \verb^<name>0x01AB89EF 0x23CD45</name>^\\
\end{tabular}

Notice that all content is named. Under the PIRATES XML encoding,
raw data cannot occur after an end tag (this is of course
possible in XML in general) because all primitive types must
be named and hence wrapped inside a tag. The PIRATES XML parser
will reject anything which does not follow this rule,
even if it valid XML.

Text strings \textit{must} be quoted if: a) they start with
a digit, a minus sign, a hash or whitespace, b) they
consist of the strings \verb^true^ or \verb^false^,
c) they end in whitespace, or d) they have zero length.
Otherwise, quoting text strings is optional (the
export function only quotes strings where required).

The quotes are removed from strings by the import function.
Inside a quoted string, or at the beginning of a string,
quotation marks themselves are
encoded as \verb^\"^. Inside unquoted strings, opening
angle brackets are encoded as \verb^\<^ to avoid clashing
with the start of the next tag. On import, the sequences
\verb^\"^, \verb^\<^ and \verb^\\^ are reduced accordingly.

Floating point numbers must contain a decimal point
or an exponent. If a floating point is integral then the
suffix \verb^.0^ must be added.

Binary data may be broken into any number of segments
(separated by whitespace) but each segment must begin with
\verb^0x^ and contain an even number of hexadecimal digits.

\subsubsection*{Text formats for \texttt{clk}}

The timezone is assumed to be fixed (local time or GMT depending on
the scope of the system).

\begin{verbatim}
time ::= HH:MM:SS[.MMM[UUU]]
day ::= DD/MM/YYYY
datetime ::= time | day | time,day | day,time
\end{verbatim}

\subsubsection*{Text formats for \texttt{loc}}

\begin{verbatim}
location ::= lat,lon | lat,lon,elev
lat ::= [-]DDD.FFFF[FFF]
lon ::= [-]DDD.FFFF[FFF]
elev ::= [-]MMMMM.CCM
\end{verbatim}

\subsection{Composite types}

\begin{tabular}{l|l}
\verb^name { elt1 elt2... eltN }^ & \verb^<name>elt1... eltN</name>^\\
\verb^name ( elt )^ & \verb^<name>elt elt...</name>^\\
\verb^name (+ elt )^ & \verb^<name>elt elt...</name>^\\
\verb^name (N elt )^ & \verb^<name>elt elt...</name>^\\
\verb^[ elt ]^ & \verb^elt^ or \verb^<name>-</name>^, where \verb^name^ is
	\verb^elt^'s top-level name\\
\verb^< elt1 elt2... eltN >^ & \verb^elt1^ or \verb^elt2^ ... or \verb^eltN^\\
\verb^name < #val1 #val2... #valN >^ &
	\verb^<name>#val1</name>^ or similar for the other values\\
\verb^@elt^ & (definitions do not need to be serialised)\\
\verb~^label name~ & (expanded before serialisation)\\
\verb~@"filename"~ & (expanded before serialisation)\\
\end{tabular}

\subsection{Empty elements}

Empty elements are supported, and may be written equivalently as either
\verb^<name></name>^ or \verb^<name/>^, in the usual way.
An empty element always signifies an empty list.

N.B. Do not mistake this for an empty string, which is represented by
\verb^<name>""</name>^, or a missing optional element, which is represented
by \verb^<name>-</name>^.

\subsection{XML conformance}

The XML format used by PIRATES does not utilise all of the features in
standard XML. The particular aspects left out are as follows:

\begin{bulletlist}
\item XML attributes are never used by the encoding! This is
	a simplification following the observation that they are
	functionally identical to sub-elements of primitive type
	specified in a certain order.
\item There are no processing instructions (those which start with
   ``\verb^<?^'').
\item There are no comments.
\item The document prolog is omitted;
	A consequence of this is that the XML declaration at the
	start of each message itself (such as \verb^<?xml version="1.0"?>^)
	must be omitted (the format is implicit from the PIRATES protocol
	version number anyway).
	Another consequence is that there are no references to
	document type definitions (DTD's).
\item There are no entity declarations or entity references.
\item There are no CDATA sections.
\item There are no namespaces.
\end{bulletlist}

Whitespace around tags is stripped.

Each message always consists of a \textit{single} XML document.

\newpage
\section{Component metadata format}

The metadata is split into static (class properties) and dynamic
(instance properties). Only the status of one instantiation is
accessible by contacting a component itself, whereas the RDC
collates status information on all instances of a component.

%Authorised modifications (which require the private key used at
%time of original component definition) to the metadata may add new SAP's or
%new versions of existing SAP's API, or change the administrative fields.

\subsection{Static metadata}
\label{metadata}

\begin{verbatim}
@component-metadata
{
   txt name
   txt description
   keywords ( txt keyword )
   txt designer
   txt pubkey
   managed-roles ( ^role-defn - )
   remap-permission ( ^role - )
   eps
   (
      endpoint
      {
         txt name
         ^endpoint-type type
         ^api -
         alternate ( ^api - )
         obsolete ( ^api - )
      }
   )
}

@api
{
   access-control ( ^role - )
   ^idl message + response
}

@role
{
   txt name
   txt issuer
   txt pubkey
}

@role-defn
{
   txt name
   ^idl request
}

@idl txt
@endpoint-type < #client #server #source #sink >
\end{verbatim}

\subsubsection{Administrative fields}

\begin{bulletlist}
\item \verb^name^ -- Component name.
\item \verb^description^ -- Explanation of what the component does, for
		human consumption.
\item \verb^keywords^ -- Keywords which can be used when searching for
		components.
\item \verb^designer^ -- Name of the person who created the component metadata.
\end{bulletlist}

\subsubsection{Endpoint fields}

\begin{bulletlist}
\item \verb^endpoint^ -- Describes one endpoint. Note that clones
		are not part of the static metadata.
\item \verb^api^ -- The primary API for this endpoint.
\item \verb^alternate^ and \verb^obsolete^ -- Alternative API's provided
		for the same endpoint. Those in the alternate list can be used
		instead of the primary one, for example by older clients.
		The obsolete list contains previous API's which are no longer
		supported, and acts as a historic reference for the schema evolution.
		Note: neither of these fields are used at present.
\item \verb^idl^ -- An alias for a text field, when it is used to
		store LITMUS IDL text.
\item \verb^message^ and \verb^response^ -- Describes the format of
		messages sent to and from this endpoint.
		Sink and source endpoints have no response (schema \verb^"!"^).
		Some RPCs may have an empty payload for either message or response
		(schema \verb^"0"^).
\end{bulletlist}

\subsubsection{Access control fields}

\textbf{Note:} access control is unimplemented, so all of the fields in this
section are currently unused (but must still be present).

\begin{bulletlist}
\item \verb^pubkey^ -- Public key for authentication of component implementors.
%\item \verb^signature^ -- fingerprint of rest of a message
%		signed with private key; used for update messages (not stored in RDC).
\item \verb^managed-roles^ -- Defines roles managed by this component,
		and provides standard access points to request certificates for each one.
\item \verb^remap-permission^ -- Lists the roles needed to dynamically remap
		connections to a running instance of this component.
\item \verb^access-control^ -- A list of roles required to connect to this
		endpoint.
\item \verb^issuer^ -- The name of the component which manages this role.
\item \verb^request^ -- A built-in access point used to request
		that a role managed by this component be granted to the client,
		based on evidence presented in the message. This access point always
		has an access type of RPC.
\end{bulletlist}

\newpage
\subsection{Dynamic metadata}
\label{status}

\begin{verbatim}
@component-state
{
   txt address
   txt instance
   txt creator
   txt cmdline
   load { int cpu + buffers }
   ^endpoints -
   [ ^freshness - ]
}

@endpoints
(
   endpoint
   {
      txt name
      int processed + dropped * Message counts
      [ txt subscription ] [ txt topic ]
      peers ( ^peer - )
   }
)

@peer
{
   txt cpt_name
   txt instance
   txt endpoint
   txt address * Host and port
   [ txt subscription ] [ txt topic ]
   int lastseq
   int latency
}

@freshness
{
   int last_ping_secs
   int latency
}
\end{verbatim}

\newpage
\subsubsection{Dynamic fields}

\begin{bulletlist}
%\item \verb^current^ -- This group contains status blocks
%      for all the instantiations of the component which are believed
%      to be currently alive (i.e. responded to their last ping).
%		There will be more than one of these if the component is
%		replicated.
%\item \verb^recent^ -- This section describes previous instantiations.
\item \verb^address^ -- Host IP / DNS name, and port number (separated by
		a colon).
\item \verb^instance^ -- Instance name of this component instantiation.
\item \verb^creator^ -- The person running this instance (and possibly
		responsible if it needs restarting...)
\item \verb^cmdline^ -- Command line by which the component was invoked.
\item \verb^cpu^ --
	current total CPU usage on the machine, as a percentage from 0 to 100.
\item \verb^buffers^ -- how much memory (in KB) is being used for
	buffering messages that are awaiting flow control to clear.
\item \verb^name^ -- The endpoint name.
		Cloned endpoints are identified as
		\verb^epname#clone^, where \verb^epname^ is the endpoint
		name and \verb^clone^ is the clone number.
\item \verb^processed^ -- contains a count of all messages
	processed by the endpoint.
\item \verb^dropped^ -- counts all the messages which have been dropped
	by the component for flow control reasons.
\item \verb^peers^ --  Other components mapped to this one.
		\verb^lastseq^ is the maximum sequence number used.
      Indirect data sources can be inferred by chasing through the
		\verb^peers^ of different components at run-time
		(limited to a list of length 100, using breadth-first search,
		in order to conserve RDC CPU).
		\verb^latency^ is the most recently measured latency for this endpoint
\item \verb^last_ping_secs^ -- The time the component was last pinged by a RDC.
\item \verb^latency^ -- The currently estimated round-trip time for this
	component's \verb^get_status()^ endpoint, in microseconds.
\end{bulletlist}

\newpage
\section{Built-in endpoints}
\label{builtin}

All components implement these. This is done for you by the wrapper.
Components which are pure client still support all these,
although some of the information is not relevant.
This allows administration tools to crawl the graph of components and
clients, even without any widespread use of shared brokers. This graph
is a generalisation of the component list functionality possible in
SCOP and other centralised systems.

\begin{itemize}

\item \verb^get_metadata()^

Returns a \verb^component-metadata^ object (see the data representation
document for the corresponding IDL). This is useful for components
which are not registered with a RDC (if they are, this information
is also available from the component definition in the RDC).

\item \verb^get_status()^

Returns a \verb^component-state^ object (see the data representation
document for the corresponding IDL. The \verb^freshness^
fields are ignored). The RDC calls this
method periodically, and also uses the response time to
measure latency.

\item \verb^lookup_schema(txt hashcode)^

Returns the schema (as a plain text field)
corresponding to the given hashcode. This is useful for polymorphic
components which send messages with different types from any of
those defined in the component metadata itself. The schemas of
such messages are held in a cache by the component.

\item \verb^map(txt endpoint, txt peer_address, txt peer_endpoint, txt certificate)^

Tries to map the specified endpoints to a given peer. This mapping
is in addition to any already set up.

\item \verb^unmap(txt endpoint, txt certificate)^

Unmaps all peers from the specified endpoint.

\item \verb^subscribe(txt endpoint, txt peer, txt subscription, txt topic)^

This changes the default subscription for new maps if \verb^peer^ is
NULL, or the current subscription for a specific peer (component or
instance name), or the current subscription for all presently mapped
peers (if \verb^peer^ is \verb^*^).

\item \verb^divert(txt endpoint, txt new_address, txt new_endpoint, txt certificate)^

This causes divert messages to be sent to all peers currently
connected to the endpoint, requesting them to remap to the
named target instead of the endpoint in question.

If all goes well then some time after the divert request there will be no
peers mapped to the endpoint, except for any new arrivals after
the divert was called.

\item \verb^terminate()^

Instructs the component to shut down.

% Called by the RDC or other component with sufficient authority.
% The RDC uses it when a new instance of a component which is
% supposed to be unique is created, and it has requested that
% the older instance be overridden.

\item \verb^set_log_level(int log_level, int echo_level)^

Changes the log levels for the component. The levels are a sum of
1 for errors, 2 for warnings, 4 for messages and 8 for debugging.

% Errors = 1, Load statistics = 2, Connections = 4, Actions = 8.

\end{itemize}

% \subsubsection*{Expected future endpoints}
% 
% \begin{itemize}
% 
% \item \verb^ping()^
%
% Called by the RDC periodically. Returns an empty acknowledgement
% immediately and is used to measure round-trip times.
%
% \item \verb^migrate(txt new_address)^
% 
% A request to initiate component migration.
% 
% \end{itemize}

\newpage
\section{Subscription language}

\textbf{Note:} \textit{Not all aspects of the subscription language are
implemented at present (in particular wildcards, built-in variables and
time/place comparisons are not functional).}

Source endpoints allow topic and content-based subscriptions
(or both). These facilities are used heavily by the optional broker
component, but any component may enable subscriptions in order to
filter stream output. The subscription is specified by the sink,
and sent across the network so that filtering can be done on a
per-connection basis at source.

Topic-based subscription consists of a simple match on topic name.
Content-based subscriptions are described by a match string, such as this:

\begin{verbatim}
vehicle/owner? & vehicle/position.lon < -1.5 &
vehicle/timestamp > '1/6/2006' & vehicle/timestamp < '16/6/2006' &
((vehicle/occupants.items = 2 & vehicle/occupants/#1 = 'fred bloggs') |
(vehicle/description/colour = '*green' & vehicle/speed > 50 &
vehicle/type ~ 'car'))
\end{verbatim}
% vehicle:type ~ 'car'))

This matches events for which there exists a \verb^vehicle/owner^ tag, observed
west of -1.5 degrees longitude in the first half of June, in which either Fred
is the passenger or the vehicle is green (including dark green, olive green,
etc), not a car and travelling faster than 50mph.

The basic syntax to describe a tag is a slash-separated path from
the XML root. This may contain simple wildcards: \verb^*^ matches
any name and \verb^**^ also matches \verb^/^ characters. Tag paths
evaluate to the
content of the tag. If the tag does not exist or does not have a
primitive data type (i.e. it contains other tags)
the whole match fails immediately. Path elements may be element names,
or the index within a structure or array specified as a \verb^#^ sign
followed by an integer.
The special sequence of a path followed by \verb^.items^ evaluates
to the number of child elements of that node (or -1 if it doesn't exist).

%The expression
%\texttt{tag/tag/tag:attr-name} returns the value of the named
%attribute rather than the content of the tag (and fails the match
%if there is no attribute with that name).

Single quotes specify a value, which may begin or end with an asterisk
wildcard.
The comparison operators are
\texttt{=}, \verb^~^, \texttt{<} and \texttt{>}. The tilde
means ``not equal''. The comparison is data-type aware and works
with the schema datatypes \texttt{int} (numeric),
\texttt{dbl} (decimal), \texttt{txt} (a string in single quotes),
\texttt{flg} (\verb^0^ or \verb^1^) and
\texttt{clk} (an appropriately formatted timestamp string in single quotes).
For non-numeric types \texttt{<} and \texttt{>} implement
lexicographic comparison (forbidden if one of the operands is
wildcarded).

Tests are combined with the \verb^&^ and \verb^|^ operators (and, or)
and grouped with parentheses as required. In the absence of parentheses
expressions are parsed from left to right (there is no precedence
for operators). The special test \verb=path/name?= succeeds if
the tag or attribute mentioned exists. Negation can be performed
with the \texttt{!} operator, for example \verb^! foo/bar < 'value'^
or \verb^! (some grouped tests...)^.

It is legitimate to compare two fields within the same message
rather than a field with a fixed value, hence you can say things like
\verb^alpha/beta/gamma > foo/bar^.

There are special values
\verb^$srccpt^, \verb^$srcinst^, \verb^$srcep^, \verb^$destep^ and
\verb^$sequenceid^ which contain the details of
the message origin and destination, and can be compared like any
other value, e.g.\\
\verb^$srccpt ~ 'bogus-component-*'^.

There is a special syntax for the data types \verb^clk^ and
\verb^loc^ which allow you to break out the day, time-of-day, latitude
and longitude as separate numeric fields for comparison:
\verb^foo/bar.day^, \verb^foo/bar.month^,
\verb^foo/bar.year^, \verb^foo/bar.hour^, \verb^foo/bar.min^,
\verb^foo/bar.sec^, \verb^foo/bar.micros^, \verb^moo/baa.lat^,
\verb^moo/baa.lon^, \verb^moo/baa.elev^.

\newpage
\section{RDC endpoints}

\begin{bulletlist}

\item \verb~register(^event)~

\begin{verbatim}
@event
{
   txt address
   flg arrived
}
\end{verbatim}

Deregistration is achieved by setting the \verb^arrived^ flag to
false, indicating that the component is departing.

\item \verb~lost(^event)~

The \verb^arrived^ flag is ignored by this call, and should be set to zero.

\item \verb~lookup(^criteria) : @results ( txt address )~

\begin{verbatim}
@criteria
{
   ^map-constraints -
   ^interface -
}              
@map-constraints
{
   [ txt cpt_name ]
   [ txt instance_name ]
   [ txt creator ]
   [ txt pub_key ]
   keywords ( txt keyword )
   parents ( txt cpt )
   ancestors ( txt cpt )
   flg match-endpoint-names
}
@interface
(
   endpoint
   {
      [ txt name ]
      ^endpoint-type type
      txt msg-hash + reply-hash
   }
)
\end{verbatim}

\item \verb~list() : ^cpt-list~

\begin{verbatim}
@cpt-list
(
   component
   {
      txt address
      txt cpt-name
      [ txt instance ]
   }
)
\end{verbatim}

\item \verb~cached_metadata(@txt address) : ^component-metadata~

\item \verb~cached_status(@txt address) : ^component-state~

\item \verb~dump() : ^rdc-dump~

\begin{verbatim}
@rdc-dump
(
   component
   {
      txt address
      ^component-metadata metadata
      ^component-state state
   }
)
\end{verbatim}

Use of this endpoint is discouraged because it will amount to a lot of
data; it is usually better to be selective and use a couple of
RPC's to enquire about a specific component.

The \verb^dump()^ call is provided anyway so that RDC's can
synchronise after restarting; in this case they do need to know
everything, and packaging it as a single RPC is more efficient
than a large number of calls to fetch the information about
every component.

\item \texttt{events} -- source

\begin{verbatim}
@event
{
   txt address
   flg arrived
}
\end{verbatim}

A stream of events which signify when new components
connect, components migrate, and old components disconnect.
Replication of RDCs works by connecting the \verb^events^ stream
of each to the \verb^register^ sink on the others.

%\item \texttt{metadata\_events} \textit{source}
%
%A stream of events containing all changes to the
%components' metadata, and new component definition.
%
%\item \verb=create_component(^component-metadata -)=
%\textit{sink}
%
%\item \verb=update_metadata(txt cpt_name, ^component-metadata data,=\\
%\verb=private_key(cpt_name, data))=
%\textit{sink}
	
\end{bulletlist}

\appendix
\newpage
\section{LITMUS quick reference}

\begin{tabular}{ll}
\verb^int name^                    & Integer\\
\verb^dbl name^                    & Double-precision floating point\\
\verb^flg name^                    & Flag (boolean)\\
\verb^txt name^                    & Text string\\
\verb^bin name^                    & Binary data\\
\verb^clk name^                    & Date and time (clock)\\
\verb^loc name^                    & Location\\
\verb^name { elt1 elt2... eltN }^  & Structure\\
\verb^name ( elt )^                & List, zero or more occurrences of \verb^elt^\\
\verb^name (+ elt )^               & List, one or more occurrences of \verb^elt^\\
\verb^name (N elt )^               & Array, N occurrences of \verb^elt^\\
\verb^[ elt ]^                     & Optional\\
\verb^< elt1 elt2... eltN >^       & Choice\\
\verb^name < #val1 #val2... #valN >^ & Enumeration\\
\verb^@elt^                        & Type definition (top-level name becomes type's label)\\
\verb~^label name~                 & Type reference\\
\verb^name1 + name2... + nameN^    & Multiple declaration\\
\verb^-^                           & Unnamed element\\
\verb^* Foo bar^                   & Comment\\
\verb^@"filename"^                 & Import types from a file\\
\end{tabular}

\end{document}
