\documentclass[12pt,a4paper,twoside]{article}
\usepackage{fancyhdr,graphicx,latexsym}

\setlength{\parindent}{0cm}
\setlength{\parskip}{2ex plus1ex minus 0.5ex}

\addtolength{\evensidemargin}{-2.5cm}
\addtolength{\oddsidemargin}{-0.5cm}
\addtolength{\textwidth}{3cm}

\addtolength{\headheight}{0.2cm}
\addtolength{\topmargin}{-1cm}
\addtolength{\textheight}{2.5cm}

\renewcommand{\_}{\texttt{\symbol{95}}}
\addtolength{\fboxsep}{0.1cm}
\newcommand{\param}[1]{\textit{\textrm{\textmd{#1}}}}
\newcommand{\codebar}{\rule{\textwidth}{0.3mm}}
% \newcommand{\spc}{\hspace{0.5mm}$\sqcup$\hspace{0.6mm}}
\newcommand{\spc}{\hspace{0.5mm}$\Box$\hspace{0.5mm}}
\newcommand{\todo}[1]{\textbf{TODO: #1}}

\newlength{\codelen}
\newcommand{\code}[1]
{\begin{center}\fbox{\parbox{16cm}{\texttt{#1}}}\end{center}}

\fancyhead{}
\fancyhead[RO,LE]{\thepage}
\fancyhead[LO,RE]{SBUS Component Metadata and the RDC}
\fancyfoot{}
\pagestyle{empty}

\newenvironment{bulletlist}
{
	\begin{itemize}
	\setlength{\itemsep}{0ex}
	\setlength{\parsep}{0ex}
}
{
	\end{itemize}
}

\newenvironment{alphalist}
{
	\begin{enumerate}
	\setlength{\itemsep}{0ex}
	\setlength{\parsep}{0ex}
	\renewcommand{\labelenumi}{(\alph{enumi})}
}
{
	\end{enumerate}
}

\newenvironment{numericlist}
{
	\begin{enumerate}
	\setlength{\itemsep}{0ex}
	\setlength{\parsep}{0ex}
}
{
	\end{enumerate}
}

\begin{document}

% \sfseries
\centerline{\textbf{\LARGE SBUS Component Metadata and the RDC}}
\begin{center} \large
David Ingram\\
TIME-EACM Project\\
University of Cambridge Computer Lab\\
27th September 2008\\
\end{center}

{ \parskip 1mm plus 1pt \tableofcontents }
\pagestyle{fancy}

\section{Component metadata format}

The metadata is split into static (class properties) and dynamic
(instance properties). Only the status of one instantiation is
accessible by contacting a component itself, whereas the RDC
collates status information on all instances of a component.

%Authorised modifications (which require the private key used at
%time of original component definition) to the metadata may add new SAP's or
%new versions of existing SAP's API, or change the administrative fields.

\newpage
\subsection{Static metadata}
\label{metadata}

\begin{verbatim}
@component-metadata
{
   txt name
   txt description
   keywords ( txt keyword )
   txt designer
   txt pubkey
   managed-roles ( ^role-defn - )
   remap-permission ( ^role - )
   eps
   (
      endpoint
      {
         txt name
         ^endpoint-type type
         ^api -
         alternate ( ^api - )
         obsolete ( ^api - )
      }
   )
}

@api
{
   access-control ( ^role - )
   ^idl message + response
}

@role
{
   txt name
   txt issuer
   txt pubkey
}

@role-defn
{
   txt name
   ^idl request
}

@idl txt
@endpoint-type < #client #server #source #sink >
\end{verbatim}

\subsubsection{Administrative fields}

\begin{bulletlist}
\item \verb^name^ -- Component name.
\item \verb^description^ -- Explanation of what the component does, for
		human consumption.
\item \verb^keywords^ -- Keywords which can be used when searching for
		components.
\item \verb^designer^ -- Name of the person who created the component metadata.
\end{bulletlist}

\subsubsection{Endpoint fields}

\begin{bulletlist}
\item \verb^endpoint^ -- Describes one endpoint. Note that clones
		are not part of the static metadata.
\item \verb^api^ -- The primary API for this endpoint.
\item \verb^alternate^ and \verb^obsolete^ -- Alternative API's provided
		for the same endpoint. Those in the alternate list can be used
		instead of the primary one, for example by older clients.
		The obsolete list contains previous API's which are no longer
		supported, and acts as a historic reference for the schema evolution.
		Note: neither of these fields are used at present.
\item \verb^idl^ -- An alias for a text field, when it is used to
		store LITMUS IDL text.
\item \verb^message^ and \verb^response^ -- Describes the format of
		messages sent to and from this endpoint.
		Sink and source endpoints have no response (schema \verb^"!"^).
		Some RPCs may have an empty payload for either message or response
		(schema \verb^"0"^).
\end{bulletlist}

\subsubsection{Access control fields}

\textbf{Note:} access control is unimplemented, so all of the fields in this
section are currently unused (but must still be present).

\begin{bulletlist}
\item \verb^pubkey^ -- Public key for authentication of component implementors.
%\item \verb^signature^ -- fingerprint of rest of a message
%		signed with private key; used for update messages (not stored in RDC).
\item \verb^managed-roles^ -- Defines roles managed by this component,
		and provides standard access points to request certificates for each one.
\item \verb^remap-permission^ -- Lists the roles needed to dynamically remap
		connections to a running instance of this component.
\item \verb^access-control^ -- A list of roles required to connect to this
		endpoint.
\item \verb^issuer^ -- The name of the component which manages this role.
\item \verb^request^ -- A built-in access point used to request
		that a role managed by this component be granted to the client,
		based on evidence presented in the message. This access point always
		has an access type of RPC.
\end{bulletlist}

\newpage
\subsection{Dynamic metadata}
\label{status}

\begin{verbatim}
@component-state
{
   txt address
   txt instance
   txt creator
   txt cmdline
   load { int cpu + buffers }
   ^endpoints -
   [ ^freshness - ]
}

@endpoints
(
   endpoint
   {
      txt name
      int processed + dropped * Message counts
      [ txt subscription ] [ txt topic ]
      peers ( ^peer - )
   }
)

@peer
{
   txt cpt_name
   txt instance
   txt endpoint
   txt address * Host and port
   [ txt subscription ] [ txt topic ]
   int lastseq
   int latency
}

@freshness
{
   int last_ping_secs
   int latency
}
\end{verbatim}

\newpage
\subsubsection{Dynamic fields}

\begin{bulletlist}
%\item \verb^current^ -- This group contains status blocks
%      for all the instantiations of the component which are believed
%      to be currently alive (i.e. responded to their last ping).
%		There will be more than one of these if the component is
%		replicated.
%\item \verb^recent^ -- This section describes previous instantiations.
\item \verb^address^ -- Host IP / DNS name, and port number (separated by
		a colon).
\item \verb^instance^ -- Instance name of this component instantiation.
\item \verb^creator^ -- The person running this instance (and possibly
		responsible if it needs restarting...)
\item \verb^cmdline^ -- Command line by which the component was invoked.
\item \verb^cpu^ --
	current total CPU usage on the machine, as a percentage from 0 to 100.
\item \verb^buffers^ -- how much memory (in KB) is being used for
	buffering messages that are awaiting flow control to clear.
\item \verb^name^ -- The endpoint name.
		Cloned endpoints are identified as
		\verb^epname#clone^, where \verb^epname^ is the endpoint
		name and \verb^clone^ is the clone number.
\item \verb^processed^ -- contains a count of all messages
	processed by the endpoint.
\item \verb^dropped^ -- counts all the messages which have been dropped
	by the component for flow control reasons.
\item \verb^peers^ --  Other components mapped to this one.
		\verb^lastseq^ is the maximum sequence number used.
      Indirect data sources can be inferred by chasing through the
		\verb^peers^ of different components at run-time
		(limited to a list of length 100, using breadth-first search,
		in order to conserve RDC CPU).
		\verb^latency^ is the most recently measured latency for this endpoint
\item \verb^last_ping_secs^ -- The time the component was last pinged by a RDC.
\item \verb^latency^ -- The currently estimated round-trip time for this
	component's \verb^get_status()^ endpoint, in microseconds.
\end{bulletlist}

\newpage
\section{Built-in endpoints}
\label{builtin}

All components implement these. This is done for you by the wrapper.
Components which are pure client still support all these,
although some of the information is not relevant.
This allows administration tools to crawl the graph of components and
clients, even without any widespread use of shared brokers. This graph
is a generalisation of the component list functionality possible in
SCOP and other centralised systems.

\begin{itemize}

\item \verb^get_metadata()^

Returns a \verb^component-metadata^ object (see Section \ref{metadata}
for the corresponding IDL). This is useful for components
which are not registered with a RDC (if they are, this information
is also available from the component definition in the RDC).

\item \verb^get_status()^

Returns a \verb^component-state^ object (see Section \ref{status}
for the corresponding IDL. The \verb^freshness^
fields are ignored). The RDC calls this
method periodically, and also uses the response time to
measure latency.

\item \verb^lookup_schema(txt hashcode)^

Returns the schema (as a plain text field)
corresponding to the given hashcode. This is useful for polymorphic
components which send messages with different types from any of
those defined in the component metadata itself. The schemas of
such messages are held in a cache by the component.

\item \verb^map(txt endpoint, txt peer_address, txt peer_endpoint, txt certificate)^

Tries to map the specified endpoints to a given peer. This mapping
is in addition to any already set up.

\item \verb^unmap(txt endpoint, txt certificate)^

Unmaps all peers from the specified endpoint.

\item \verb^subscribe(txt endpoint, txt peer, txt subscription, txt topic)^

This changes the default subscription for new maps if \verb^peer^ is
NULL, or the current subscription for a specific peer (component or
instance name), or the current subscription for all presently mapped
peers (if \verb^peer^ is \verb^*^).

\item \verb^divert(txt endpoint, txt new_address, txt new_endpoint, txt certificate)^

This causes divert messages to be sent to all peers currently
connected to the endpoint, requesting them to remap to the
named target instead of the endpoint in question.

If all goes well then some time after the divert request there will be no
peers mapped to the endpoint, except for any new arrivals after
the divert was called.

\item \verb^terminate()^

Instructs the component to shut down.

% Called by the RDC or other component with sufficient authority.
% The RDC uses it when a new instance of a component which is
% supposed to be unique is created, and it has requested that
% the older instance be overridden.

\item \verb^set_log_level(int log_level, int echo_level)^

Changes the log levels for the component. The levels are a sum of
1 for errors, 2 for warnings, 4 for messages and 8 for debugging.

% Errors = 1, Load statistics = 2, Connections = 4, Actions = 8.

\end{itemize}

% \subsubsection*{Expected future endpoints}
% 
% \begin{itemize}
% 
% \item \verb^ping()^
%
% Called by the RDC periodically. Returns an empty acknowledgement
% immediately and is used to measure round-trip times.
%
% \item \verb^migrate(txt new_address)^
% 
% A request to initiate component migration.
% 
% \end{itemize}

\newpage
\section{Resource discovery component (RDC)}
\label{rdc}

\begin{bulletlist}
\item Provides component discovery: listing components which exist,
		retrieving and updating metadata \& addressing information
\item Special service at a well known address
\item Implemented as a component itself
\item Security checks for adding completely new components very minimal
	(anti-DOS only)
\item Contact service to update location (e.g. when component port numbers
	change)
\end{bulletlist}

\subsection{Synposis}

\verb^rdc [ <option> ... ]^

Options:

\begin{tabular}{ll}
\verb^-a^ & Use any port (rather than the default one)\\
\verb^-p <port>^ & Try to use the supplied port number\\
\verb^-j <address>^ & Join (replicate) another RDC\\
\verb^-bg^ & Run in background\\
\verb^-f <file>^ & Join all RDCs listed in file, one address per line\\
\end{tabular}

Note: you may include more than one \verb^-j^ option.
Addresses of RDCs to replicate must be in one of the forms
\verb^hostname:port^, \verb^IP-address:port^, \verb^:port^,
\verb^hostname^ or \verb^IP-address^.
If you specify an address without a port number then the default is used.
If an RDC to join with is not running it is silently skipped
(so it is safe to specify a file of many RDCs, some of which \textit{might}
be running).

\subsection{Status reporting -- tracking liveness}

The RDC maintains information on the liveness of all components it
knows about. Components are not required to keep a TCP connection to
the RDC open for this purpose; instead the RDC pings them peridiocally.
This conserves port resources on the RDC in large deployments, but
limits the freshness of the liveness information.

The status of each component is collated by the resource
discovery component using the \verb^get_status()^ standard SAP.
It then disconnects from the component immediately.

Note that this only establishes network connectivity and proper
functioning of the component wrapper and does not guarantee any data
sources themselves are alive; however any internal failures are the
component's responsibility.

The RDC publishes a stream of updates to liveness information on its
\verb^events^ endpoint.

%If there are no currently live instantiations of a given component its
%``last address known at'' is stored in the \verb^recent^ status
%block in the metadata.

\subsubsection*{Use of liveness for mutual exclusion (Unimplemented)}

When a new component starts up and specifies the \verb^UNIQ_SINGLE^
uniqueness flag, the RDC must establish whether any other instances
are running at that moment. It does this by pinging any marked
as currently live, to check they have not disappeared since the
last ping time.

\subsection{API}

\begin{bulletlist}

\item \verb~register(^event)~

\begin{verbatim}
@event
{
   txt address
   flg arrived
}
\end{verbatim}

Deregistration is achieved by setting the \verb^arrived^ flag to
false, indicating that the component is departing.

\item \verb~lost(^event)~

The \verb^arrived^ flag is ignored by this call, and should be set to zero.

\item \verb~lookup(^criteria) : @results ( txt address )~

\begin{verbatim}
@criteria
{
   ^map-constraints -
   ^interface -
}              
@map-constraints
{
   [ txt cpt_name ]
   [ txt instance_name ]
   [ txt creator ]
   [ txt pub_key ]
   keywords ( txt keyword )
   parents ( txt cpt )
   ancestors ( txt cpt )
   flg match-endpoint-names
}
@interface
(
   endpoint
   {
      [ txt name ]
      ^endpoint-type type
      txt msg-hash + reply-hash
   }
)
\end{verbatim}

\item \verb~list() : ^cpt-list~

\begin{verbatim}
@cpt-list
(
   component
   {
      txt address
      txt cpt-name
      [ txt instance ]
   }
)
\end{verbatim}

\item \verb~cached_metadata(@txt address) : ^component-metadata~

\item \verb~cached_status(@txt address) : ^component-state~

\item \verb~dump() : ^rdc-dump~

\begin{verbatim}
@rdc-dump
(
   component
   {
      txt address
      ^component-metadata metadata
      ^component-state state
   }
)
\end{verbatim}

Use of this endpoint is discouraged because it will amount to a lot of
data; it is usually better to be selective and use a couple of
RPC's to enquire about a specific component.

The \verb^dump()^ call is provided anyway so that RDC's can
synchronise after restarting; in this case they do need to know
everything, and packaging it as a single RPC is more efficient
than a large number of calls to fetch the information about
every component.

\item \texttt{events} -- source

\begin{verbatim}
@event
{
   txt address
   flg arrived
}
\end{verbatim}

A stream of events which signify when new components
connect, components migrate, and old components disconnect.

%\item \texttt{metadata\_events} \textit{source}
%
%A stream of events containing all changes to the
%components' metadata, and new component definition.
%
%\item \verb=create_component(^component-metadata -)=
%\textit{sink}
%
%\item \verb=update_metadata(txt cpt_name, ^component-metadata data,=\\
%\verb=private_key(cpt_name, data))=
%\textit{sink}
	
\end{bulletlist}

\subsection{Replication of the RDC}

The RDC is not on any component's data path, and hence the system can
run for some time without it. It is however important for setting up
new connections, and as such forms a single point of failure. The
machine on which it is running may well be rebooted like any other, hence
to avoid major denial of service it is necessary to replicate the RDC.
RDCs never migrate, but additional replicas can be started and stopped.

Each RDC has a list of other RDCs it knows about. Typically there
will be two or three and they will form a group which know about each
other, but in theory you can have any number. The other RDCs to join
with are specified on the command line when the RDC is started.
It is not currently possible to add a new entry to the known list
for a currently running RDC. The RDC addresses given are only
``possible'' locations, however (they are simply skipped if not
running at the time), so it is safe to supply an overly-inclusive file
listing all likely RDCs which you may wish to join.

A group of RDCs attempt to mirror the information received by any of
them as closely as possible. This is easy to do by listening to the
\verb^events^ streams of all the
other RDCs (hence there's no need for a special inter-RDC protocol).
When a RDC restarts it calls the \verb^dump()^ method of the
other RDCs it knows about, in order to update its stored data.

\subsection{Persistence (Draft -- Not implemented yet)}

RDCs commit changes to disc immediately on receipt. All data is stored
in a single append-only logfile. Each change is a separate entry;
complete snapshots are also written to the log periodically. The
logfile does not have read or write permission for ordinary users. If
the RDC is killed whilst writing to the log, there will be an
incomplete entry at the end of the log. This will be detected and
truncated next time it restarts.

The log serves two purposes: reloading state after the RDC is stopped
and restarted, and insurance against someone getting control of a RDC
and deleting or modifying all the components. In the latter case after
the problem is detected the original data could be reinstated by
examining the log, since it is append-only (apart from truncation of
partially-written entries).

Each entry in the log is in the form of a complete XML message.
Entries are separated by a blank line (two newlines in a row)
for easy detection. When the RDC restarts it scans backwards for
the most recent snapshot, loads that, then replays the updates
added after the snapshot upto the end of the log. A special
restart message is then logged to indicate the times at which
the RDC was restarted. A complete snapshot is logged once per week
(i.e. when 168 hours of real time have passed since the timestamp
on the last snapshot).

The types of entry are therefore snapshots, restart messages,
and registration/deregistration events. All types are formatted
in XML and have a timestamp.

\subsection{Schema repository (Unimplemented)}

The RDC also performs the function of persistently storing all
schemas which it has seen in the past, together with their
hash codes. It provides the same service as other components
of returning the text of a schema given a hash code lookup.
The difference with a RDC is that it stores them persistently.

\end{document}
