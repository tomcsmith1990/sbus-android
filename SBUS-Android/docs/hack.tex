\documentclass[12pt,a4paper,twoside]{article}
\usepackage{fancyhdr,graphicx,latexsym}

\setlength{\parindent}{0cm}
\setlength{\parskip}{2ex plus1ex minus 0.5ex}

\addtolength{\evensidemargin}{-2.5cm}
\addtolength{\oddsidemargin}{-0.5cm}
\addtolength{\textwidth}{3cm}

\addtolength{\headheight}{0.2cm}
\addtolength{\topmargin}{-1cm}
\addtolength{\textheight}{2.5cm}

\renewcommand{\_}{\texttt{\symbol{95}}}
\addtolength{\fboxsep}{0.1cm}
\newcommand{\param}[1]{\textit{\textrm{\textmd{#1}}}}
\newcommand{\codebar}{\rule{\textwidth}{0.3mm}}
% \newcommand{\spc}{\hspace{0.5mm}$\sqcup$\hspace{0.6mm}}
\newcommand{\spc}{\hspace{0.5mm}$\Box$\hspace{0.5mm}}
\newcommand{\todo}[1]{\textbf{TODO: #1}}

\newlength{\codelen}
\newcommand{\code}[1]
{\begin{center}\fbox{\parbox{16cm}{\texttt{#1}}}\end{center}}

\fancyhead{}
\fancyhead[RO,LE]{\thepage}
\fancyhead[LO,RE]{SBUS Hackers Guide}
\fancyfoot{}
\pagestyle{empty}

\newenvironment{bulletlist}
{
	\begin{itemize}
	\setlength{\itemsep}{0ex}
	\setlength{\parsep}{0ex}
}
{
	\end{itemize}
}

\newenvironment{alphalist}
{
	\begin{enumerate}
	\setlength{\itemsep}{0ex}
	\setlength{\parsep}{0ex}
	\renewcommand{\labelenumi}{(\alph{enumi})}
}
{
	\end{enumerate}
}

\newenvironment{numericlist}
{
	\begin{enumerate}
	\setlength{\itemsep}{0ex}
	\setlength{\parsep}{0ex}
}
{
	\end{enumerate}
}

\begin{document}

% \sfseries
\centerline{\textbf{\LARGE SBUS Hackers Guide}}
\begin{center} \large
David Ingram\\
TIME-EACM Project\\
University of Cambridge Computer Lab\\
20th February 2008\\
\end{center}

{ \parskip 1mm plus 1pt \tableofcontents }
\pagestyle{fancy}

\vspace{5mm}
This documents some of the SBUS internals for hackers.
Ordinary developers who wish to create application components
shouldn't need this -- only people who wish to port the library
to another language binding, or to modify the middleware itself.

\section{Directory structure familiarisation}

The SBUS archive includes subdirectories called \verb^library/^,
\verb^wrapper/^, \verb^tools/^, \verb^utils/^,
\verb^docs/^, \verb^idl/^ and \verb^examples/^. The \verb^library/^
subdirectory contains the source code for the library,
\verb^wrapper/^ contains the wrapper program and an additional library
with functionality such as schema parsing nor needed by the component
library, the \verb^tools/^ subdirectory contains pre-built components
including the RDC, event broker and demo programs, and \verb^utils/^
contains utilities for testing the system and analysing component
metadata.

Documentation can be found in \verb^docs/^.
An general description of the SBUS system is in \verb^docs/overview.ps^,
and it is recommended to read the overview document first.
Changes since the last release are in \verb^docs/Changelog^.

The \verb^idl/^ directory contains component metadata (\verb^.cpt^ files) and
some important individual schemas (\verb^.idl^ files). Metadata for the sample
component is in \verb^demo.cpt^ and the master schema in
\verb^cpt_metadata.idl^. The master schema is a LITMUS schema which describes
the component metadata format itself.

The \verb^examples/^ directory contains
some sample schemas in separate files, together with typical
messages adhering to those schemas in the XML encoding.
There is also one example of a series of messages stored in
the persistent data file format, in \verb^bus_stream.xml^.

\section{Tests}

The \verb^utils/^ directory contains a number of test programs,
including:
\begin{bulletlist}
\item \verb^checkimport^ -- tests the XML import and export function
\item \verb^checkvalidate^ -- validates an XML message against a schema
\end{bulletlist}
If run with no arguments they all print their usage information.
There are also two scripts, called \verb^runtests^ and \verb^regress^.
The former runs the various test programs on data from the
\verb^examples/^ directory. The \verb^regress^ script automates
this further by calling \verb^runtests^ for you and comparing the
output with that in the file \verb^correct_output^.

\section{Changing IDL for built-ins}

\section{Major classes}

\section{Protocol message to net class mapping}

% \section{Wrapper implementation}
% 
% The library called by the component has no threads.
% It talks to the wrapper synchronously via one pipe per endpoint.
% Wrappers talk to other wrappers via one TCP connection per endpoint
% (in the future all links between the same pair of components may
% share one connection).
% 
% \section{Creating a pure-client component}
% 
% Even applications which act only as SBUS clients and sinks need an
% \verb^scomponent^ object. They merely do not provide any services.
% 
% Strictly speaking a pure-client component need not be \textit{registered}
% with a RDC as most components are; however RDC addresses are specified
% in order to look up other components anyway, so the client component
% usually does get registered somewhere.
% The client component must of course still supply a
% name of some sort for itself, which is useful to improve the clarity of the
% monitoring utilities when labelling connections.

\section{Sequence of data transformation operations}

The following sequences are illustrative of the typical data
transformations performed by SBUS. Most of these are done for
you by the library routines in practice. For each step the types of
the input data and the result are given.

\subsection*{Message composition and send/receive sequence:}

\begin{tabular}{l|l|l|l}
Stage & Inputs & Outputs & Function\\
\hline
1. Schema parse: & LITMUS text & \verb^Schema *^ & \verb^Schema::load()^\\
2. Pack:         & native types & \verb^snode *^ & \verb^pack()^\\
3. Marshall:     & \verb^snode *, Schema *^ & binary data & \verb^marshall()^\\
4. Transmit      & binary data & binary data & \verb^sendpoint::*()^\\
5. Unmarshall:   & binary data, \verb^Schema *^ & \verb^snode *^ (or error) &
\verb^unmarshall()^\\
6. Extract:      & \verb^snode *^ & native types & \verb^snode::extract()^\\
\end{tabular}

\subsection*{Message import, send/receive and export sequence:}

\begin{tabular}{l|l|l|l}
Stage & Inputs & Outputs & Function\\
\hline
1. Schema parse: & LITMUS text & \verb^Schema *^ & \verb^Schema::load()^\\
2. Import:       & XML text & \verb^snode *^ & \verb^snode::import()^\\
3. Marshall:     & \verb^snode *, Schema *^ & binary data & \verb^marshall()^\\
4. Transmit      & binary data & binary data & \verb^sendpoint::*()^\\
5. Unmarshall:   & binary data, \verb^Schema *^ & \verb^snode *^ (or error) &
\verb^unmarshall()^\\
6. Export:       & \verb^snode *^ & XML text & \verb^snode::toxml()^\\
\end{tabular}

\subsection*{Message import and usage sequence:}

\begin{tabular}{l|l|l|l}
Stage & Inputs & Outputs & Function\\
\hline
1. Schema parse: & LITMUS text & \verb^Schema *^ & \verb^Schema::load()^\\
2. Import:       & XML text & \verb^snode *^ & \verb^snode::import()^\\
3. Validate:     & \verb^snode *, Schema *^ & \verb^bool^
      (+ disambiguates) & \verb^validate()^\\
4. Extract:      & \verb^snode *^ & native types & \verb^snode::extract()^\\
\end{tabular}

\subsection*{Other operations:}

\begin{tabular}{l|l|l|l}
Operation & Inputs & Outputs & Function\\
\hline
Express: & \verb^const char *subs^ & \verb^subscription *^ &
\verb^new subscription^\\
Match: & \verb^snode *, subscription *^ & \verb^bool^ &
\verb^subscription::match()^\\
Translate: & \verb^Schema *^ & RELAXNG & \textbf{TODO}\\
\end{tabular}

\subsection*{Marshalling functions}

\verb^snode^s are converted to and from a serialised binary
representation suitable for sending over the wire as follows:

\begin{verbatim}
unsigned char *buf = marshall(snode *s, Schema *schema, int *length,
   const char **err);
snode *s = unmarshall(const unsigned char *buf, int length, Schema *schema,
   const char **err);
\end{verbatim}

\verb^marshall^ returns binary data. The number of bytes generated is
returned in \verb^length^. Both functions return NULL in case of error,
and place an error message into \verb^err^.

\end{document}
