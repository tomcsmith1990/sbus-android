\documentclass[12pt,a4paper,twoside]{article}
\usepackage{fancyhdr,graphicx,latexsym}

\setlength{\parindent}{0cm}
\setlength{\parskip}{2ex plus1ex minus 0.5ex}

\addtolength{\evensidemargin}{-2.5cm}
\addtolength{\oddsidemargin}{-0.5cm}
\addtolength{\textwidth}{3cm}

\addtolength{\headheight}{0.2cm}
\addtolength{\topmargin}{-1cm}
\addtolength{\textheight}{2.5cm}

\renewcommand{\_}{\texttt{\symbol{95}}}
\addtolength{\fboxsep}{0.1cm}
\newcommand{\param}[1]{\textit{\textrm{\textmd{#1}}}}
\newcommand{\codebar}{\rule{\textwidth}{0.3mm}}
% \newcommand{\spc}{\hspace{0.5mm}$\sqcup$\hspace{0.6mm}}
\newcommand{\spc}{\hspace{0.5mm}$\Box$\hspace{0.5mm}}
\newcommand{\todo}[1]{\textbf{TODO: #1}}

\newlength{\codelen}
\newcommand{\code}[1]
{\begin{center}\fbox{\parbox{16cm}{\texttt{#1}}}\end{center}}

\fancyhead{}
\fancyhead[RO,LE]{\thepage}
\fancyhead[LO,RE]{SBUS Extensions}
\fancyfoot{}
\pagestyle{empty}

\newenvironment{bulletlist}
{
	\begin{itemize}
	\setlength{\itemsep}{0ex}
	\setlength{\parsep}{0ex}
}
{
	\end{itemize}
}

\newenvironment{alphalist}
{
	\begin{enumerate}
	\setlength{\itemsep}{0ex}
	\setlength{\parsep}{0ex}
	\renewcommand{\labelenumi}{(\alph{enumi})}
}
{
	\end{enumerate}
}

\newenvironment{numericlist}
{
	\begin{enumerate}
	\setlength{\itemsep}{0ex}
	\setlength{\parsep}{0ex}
}
{
	\end{enumerate}
}

\begin{document}

% \sfseries
\centerline{\textbf{\LARGE SBUS Extensions}}
\begin{center} \large
David Ingram\\
TIME-EACM Project\\
University of Cambridge Computer Lab\\
26th September 2007\\
\end{center}

{ \parskip 1mm plus 1pt \tableofcontents }
\pagestyle{fancy}

\vspace{1cm}
The following extensions to the system are planned, but not yet implemented.
See also the SBUS Prebuilt Components document, for features which can
be provided as components (without changing the library or wrapper).

\section{Partial message matching}

This is a new feature of the wrapper. It allows
components to specify that an endpoint matches any type whose tree
\textit{covers} the specified schema. Any node in the tree may have
additional branches, but the required schema must be there as a minimum
and based at the root. This will eventually become the default
schema matching rule (the present one will have to be selected
explicitly as ``exact matches only''). The purpose of this is to enable
automatic support for nondestructive schema evolution, in which new
fields are added to existing message types without breaking the
old clients or sinks.

See the sub-tree matching feature below for details of the implementation.

\section{Sub-tree message matching}

This is another new feature of the wrapper. It allows components to specify
that an endpoint matches any schema type which contains a required sub-tree.
This is useful for writing more generic components which can process
any message containing a certain specific piece of information, such
as a timestamp, location, or well-known type.

On receiving a previously unseen
message hash code, the \verb^lookup_schema^ endpoint of the peer component
is used to retrieve the full schema for that code.
The schema is parsed and the tree is
searched for the sub-tree required. The result (either failure,
or the position of the sub-tree) is then cached for that message
hash code.

When messages arrive, the requested sub-tree extracted and
returned to the client, but the rest of the message is retained
for later reintegration (for example if the component wishes to
change the sub-tree but then retransmit the whole message somewhere
else).

\section{Access control}

An access control handshake occurs when a new TCP connection is
opened to a connection, before any data messages are sent.
This establishes which roles the client holds. For example,
they may only possess the roles required for some subset of the endpoints
available.
Access control is possible even if the connection is unsecured (no
SSL link).

Roles are controlled by components.
The certificate format is standardised as follows:

\begin{verbatim}
@certificate
{
   txt client
   txt issuer
   txt role-name
   ^access-type -
   clk expiry
   txt signature
}

@access-type < #private #redistributable #republishable >
\end{verbatim}

The \verb^signature^ is a digital signature of the remainder of the
structure. \verb^issuer^ is the name of the component which issued
this certificate. (\verb^issuer^, \verb^role-name^) is unique
up to multiple components with the same name (the signature
proves the intended issuer was used).

Clients must possess all of the roles required by a given stream.
Normally (but not always) when multiple streams are combined by one
component, any stream it republishes should require the combination
of the requisite roles.

Role membership has an associate flag with three states:
\begin{bulletlist}
\item \texttt{private:} must not disclose stream data in any form
\item \texttt{redistributable:} may redistribute stream providing the same
	roles are checked
\item \texttt{republishable:} may republish data derived from the stream
	and allow weaker sets of roles to view this
\end{bulletlist}

\subsection*{Access control for remapping}

Third parties must possess the roles in the \verb^remap_permission^
element of the component metadata.

\section{Potential components}

Potential components do not exist until required, at which point
they are dynamically created. This conserves server resources.
They can be instantiated based on remote requests.

Stateless components can be destroyed and then
recreated on demand.

Potential components are particularly useful for output filters,
such as a component which emits events when a data value changes
more than some delta, which need to exist close to the
source component in order to effectively quench messages and
save network bandwidth.

There are likely to be a very large number of possible potential
components based on the combinations of filters and so on.
It is therefore too expensive to keep them all running all the
time as ordinary components.

\section{Restarting processes}

This is related to potential components.
SBUS does not automatically restart crashed/killed processes for you.
It is possible to use boot scripts and \texttt{cron} for this, of course.

Consider adding automatic restarting via ssh, new \texttt{exe\_path} metadata,
and activation mode = Manual $|$ Always $|$ OnDemand.

\section{Security}

Build OpenSSL into the standard implementation.

Either protocol negotiation and/or metadata flags needs to specify
or negotiate whether SSL is to be used (at a performance cost)
or not.

Support session resumption for higher performance, especially for
server pings.

If SSL isn't being used for confidentiality we still need
authorisation for access control, by encrypting a certificate or
password using SSL or a payload encryption library.

External data sinks consist of an ordinary stream sink + demux
component at open firewall port + encryption. Of course, some internal
client machines may need these security features anyway.

\section{Option blocks}

This is a new LITMUS schema syntax:

\begin{verbatim}
name
{?
   contents
   etc...
}
\end{verbatim}

An option block is the same as a structure, but begins with
\verb^{?^ instead of \verb^{^. The difference is that all
elements of the structure are optional. The effect is the
same as writing the following:

\begin{verbatim}
name
{
   [ contents ]
   [ etc... ]
}
\end{verbatim}

\section{Component creation tool}

Component metadata must be created and uploaded to a running RDC (in most
cases) or stored in a file (for overlay networks etc) before you can
instantiate a component for the first time.

The \texttt{cptedit} tool can help you do this, or you can enter a complete
XML component description with any text editor.

\texttt{cptedit} facilities (not implemented yet!):
\begin{bulletlist}
\item Fields to edit metadata
\item Ability to create new components
\item Displays endpoint fields
\end{bulletlist}

\end{document}
