\documentclass[12pt,a4paper,twoside]{article}
\usepackage{fancyhdr,graphicx,latexsym}

\setlength{\parindent}{0cm}
\setlength{\parskip}{2ex plus1ex minus 0.5ex}

\addtolength{\evensidemargin}{-2.5cm}
\addtolength{\oddsidemargin}{-0.5cm}
\addtolength{\textwidth}{3cm}

\addtolength{\headheight}{0.2cm}
\addtolength{\topmargin}{-1cm}
\addtolength{\textheight}{2.5cm}

\renewcommand{\_}{\texttt{\symbol{95}}}
\addtolength{\fboxsep}{0.1cm}
\newcommand{\param}[1]{\textit{\textrm{\textmd{#1}}}}
\newcommand{\codebar}{\rule{\textwidth}{0.3mm}}
% \newcommand{\spc}{\hspace{0.5mm}$\sqcup$\hspace{0.6mm}}
\newcommand{\spc}{\hspace{0.5mm}$\Box$\hspace{0.5mm}}

\newlength{\codelen}
\newcommand{\code}[1]
{\begin{center}\fbox{\parbox{16cm}{\texttt{#1}}}\end{center}}

\fancyhead{}
\fancyhead[RO,LE]{\thepage}
\fancyhead[LO,RE]{SBUS Data Examples}
\fancyfoot{}
\pagestyle{empty}

\newenvironment{bulletlist}
{
	\begin{itemize}
	\setlength{\itemsep}{0ex}
	\setlength{\parsep}{0ex}
}
{
	\end{itemize}
}

\newenvironment{alphalist}
{
	\begin{enumerate}
	\setlength{\itemsep}{0ex}
	\setlength{\parsep}{0ex}
	\renewcommand{\labelenumi}{(\alph{enumi})}
}
{
	\end{enumerate}
}

\newenvironment{numericlist}
{
	\begin{enumerate}
	\setlength{\itemsep}{0ex}
	\setlength{\parsep}{0ex}
}
{
	\end{enumerate}
}

\begin{document}

% \sfseries
\centerline{\textbf{\LARGE SBUS Data Examples}}
\begin{center} \large
David Ingram\\
TIME-EACM Project\\
University of Cambridge Computer Lab\\
7th May 2007\\
\end{center}

{ \parskip 1mm plus 1pt \tableofcontents }
\pagestyle{fancy}

\appendix
\section{Example messages and schemas}

\subsection{Traffic message}

This example shows some typical complex structured data from a traffic
monitoring application.

\subsubsection*{LITMUS Schema:}

\begin{verbatim}
@event
{
   txt stream
   vehicle
   {
      type < #car #bike #bus #lorry >
      [ ^registration - ]
   }
   position
   {
      dbl longitude
      dbl latitude
   }
   [ direction < #east #west #south #north > ]
   [ dbl speed ]
   clk time
   data-providers (+ txt provider )
   [ dbl probability ]
}

@registration txt
\end{verbatim}

\subsubsection*{XML Message encoding:}

\begin{verbatim}
<event>
   <stream>alpha</stream>
   <vehicle>
      <type>#car</type>
      <registration>AB06 CDE</registration>
   </vehicle>
   <position>
      <longitude>0.123</longitude>
      <latitude>52.34</latitude>
   </position>
   <direction>east</direction>
   <time>01/04/2006, 07:37:33</time>
   <data-providers>
      <provider>ANPR</provider>
      <provider>Irisys</provider>
   </data-providers>
   <probability>0.97</probability>
</event>
\end{verbatim}

\subsubsection*{Binary message encoding:}

\begin{tabular}{ll}
Count & String length\\
String & ``alpha''\\
Count & 1 byte enumeration ID, value 0\\
Byte & Optional case flag, value 1\\
Count & String length\\
String & ``AB06 CDE''\\
8 bytes & Floating point, value 0.123\\
8 bytes & Floating point, value 52.34\\
Byte & Optional case flag, value 1\\
Count & 1 byte enumeration ID, value 0\\
Byte & Optional case flag, value 0\\
8 bytes & Date and Time\\
& \\
Count & 1 byte list length, value 2\\
Count & String length\\
String & ``ANPR"\\
Count & String length\\
String & ``Irisys"\\
& \\
Byte & Optional case flag, value 1\\
8 bytes & Floating point, value 0.97\\
\end{tabular}

\subsubsection*{RELAX-NG equivalent schema:}

\begin{verbatim}
<element name="event">
   <element name="stream"><text/></element>
   <element name="vehicle">
      <element name="type">
         <choice>
            <value>car</value><value>bike></value>
            <value>bus</value><value>lorry</value>
         </choice>
      </element>
      <optional>
         <element name="reg"><ref name="registration"></element>
      </optional>
   </element>
   <element name="position">
      <element name="longitude"><double/></element>
      <element name="latitude"><double/></element>
   </element>
   <optional><element name="direction">
      <choice>
         <value>east</value><value>west</value>
         <value>south</value><value>north</value>
      </choice>
   </element></optional>
   <optional><element name="speed"><double/></element></optional>
   <element name="time"><datetime/></element>
   <element name="data-providers">
      <oneOrMore><element name="provider"><text/></element></oneOrMore>
   </element>
   <optional><element name="probability"><double/></element></optional>
</element>

<define name="registration"><text/></define>
\end{verbatim}

\subsection{Simple RPC}

This example represents the return value from an RPC such as the following:\\
\verb^get_bus_posn(int fleet, int number)^\\
\verb^   : (int x, int y, double elevation, int route)^

\subsubsection*{LITMUS Schema:}

\begin{verbatim}
@bus-position
{
   clk timestamp
   int x
   int y
   [ dbl elevation ]
   int route
}
\end{verbatim}

\subsubsection*{XML Message encoding:}

\begin{verbatim}
<bus-position>
   <timestamp>2/9/2006, 14:34:27</timestamp>
   <x>123</x>
   <y>456</y>
   <elevation>3.1</elevation>
   <route>42</route>
</bus-position>
\end{verbatim}

\subsubsection*{Binary message encoding:}

\begin{tabular}{ll}
8 bytes & Date and Time\\
4 bytes & Integer, value 123\\
4 bytes & Integer, value 456\\
Byte & Optional case flag, value 1\\
8 bytes & Floating point, value 3.1\\
4 bytes & Integer, value 42\\
\end{tabular}

\subsubsection*{RELAX-NG equivalent schema:}

\begin{verbatim}
<element name="bus-position">
   <element name="timestamp"><datetime/></element>
   <element name="x"><integer/></element>
   <element name="y"><integer/></element>
   <optional>
      <element name="elevation"><double/></element>
   </optional>
   <element name="route"><integer/"></element>
</element>
\end{verbatim}

\subsection{Fixed-length list}

This example illustrates an array, for example as the return value from
the following RPC:
\verb^get_monthly_sales() : int[12]^

\subsubsection*{LITMUS Schema:}

\begin{verbatim}
@monthly-sales
(12
   int month
)
\end{verbatim}

\subsubsection*{XML Message encoding:}

\begin{verbatim}
<monthly-sales>
   <month>10</month>
   <month>20</month>
   ...
   <month>120</month>
</monthly-sales>
\end{verbatim}

\subsubsection*{Binary message encoding:}

\begin{tabular}{ll}
4 bytes & Integer, value 10\\
4 bytes & Integer, value 20\\
... & ...\\
4 bytes & Integer, value 120\\
\end{tabular}

\subsubsection*{RELAX-NG equivalent schema:}

\begin{verbatim}
<element name="monthly-sales">
   <repeat count="12">
      <element name="month"><integer/></element>
   </repeat>
</element>
\end{verbatim}

\subsection{Variable-length list of structures}

A passenger list is used to show a variable length data structure:\\
\verb^get_passengers() : (firstname, surname)[]^

\subsubsection*{LITMUS Schema:}

\begin{verbatim}
@passenger-list
(
   passenger
   {
      [ flg prebooked ]
      txt firstname
      txt surname
   }
)
\end{verbatim}

\subsubsection*{XML message encoding:}

\begin{verbatim}
<passenger-list>
   <passenger>
      <prebooked>false</prebooked>
      <firstname>Fred</firstname>
      <surname>Flintstone</surname>
   </passenger>
   <passenger>
      <prebooked>true</prebooked>
      <firstname>Barney</firstname>
      <surname>Rubble</surname>
   </passenger>
</passenger-list>
\end{verbatim}

\subsubsection*{Binary message encoding:}

\begin{tabular}{ll}
Count & 1 byte list length, value 2\\
Byte & Optional case flag, value 1\\
Byte & Boolean, value 0\\
Count & String length\\
String & ``Fred''\\
Count & String length\\
String & ``Flintstone''\\
Byte & Optional case flag, value 1\\
Byte & Boolean, value 1\\
Count & String length\\
String & ``Barney''\\
Count & String length\\
String & ``Rubble''\\
\end{tabular}

\subsubsection*{RELAX-NG equivalent schema:}

\begin{verbatim}
<element name="passenger-list">
   <zeroOrMore>
      <element name="passenger">
         <optional>
            <element name="prebooked"><boolean/></element>
         </optional>
         <element name="firstname"><text/></element>
         <element name="surname"><text/></element>
      </element>
   </zeroOrMore>
</element>
\end{verbatim}

\subsection{Nested with options}

This calendar has optional \verb^<image>^ and \verb^<advance-warning>^ tags:

\subsubsection*{LITMUS Schema:}

\begin{verbatim}
@calendar
(
   holiday
   {
      type < #national #personal >
      txt name
      date
      {
         txt month
         int day
      }
      [ bin image ]
      [ int advance-warning ]
   }
)
\end{verbatim}

\subsubsection*{XML Message encoding:}

\begin{verbatim}
<calendar>
   <holiday type="national">
      <name>Christmas</name>
      <date>
         <month>December</month>
         <day>25</day>
      </date>
      <image>
         4D18C237
         0A973B5C
      </image>
      <advance-warning>-</advance-warning>
   </holiday>
   <holiday type="personal">
      <name>Amber's birthday</name>
      <date>
         <month>April</month>
         <day>23</day>
      </date>
      <image>-</image>
      <advance-warning>14</advance-warning>
   </holiday>
</calendar>
\end{verbatim}

\subsubsection*{Binary message encoding:}

\begin{tabular}{ll}
Count & 1 byte list length, value 2\\
& \\
Count & 1 byte enumeration ID, value 0\\
Count & String length\\
String & ``Christmas"\\
Count & String length\\
String & ``December"\\
4 bytes & Integer, value 25\\
Byte & Optional case flag, value 1\\
Count & Number of bytes of data\\
Data & Binary bytes\\
Byte & Optional case flag, value 0\\
& \\
Count & 1 byte enumeration ID, value 1\\
Count & String length\\
String & ``Amber's birthday"\\
Count & String length\\
String & ``April"\\
4 bytes & Integer, value 23\\
Byte & Optional case flag, value 0\\
Byte & Optional case flag, value 1\\
4 bytes & integer, value 14\\
\end{tabular}

\subsubsection*{RELAX-NG equivalent schema:}

\begin{verbatim}
<element name="calendar">
   <zeroOrMore>
      <element name="holiday">
         <element name="type">
            <choice>
               <value>national</value>
               <value>personal</value>
            </choice>
         </element>
         <element name="name"><text/></element>
         <element name="date">
            <element name="month"><text/></element>
            <element name="day"><integer/></element>
         </element>
         <optional>
            <element name="image"><binary/></element>
         </optional>
         <optional>
            <element name="advance-warning"><integer/></element>
         </optional>
      </element>
   </zeroOrMore>
</element>
\end{verbatim}

\section{Mapping from LITMUS to RELAX-NG schema language}

This mapping is provided for compatability with other tools only, and is
not reversible (use LITMUS for your master schemas).

\subsection{Primitive types}

\begin{tabular}{l|l}
\hline
LITMUS \rule[-2mm]{0mm}{6.5mm} & RELAX-NG\\
\hline
\verb^int foo^ \rule[0mm]{0mm}{4.5mm} & \verb^<element name="foo"><integer/></element>^\\
\verb^dbl foo^ & \verb^<element name="foo"><double/></element>^\\
\verb^flg foo^ & \verb^<element name="foo"><boolean/></element>^\\
\verb^txt foo^ & \verb^<element name="foo"><text/></element>^\\
\verb^bin foo^ & \verb^<element name="foo"><binary/></element>^\\
\verb^clk foo^ & \verb^<element name="foo"><datetime/></element>^\\
\verb^loc foo^ \rule[-2mm]{0mm}{2mm} & \verb^<element name="foo"><location/></element>^\\
\hline
\end{tabular}

\subsection{Composite types}

\begin{tabular}{l|l}
\hline
LITMUS \rule[-2mm]{0mm}{6.5mm} & RELAX-NG\\
\hline
\verb^foo { elt1 elt2... eltN }^ \rule[0mm]{0mm}{4.5mm} &
	\verb^<element name="foo">elt1... eltN</element>^\\
\verb^( elt )^ & \verb^<zeroOrMore>elt</zeroOrMore>^\\
\verb^(+ elt )^ & \verb^<oneOrMore>elt</oneOrMore>^\\
\verb^(N elt )^ & \verb^<oneOrMore>elt</oneOrMore>^\\
\verb^[ elt ]^ & \verb^<optional>elt</optional>^\\
\verb^< elt1 elt2... eltN >^ & \verb^<choice>elt1 elt2... eltN</choice>^\\
\verb^< #val1 #val2... #valN >^ &
	\verb^<choice><value>val1</value>...</choice>^\\
\verb^@elt^ & \verb^<define name="label">elt</define>^\\
\verb~^label~ \rule[-2mm]{0mm}{2mm} & \verb^<ref name="label"/>^\\
\hline
\end{tabular}

Note: structures, lists, enumerations and type references in
LITMUS are all named, so the RELAX-NG equivalents given will
also be wrapped by \verb^<element name="foo">...</element>^.

\subsection{Differences from standard RELAX-NG}

RELAX-NG doesn't define any built-in data types, except for
\verb^<text/>^; conversely SBUS has built-in data types but
doesn't allow access to external libraries of datatypes,
such as XML Schema's via \verb^<data type="foo"/>^.

RELAX-NG has no way to express fixed-length arrays\\
(e.g. via some sort of \verb^<repeat count="n">^ tag).

SBUS supports named patterns via \verb^<define>^ and \verb^<ref>^;
however it does not use the \verb^<grammar>^ and \verb^<start>^ tags.
Instead the \verb^<define>^s are based on the special
\verb^<datatypes>^ section of the component description.
SBUS does not allow recursive \verb^<ref>^s.

\subsection{RELAX-NG features not used by the SBUS mapping}

\begin{verbatim}
<attribute name="foo">...</attribute>
<interleave>
<group>, <list> (whitespace separated tokens)
<externalRef>, <parentRef>, the combine attribute, <notAllowed/> and <include>
Namespaces, name classes and annotations
\end{verbatim}

In particular note that attributes are not needed, since they are no
more expressive than elements defined inside the tag (a partial reverse
mapping could also be achieved by promoting attributes to elements, and
sending them before other content in a defined order).

\subsection{Named type equivalents}

\begin{tabular}{l|l|l}
\hline
RELAX-NG \rule[-2mm]{0mm}{6.5mm} & Type system & LITMUS\\
\hline
\verb^<element>^ \rule[0mm]{0mm}{4.5mm} & Struct, Tuple & \verb^{ }^\\
\verb^<oneOrMore>^ & Sequence / List & \verb^(+ )^\\
\verb^<zeroOrMore>^ & Sequence / List & \verb^( )^\\
\verb^<repeat>^ & Array & \verb^(N )^\\
\verb^<value>^ & Enumeration & \verb^< # # >^\\
\verb^<choice>^ & Overloaded fn (top-level only) or Union & \verb^< >^\\
\verb^<optional>^ & Default value & \verb^[ ]^\\
\verb^<define>^ and \verb^<ref>^ \rule[-2mm]{0mm}{2mm} & Macros
   & \verb^@^ and \verb~^~\\
\hline
\end{tabular}

\section{LITMUS design notes}

LITMUS is both more concise and more readable than schema languages such as
RELAX-NG (and considerably more so than W3C XML Schemas). It shares RELAX-NG's
ability to define inherently nested data structures with a natural nested
syntax, rather than enforcing external references for all nested structures
like XML DTD's or ICE's SLICE IDL.

LITMUS does not require the canonical \verb^name="..."^ attributes on
every element, which are tedious and detract from clarity. LITMUS does
not even have separate concepts of element and attribute as most
XML-based approaches do, since attributes in XML source data structures
can always be promoted to elements.

All well-formed schemas in LITMUS can be parsed unambiguously.
For example, the following reasonable-looking XML fragment has no
\verb^<element>^ wrappers around the \verb^<integer/>^ and
\verb^<text/>^ tags. This makes it impossible to tell if a message such as\\
\verb^<unparsable>123 Test</unparsable>^ is an integer
followed by a string, or just a string.

\begin{verbatim}
<element name="unparsable">
   <optional><integer/"></optional>
   <text/>
</element>
\end{verbatim}

All elements in a LITMUS schema are named.
Note that choices (Unions) are not named themselves, since their child elements
must be, so this would be redundant. By contrast the C language's native
type system does name union members, which gives the
illusion of an extra level of nesting when dereferenced using dot notation.

\end{document}
