\documentclass[12pt,a4paper,twoside]{article}
\usepackage{fancyhdr,graphicx,latexsym}

\setlength{\parindent}{0cm}
\setlength{\parskip}{2ex plus1ex minus 0.5ex}

\addtolength{\evensidemargin}{-2.5cm}
\addtolength{\oddsidemargin}{-0.5cm}
\addtolength{\textwidth}{3cm}

\addtolength{\headheight}{0.2cm}
\addtolength{\topmargin}{-1cm}
\addtolength{\textheight}{2.5cm}

\renewcommand{\_}{\texttt{\symbol{95}}}
\addtolength{\fboxsep}{0.1cm}
\newcommand{\param}[1]{\textit{\textrm{\textmd{#1}}}}
\newcommand{\codebar}{\rule{\textwidth}{0.3mm}}
% \newcommand{\spc}{\hspace{0.5mm}$\sqcup$\hspace{0.6mm}}
\newcommand{\spc}{\hspace{0.5mm}$\Box$\hspace{0.5mm}}
\newcommand{\todo}[1]{\textbf{TODO: #1}}

\newlength{\codelen}
\newcommand{\code}[1]
{\begin{center}\fbox{\parbox{16cm}{\texttt{#1}}}\end{center}}

\fancyhead{}
\fancyhead[RO,LE]{\thepage}
\fancyhead[LO,RE]{PIRATES Java Binding}
\fancyfoot{}
\pagestyle{empty}

\newenvironment{bulletlist}
{
	\begin{itemize}
	\setlength{\itemsep}{0ex}
	\setlength{\parsep}{0ex}
}
{
	\end{itemize}
}

\newenvironment{alphalist}
{
	\begin{enumerate}
	\setlength{\itemsep}{0ex}
	\setlength{\parsep}{0ex}
	\renewcommand{\labelenumi}{(\alph{enumi})}
}
{
	\end{enumerate}
}

\newenvironment{numericlist}
{
	\begin{enumerate}
	\setlength{\itemsep}{0ex}
	\setlength{\parsep}{0ex}
}
{
	\end{enumerate}
}

\begin{document}

% \sfseries
\centerline{\textbf{\LARGE PIRATES Java Binding}}
\begin{center} \large
David Ingram\\
TIME-EACM Project\\
University of Cambridge Computer Lab\\
13th August 2009\\
\end{center}

{ \parskip 1mm plus 1pt \tableofcontents }
\pagestyle{fancy}

\section{Compilation}

To use the Java language binding, first follow the instructions
in the general User Guide for installing and configuring PIRATES.
The Java library still needs the C++ wrapper program in order to run,
and many of the useful tools (\verb^sbus^, \verb^speek^, \verb^spoke^ etc)
require the C++ library. You will also want to run the RDC before
starting your components.

Next, run \verb^make^ in the
\verb^java/^ subdirectory to compile the Java library.
Then add the directory
\verb^java/classes^ to your CLASSPATH. There is a Java version
of the demo component in \verb^java/tools^.

\section{Example usage}

The following code does not represent a real component, but illustrates
the general pattern of use for the API. See \verb^DemoCpt.java^
for a specific working example.

\begin{verbatim}
Component com;
Endpoint ep; // We will create just one endpoint, which will be a server
ProtoMessage msg;
Node response;
int data;

String msg_hash = "AABBCCDDEEFF", reply_hash = "FFEEDDCCBBAA";
EndpointType type = EndpointType.Server;
String cpt_name = "foo-cpt", instance_name = "instance1";
String ep_name = "foo-ep", peer_ep_name = "bar-ep";
int echo_level = Log.LogErrors + Log.LogWarnings + Log.LogMessages;
int log_level = echo_level + Log.LogDebugging;
String topic = "foo-topic", subscription = "foo/bar > 42";
String rdc_address = "192.168.0.42";
String peer_address = "+Nbarcpt +Iinstance42";

com = new Component(cpt_name, instance_name);
ep = com.add_endpoint(ep_name, type, msg_hash, reply_hash);
// Optionally, com.add_rdc(rdc_address);
// Optionally, Component.set_log_level(log_level, echo_level);
com.start(cpt_name + ".cpt");
// Optionally, ep.subscribe(topic, subscription, null);
// Optionally, ep.map(peer_address, peer_ep_name);
while(true)
{
   msg = ep.rcv();
   data = msg.tree.extract_int("x");
   response = Node.pack(Node.pack(x * x, "x-squared",
      Node.pack(Math.sqrt(x), "root-x"), "math-results");
   ep.reply(msg, response);
}
\end{verbatim}

\section{Packing and extracting}

The \verb^Node^ object is the in-memory representation
for all PIRATES message content (a \verb^ProtoMessage^ includes
a \verb^Node^ and some metadata, such as the message source).

\verb^Node^s are obtained in one of three ways: by using a
set of ``packing'' functions to convert data from your programming
language's native types, by importing from some XML text, or by receiving
them from PIRATES across the network.

Given a \verb^Node^, you can use a set of ``extraction''
functions to convert to your programming language's natively
typed variables, or export it to an XML string, or send it
across the network using PIRATES.

The packing and extracting operations are
performed manually using Java functions. The recommended approach is to
consolidate these calls into specific marshalling methods you define
for your application's objects. It would also be possible to write
a stub generator to automatically create the packing and extracting
code, but such a thing is not shipped with PIRATES.

Internally the \verb^Node^ is converted to a binary
representation for transmission across the network, and then
converted back to a \verb^Node^ at the other end.
The schema is essential at the unmarshalling stage in order to
make sense of the message sent over the wire, since the default
binary encoding does not transmit verbose type information.

\subsection{Packing functions}

The packing functions are used to programmatically build messages as
\verb^Node^s in memory. The first set of pack functions simply wrap primitive
types with a \verb^Node^. The second set pack multiple existing
\verb^Node^s into compound data structures. You must pack fields
in the same order they are specified in the schema.

The packing functions allow you to specify element
names, if you wish to make your code clearer and introduce additional
field name checks. There are also versions of these functions
which do not take the \verb^opt_name^ optional argument,
indicated by square brackets here:

\begin{verbatim}
class Node
{
   static Node pack(int n          [, String opt_name])
   static Node pack(boolean bool   [, String opt_name])
   static Node pack(String s       [, String opt_name])
   static Node pack(byte[] data    [, String opt_name])
   static Node pack(double x       [, String opt_name])
   static Node pack(Datetime time  [, String opt_name])
   static Node pack(Location loc   [, String opt_name])
   static Node pack_empty(         [String opt_name])
   static Node pack_value(String s [, String opt_name])

   // Struct, List, Seq, Array:
   static Node mklist(                    [String opt_name])
   Node append(Node n)
   static Node pack(Node[] array          [, String opt_name])
   static Node pack(Node n1, ..., Node n6 [, String opt_name])
   // + Variations of the above for from 2 to 6 nodes
   ...
};
\end{verbatim}

\verb^pack_empty()^ is used to pack
optional elements when they are not present in this particular message.

There is no explicit mention of choice elements in the
packing functions; the user simply packs the alternative they have chosen for
this particular message. PIRATES tries to decide which alternative of a choice
you are using by looking at the types of the elements. It is
simplistic and does not look ahead at the types of further elements when it
does so, hence it can fail in cases where a human examining the whole message
could differentiate. If there is an ambiguity it can't resolve then
element names must be provided for those elements (this is the only
case where element names are not optional).

\subsection{Extraction functions}

The extraction functions are the logical counterpart of the packing
functions. The programmer is assumed to know what types to expect
(they are working from the same schema, after all). There are
extract functions for each primitive type to unwrap a \verb^Node^,
converting its contents back to a native value.

If the \verb^Node^ you hold is a compound type, you have three options for
extracting its contents. Firstly, you can call \verb^extract_item()^ specifying
either the index or field name of the child member required. This returns
another \verb^Node^ which can then itself be extracted. Secondly, if the
primitive field you wish to retrieve is only nested inside one level, you can
access it directly by specifying an optional \verb^item^ index to the relevant
typed extraction function (e.g. \verb^node.extract_dbl(3)^ means extract the
floating point value which is the fourth sub-element inside the structure or
list \verb^node^). Finally, if the \verb^Node^ is a structure rather than
a list, you can do the same but specify the child by name
instead (there is a separate set of extraction functions which accept a string
argument for this purpose).

\verb^class Node^\\
\verb^{^\\
\verb^   ...^\\
\verb^   int count()^\\
\verb^   Node extract_item(int item)^\\
\verb^   Node extract_item(String name)^\\
\verb^   Node[] extract_array()^\\
\verb^   String get_name(int opt_item)^\\
\verb^   String get_name()^\\
\verb^	^\\
\verb^   int extract_int(^\textit{args}\verb^)^\\
\verb^   boolean extract_flg(^\textit{args}\verb^)^\\
\verb^   double extract_dbl(^\textit{args}\verb^)^\\
\verb^   String extract_txt(^\textit{args}\verb^)^\\
\verb^   byte[] extract_bin(^\textit{args}\verb^)^\\
\verb^   int num_bytes(^\textit{args}\verb^)^\\
\verb^   Datetime extract_clk(^\textit{args}\verb^)^\\
\verb^   Location extract_loc(^\textit{args}\verb^)^\\
\verb^   String extract_value(^\textit{args}\verb^)^\\
\verb^   int extract_enum(^\textit{args}\verb^)^\\
\verb^   ...^\\
\verb^};^\\

\textit{args} may be (i) nothing, (ii) \verb^int item^, or (iii)
\verb^String name^.\\
For example, you may call:\\
\verb^extract_int()^, \verb^extract_int(int item)^,
or \verb^extract_int(String name)^.\\
The value \verb^item^ specifies the child index -- 0, 1, 2 etc.

\section{Component initialisation}

The \verb^Component^ class represents one component.
The following sections deal with the API calls in roughly the order
they are likely to be used within a program.

\subsection{Establishing a component instance}

\begin{verbatim}
class Component
{
   Component(String cpt_name);
   Component(String cpt_name, String instance_name);
   ...
};
\end{verbatim}

Creating the component object also starts the wrapper process running.
The component is not active, however -- the wrapper waits for further
configuration.

The \verb^instance_name^ may be omitted, in which case it is set to
the same as \verb^cpt_name^. Instance names are passed to the RDC
and can be used to specify individual replicas of a component for
mapping.

\subsection{Adding endpoints}

\begin{verbatim}
class Component
{
   ...
   Endpoint add_endpoint(String name, EndpointType type, String msg_hash);
   // or, for client/server endpoints (RPCs):
   Endpoint add_endpoint(String name, EndpointType type, String msg_hash,
      String reply_hash);
   ...
};

enum EndpointType { Server, Client, Source, Sink };
\end{verbatim}

LITMUS hash codes such as \verb^msg_hash^ and \verb^reply_hash^
represent the schema for a message or endpoint.
They are passed to the API as strings containing a 12 digit
hexadecimal number, e.g. ``\verb^AABBCCDDEEFF^''.

There are additional versions of \verb^add_endpoint()^ which accept
\verb^HashCode^ objects instead of hex strings, which is sometimes
more convenient:

\begin{verbatim}
class Component
{
   ...
   Endpoint add_endpoint(String name, EndpointType type, HashCode msg_hc);
   // or, for client/server endpoints (RPCs):
   Endpoint add_endpoint(String name, EndpointType type, HashCode msg_hc,
      HashCode reply_hc);
   ...
};
\end{verbatim}

Repeat calls to \verb^add_endpoint()^ as many times as necessary
to declare all the endpoints supported by the component.
You may call \verb^add_endpoint()^ before or after starting the
component.
Starting a component serves the dual purpose of attaching it to
its component metadata and starting the wrapper.
The endpoints added before the \verb^start()^ call
are checked against those listed in the component metadata,
those added afterwards are not.

\subsection{Starting the component}

\begin{verbatim}
class Component
{
   ...
   void start(String metadata_filename);
   void start(String metadata_filename, int port);
};
\end{verbatim}

This instructs the wrapper to go ahead and validate the component's
endpoints, and then (if the schemas match the metadata) to register
with the RDCs if applicable and to begin listening for external
connections from other components.

The \verb^LITMUS_PATH^ environment variable is used to locate the
metadata file.

\subsection{Mapping an endpoint}

Mapped \verb^Endpoint^s correspond with an open TCP/IP connection
from the wrapper to another component's wrapper.
Newly created endpoints are unmapped.
Unmapped endpoints can be used without error, although
obviously no data will be returned, and RPCs will
block indefinitely.

Once a component is running its endpoints will normally be
mapped by a third party mapping engine, by a script calling
\verb^spoke map^, or with management reconfiguration tools.
It is however possible to map endpoints programmatically within
the component itself (for example, \verb^spoke^ itself needs
to do this). Bear in mind that hardcoding
\verb^map()^ calls into the application is a lot less flexible
since users cannot easily change it,
and should be avoided where possible. If your endpoints will
be mapped by external agents, you can skip this section entirely.

An endpoint of type client may be mapped to at most one server.
The \verb^map()^ call \textit{remaps} endpoints of type client
if they are already mapped.
An endpoint of type server may be mapped
to any number of clients. Source endpoints may be mapped to any number
of sink endpoints, and a sink endpoint may be mapped to any number of
sources.

\begin{verbatim}
class Endpoint
{
   // Mapping: 
   String map(String address, String endpoint);
   void unmap();
   int ismapped();
   ...
};
\end{verbatim}

The \verb^map()^ function returns a string containing the address of the
other component if successful, otherwise null.

The address string is formatted as described in the ``Component
addresses'' section of the user guide. It can take one of two
main forms.
The first form specifies an explicit network address, and does not
use a RDC.
The address is then a string of the form \verb^hostname:port^
(or just \verb^:port^, which is equivalent to \verb^localhost:port^).
The \verb^hostname^ part may be a DNS name or an IP address in
dotted decimal notation.
% and an optional public key if
% identity must be confirmed.

The second, considerably more flexible form of mapping
consults the RDC and tries to satisfy the map constraints
specified by the address string.
This is the normal approach, since usually the current network address
of the component required is not known.
See the user guide for the list of valid constraints (each of which
begins with a plus sign in the string).

The \verb^Endpoint^ parameter may be null, in which case it is
assumed that you wish to map to an endpoint on the target component
with the same name as the local endpoint being mapped (it is common
for client/server and source/sink pairs of endpoints which are
intended to be connected to have the same name).

Normally endpoints are mapped once and then only remapped occasionally,
e.g. by external agencies to achieve migration. A P2P overlay network
is an exception -- in this case remapping will be very frequent
and would use the explicit address method rather than an RDC, since
components may be randomly accessed potentially worldwide.

% \subsubsection{Mapping flags}
%
% Mapping flags are not implemented yet, so currently \verb^sflags^
% should be set to zero.
%
% \begin{verbatim}
% // Values for sflags:
% const int UnavailableSilent = 1 << 0;
% const int DisconnectSilent  = 1 << 1;
% const int AttemptFailover   = 1 << 2;
% const int AttemptReconnect  = 1 << 3;
% const int AllowMigration    = 1 << 4;
% const int AwaitTransfer     = 1 << 5;
% \end{verbatim}

\subsection{Subscriptions}

To keep the interface straightforward, the \verb^map()^ call does not
include parameters for subscriptions or topics; these are
specified separately by the \verb^subscribe()^ function.

You can change the subscription for currently mapped endpoints,
or specify the subscription for new connections. If you are
mapping an endpoint and need a subscription set up immediately
(at connection time), you should call \verb^subscribe()^ first
(with \verb^peer^ set to null), and then \verb^map()^.

\begin{verbatim}
class Endpoint
{
   ...
   void subscribe(String topic, String subs, String peer);
   ...
};
\end{verbatim}

In \verb^subscribe()^ calls, either \verb^topic^ or \verb^subs^
(or both) may be null. If both are null, then all messages from this
endpoint will be returned. If neither are null, then both must match
for an event to be delivered. The subscription is stated in the
content-matching language also used by the prebuilt Broker component
and described in the PIRATES Overview document.
The topic is a straightforward comparison on the topic name used to emit
events by the component. Content and topic based subscriptions are evaluated
automatically for each message and each subscriber by the emitting
component's wrapper process.

The \verb^peer^ parameter may be the name (component or instance)
of a currently mapped peer whose subscription should be altered.
It may also be the special string \verb^"*"^, in which case
the subscription in force for all currently mapped peers is changed.
Finally, if \verb^peer^ is null then the default subscription
used for future maps to this endpoint is altered.

\section{Data transfer}

\subsection{Data transfer methods}

\begin{verbatim}
class Endpoint
{
   ...
   SocketChannel sc;
      
   void emit(Node node);
   void emit(Node node, String topic);
   void emit(Node node, HashCode hc);
   void emit(Node node, String topic, HashCode hc);
   ProtoMessage rpc(Node query);
   ProtoMessage rpc(Node query, HashCode hc);
   ProtoMessage rcv();
   void reply(ProtoMessage query, Node result);
   void reply(ProtoMessage query, Node result, HashCode hc);
   ...
};
\end{verbatim}

For Source endpoints, the wrapper automatically tracks subscribers, so the
component just has to call \verb^emit()^, optionally supplying a topic.

Client endpoints use the \verb^rpc()^ call.

For Server endpoints the component must call \verb^rcv()^ and then
\verb^reply()^, passing back the query so the result can be matched up with it.

For sink endpoints, the component just has to call the \verb^rcv()^ method.

All out-of-band messages such as connection closed and flow control
are currently dealt with by the wrapper, and not passed up to the
component through the \texttt{Endpoint} interface.

The optional \verb^HashCode^ arguments to \verb^emit()^, \verb^rpc()^
and \verb^reply()^ are used by polymorphic endpoints to indicate the
type of the message being sent. Normal, non-polymorphic endpoints
should omit \verb^HashCode^, and the predefined type for the endpoint
will be used instead.

\verb^rpc()^ returns null if the endpoint is unmapped, or mapped to
more than one peer, or if it becomes unmapped during the call due to
the server disconnecting.

%If \verb^exception^ is \verb^1^ then the result contains error information
%and isn't a normally formatted \verb^snode^.

Methods which return data to the caller, i.e. \verb^rcv()^ and \verb^rpc()^,
return a \verb^ProtoMessage^ structure, defined below. The payload
is stored in the \verb^tree^ field; the other members are useful
metadata. \verb^seq^ is used for matching up RPC replies.

\begin{verbatim}
class ProtoMessage
{
   String source_cpt, source_inst, source_ep;
   String topic; // null if not applicable
   int seq;
   HashCode hc;
   Node tree;
};
\end{verbatim}
%  Object state; // for conversations only 

All the \verb^Endpoint^ data transfer methods are blocking calls.
You are free to write single- or multi-threaded components.
The endpoint's \verb^SocketChannel^ object, \verb^sc^, can be used with a
\verb^Selector^ to wait for messages to arrive.
If a component is single-threaded and has multiple services it
will be necessary to do this to see which endpoint to process next.
It is also possible
to find out the endpoint associated with a particular \verb^SocketChannel^
(or by name, etc):

\begin{verbatim}
class Component
{
   ...
   Endpoint sc_to_endpoint(SocketChannel sc);
   
   int count_endpoints();
   Endpoint get_endpoint(int index);
   Endpoint get_endpoint(String name);
   ...
};
\end{verbatim}

The wrapper only delivers incoming messages once they have been completely
received from the remote component, hence after a \verb^SocketChannel^
becomes ready a \verb^rcv()^ call should acquire the message without any
serious delay.

%For conversations, the component calls \verb^rcv()^, \verb^reply()^,
%\verb^rcv()^, \verb^reply()^ and so on.
%Replies also include the last message received.
%The state held by this side concerning the conversation can be saved
%in the \verb^ProtoMessage.state^ variable -- this value, if set, is
%returned by all subsequent \verb^rcv()^'s from the same conversation.

\section{Shutting down}

\subsection{Disconnecting from another component}

This closes the TCP connection between the components.

\begin{verbatim}
class Endpoint
{
   ...
   void unmap();
};
\end{verbatim}

\subsection{Terminating your component}

\begin{verbatim}
class Component
{
   ...
   void stop();
};
\end{verbatim}

This deregisters with the RDC, terminates the wrapper process and frees
resources.

\section{Additional APIs}

These calls are used less frequently than those discussed so far.

\subsection{Adding RDCs}

RDCs are used to register the component with, and also to
lookup other components for mapping. Generally these are
set by the \verb^SBUS_RDC_PATH^ environment variable (together
with the default location on the localhost). In rare cases
programs may wish to specify additional RDCs:

\begin{verbatim}
class Component
{
   ...
   void add_rdc(String address);
   ...
};
\end{verbatim}

Note that this call must be made after the component object
has been created, but \textit{before it is started}.
Do not explicitly add the default RDC location on the local host,
or your component will get duplicate registration errors when it runs.

Lookups will try all the RDCs sequentially until one succeeds.
The component will try to register itself with \textit{all}
the RDCs you have added (some may not exist, which is not an error).
The exception is addresses which begin with an exclaimation mark '!'
as a special flag, which indicates that they are for lookup only,
not registration.
In all cases the default RDC port number is used if none is supplied.

\subsection{Controlling logging}

There are four types of log output: errors, warnings, messages and
debugging. Errors are only generated in fatal situations (the
component always shuts down after emitting an error). Each of
these four types of logging may be directed to the component's
log file and/or to stdout. This is controlled by the log levels:
0 = nothing, 1 = errors, 2 = warnings, 4 = messages, 8 = debugging.
You can add these values together to select a mixture of message types.
The initial log level for the log file is read from the environment
variable \verb^SBUS_LOG_LEVEL^, and for stdout from \verb^SBUS_ECHO_LEVEL^.
If these are not set then everything is logged to file, and everything apart
from debugging is sent to stdout (so the default is highly verbose).
A component may change its own log levels, by calling the following
static API function:

\begin{verbatim}
class Component
{
   ...
   static void set_log_level(int log, int echo);
   ...
};
\end{verbatim}

\textbf{Note} that \verb^Component.set_log_level()^ must be called
\textit{before} \verb^Component.start()^. The API does not currently
provide a way to change log levels after that point.
Also, if you wish to inhibit the initial message from the library saying
that the wrapper has started, call the static \verb^set_log_level^ function
before \verb^new Component^ creates the component.
Log levels \textit{can} be changed remotely at any time using the built-in
endpoint for this purpose.

Both \verb^log^ and \verb^echo^ are formed by adding together
constants representing the desired types of information:
\verb^Log.LogErrors^, \verb^Log.LogWarnings^, \verb^Log.LogMessages^ and
\verb^Log.LogDebugging^.

The log files are stored in a directory specified by the \verb^SBUS_DIR^
environment variable: \verb^$SBUS_DIR/log/^.
If this environment variable is not set then the
directory \verb^$HOME/.sbus/log/^ is used by default.
Log files are named \verb^cptname-instance-X.log^ or \verb^cptname-X.log^
(the latter if no instance name is set).
Here \verb^X^ stands for either \verb^W^ or \verb^L^, indicating whether
the message arises from the wrapper or the library respectively.
Component log files overwrite previous files with the same name.
If you simultaneously run two components on the same machine with
identical component and instance names, the contents of the log
file are undefined.

\subsection{Declaring schemas}

\begin{verbatim}
class Component
{
   ...
   HashCode declare_schema(String schema);
   HashCode load_schema(String file);
	...
};
\end{verbatim}
%  void add_certificate(const char *cert);

The \verb^declare_schema^ and \verb^load_schema^ functions add
a new schema (read from a string or a file respectively) to the wrapper's
schema cache whilst the component is running.
This is useful in the special case of polymorphic
source endpoints, since these may need to define new types
before emitting them.
Once declared, a schema is available to all endpoints.
The HashCode returned by these functions must be deleted by the caller.

It is not an error if the schema is already in the cache; in this
case nothing happens, but the HashCode is still returned. This makes
these functions a convenient way to look up HashCodes for schemas,
without needing the C++ \verb^libsbuswrap^ library in order to parse the
schema in the application process.

\verb^load_schema^ uses the \verb^LITMUS_PATH^ environment variable
to locate the file containing the schema.

% The \verb^cert^ is a role certificate (there may be several)
% which is later presented to the server. Currently certificates are
% unimplemented.

\subsection{Obtaining wrapper state}

The following methods of \verb^Component^ are valid after
\verb^start()^ has returned.

\begin{verbatim}
class Component
{
   ...
   int count_builtins();
   Builtin get_builtin(int index);
   int get_listen_port();
   String get_address();
   Node get_status();
   String get_schema(HashCode hc);
   ...
};

class Builtin
{
   String name;
   EndpointType type;
   HashCode msg_hc, reply_hc;
};
\end{verbatim}

\verb^get_address()^ returns the canonical address string for this
component (formed from the IP address of the first non-loopback interface,
together with the listening port).

\verb^get_status()^ returns a \verb^Node^ containing the current
component status structure (this is exactly the same thing as that
returned to third parties by the built-in endpoint of the same name).

\verb^get_schema()^ interrogates the wrapper's schema cache, and
returns the full text for the schema represented by the supplied
hash code. This is useful if a component has polymorphic sink
endpoints and hence may receive messages it doesn't know the schema
for. If the hash code presented does not appear in the schema
cache, the string \verb^"?"^ is returned.

\section{Thread safety}

\textbf{Each endpoint must be accessed by only one thread}. Multiple threads
are fine provided each is responsible for different endpoints.
A single thread may use many endpoints, but it must do so exclusively.

In a multi-threaded component, endpoints of type Server, Source or Sink
can generally be handled conveniently by their own thread. Multiple
threads often need to employ the same Client endpoint, however.
To make this possible without breaking the rule of one thread per
endpoint, a \verb^clone^ method is provided to duplicate endpoints.
This is merely a convenience method which simply calls
\verb^add_endpoint()^ for you to generate another endpoint of exactly
the same type.

\begin{verbatim}
class Component
{
   ....
   Endpoint com.clone(Endpoint ep);
   ...
};
\end{verbatim}

Only the thread which starts the component may safely call \verb^clone()^.

\section{Import and Export}

\begin{verbatim}
class Node
{
   ...
   static Node fromxml(String s) throws ImportException;
   static Node fromxml_file(String path) throws ImportException;
   static String toxml(Node node, boolean pretty);
   ...
};

\end{verbatim}

The \verb^fromxml()^ functions convert XML text in a string or from a file
into a parsed \verb^Node^. In case of error they throw an
\verb^ImportException^.

\verb^toxml()^ exports a \verb^Node^ as XML text. If \verb^pretty^ is
\verb^true^ the output will contain newlines and indentation; if
\verb^pretty^ is \verb^false^ it will all be on one line.

%\section{Simplified \texttt{select()} wrapper}
%
%This is an optional utility provided by the library. It wraps
%the system \verb^select()^ call in a nicer interface. You can use
%either \verb^multiplex^ or the native \verb^select()^ to manage
%your endpoint file descriptors in components.
%
%Adding an endpoint to the multiplex, e.g. \verb^multi->add(ep)^,
%is similar to\\
%\verb^multi->add(ep->fd, MULTI_READ)^, but also
%enables checking of the \verb^ep->message_waiting()^ method. This
%is necessary if your endpoint might receive a message whilst
%it is being mapped or unmapped (in such cases the message is
%automatically enqueued for later delivery, after the map call returns,
%but ordinary \verb^select()^ will not notice a message has arrived
%because it has been read from the file descriptor already).
%
%\begin{verbatim}
%class multiplex
%{
%   enum multi_mode { MULTI_READ, MULTI_WRITE };
%
%   multiplex();
%   ~multiplex();
%
%   void add(int fd, multi_mode mode = MULTI_READ);
%   void add(sendpoint *ep); // Also enables check for ep->message_waiting()
%   void remove(int fd, multi_mode mode = MULTI_READ);
%   void remove(sendpoint *ep, int silent = 0);
%   void clear();
%
%   // These functions return a FD, or -1 if none ready before timeout:
%   int poll();
%   int wait();
%   int pause(int us);
%
%   multi_mode last_mode();
%}
%\end{verbatim}

%=========================

% \section{Remote Wrappers}
% 
% The Java and Javascript bindings use a remote wrapper. The wrapper is
% the standard C++ program, and runs on the server supplying the
% webpage. This allows the Java library to make network connections to it.
% The server machine runs a wrapper factory, which reuses wrappers from
% a pool, and includes timeouts to prevent misuse.
% 
% The use of a remote wrapper allows Java components to be \textit{real
% components}; i.e. they support all the standard SAP's and so on.
% It makes Javascript bindings possible since the library which talks
% to the wrapper does not need threads, and it keeps the language-specific
% part thin for easy port maintainance.
% 
% One possible drawback of a remote wrapper is more independent failure
% conditions for the wrapper and application. We can deal with this
% easily by making either simply abort if the other goes away,
% preserving the "neither or both" semantics which occur when both
% run on the same machine (which may be rebooted, for example).
% In fact this logic is necessary anyway, since we have to assume an
% application might crash and the wrapper then needs to notice this
% and terminate. Wrapper crashes independent of the application are
% hopefully unlikely; if they occur the semantics should be that the
% application terminates \textit{eventually}, that is to say it never
% tries to restart the wrapper but it may continue running without
% PIRATES services, for example to save work or continue with its other
% functions if that is more important than restarting.

\end{document}
