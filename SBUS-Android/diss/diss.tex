\documentclass[12pt,twoside,notitlepage]{report}

\usepackage{a4}
\usepackage{verbatim}
\usepackage{caption}
\usepackage{subcaption}
\usepackage{listings}

% set tab size for lstlisting
\lstset{tabsize=4}

\input{epsf}

\raggedbottom                           % try to avoid widows and orphans
\sloppy
\clubpenalty1000%
\widowpenalty1000%

\addtolength{\oddsidemargin}{6mm}       % adjust margins
\addtolength{\evensidemargin}{-8mm}

\renewcommand{\baselinestretch}{1.1}    % adjust line spacing to make
                                        % more readable

\begin{document}

\bibliographystyle{plain}


%%%%%%%%%%%%%%%%%%%%%%%%%%%%%%%%%%%%%%%%%%%%%%%%%%%%%%%%%%%%%%%%%%%%%%%%
% Title


\pagestyle{empty}

\hfill{\LARGE \bf Thomas Smith}

\vspace*{60mm}
\begin{center}
\Huge
{\bf Automatic Configuration in Mobile Environments} \\
\vspace*{5mm}
Computer Science Tripos Part II \\
\vspace*{5mm}
Sidney Sussex College \\
\vspace*{5mm}
\today  % today's date
\end{center}

\cleardoublepage

%%%%%%%%%%%%%%%%%%%%%%%%%%%%%%%%%%%%%%%%%%%%%%%%%%%%%%%%%%%%%%%%%%%%%%%%%%%%%%
% Proforma, table of contents and list of figures

\setcounter{page}{1}
\pagenumbering{roman}
\pagestyle{plain}

\chapter*{Proforma}

{\large
\begin{tabular}{ll}
Name:               & \bf Thomas Smith	\\
College:            & \bf Sidney Sussex College	\\
Project Title:      & \bf Automatic Configuration \\ &\bf in Mobile Environments	\\
Examination:        & \bf Computer Science Tripos Part II, 2013 	\\
Word Count:         & \bf 6410\footnotemark[1] \\
Project Originator: & Dr Jatinder Singh		\\
Supervisor:         & Dr Jatinder Singh		\\ 
\end{tabular}
}

\footnotetext[1]{This word count was computed by {\tt detex -e `appendix,figure' diss.tex | tr -cd `0-9A-Za-z $\tt\backslash$n' | wc -w}}
\stepcounter{footnote}


\section*{Original Aims of the Project}

The original aim of the project was to create a system which could adapt to the changes in context which come with mobility, such as location or networks. 
This required the ability to detect these changes, in order to find and connect to new data sources. 
As a new data source may have different data schema to the expected schema, the system must offer a way to make meaningful use of this data in relation to the expected schema.

\section*{Work Completed}

A messaging middleware was modified to allow users to search for data sources by aspects of their schema. Users can search for exact matches on some part of a schema, or a looser matches, considering field types only, not names.
Once this middleware was ported for use on Android, an application was created to monitor changes in context, particularly those that arise as a result of mobility. Upon detection of a change in context, the application informs other applications of the change, or automatically reconfigures their connections to connect them to a new data source, based on their policies.

\section*{Special Difficulties}

The main difficulty in the project came from porting the messaging middleware to Android. The Android NDK\footnote{http://developer.android.com/tools/sdk/ndk/index.html} toolkit was used in order to compile C++ for use on Android, however missing libraries and poor documentation made this task more difficult than expected.
 
\newpage
\section*{Declaration}

I Thomas Smith of Sidney Sussex College, being a candidate for Part II of the Computer
Science Tripos, hereby declare that this dissertation and the work described in it are my own work,
unaided except as may be specified below, and that the dissertation
does not contain material that has already been used to any substantial
extent for a comparable purpose.

\bigskip
\bigskip
\leftline{Signed: }

\bigskip
\bigskip
\leftline{Date: }

\cleardoublepage

\tableofcontents

\listoffigures

\newpage

%%%%%%%%%%%%%%%%%%%%%%%%%%%%%%%%%%%%%%%%%%%%%%%%%%%%%%%%%%%%%%%%%%%%%%%
% now for the chapters

\cleardoublepage        % just to make sure before the page numbering is changed

\setcounter{page}{1}
\pagenumbering{arabic}
\pagestyle{headings}

% Intro + Preparation ~ 1400 words
\chapter{Introduction}

% summarise
While many applications now exist for mobile devices, few applications themselves are truly mobile. 
The user may happen to be mobile, however applications do not adapt to different environments. 
This project addresses the issues faced in creating applications which can change to suit different environments. 
I show how relevant data sources in an environment can be found for applications, and how flexibility in the data they use can be achieved. 

\section{Problem}
% what's the problem

% describe current approach
Current applications tend to be inflexible in data sources they use, often selecting from a list of possibilities, regardless of the environment. 
While some localisation might be achieved by including parameters during connection, the possible parameters which will be accepted have already been fixed, restricting how well applications can adapt to their environment. 
The data formats that applications and data sources use have also been fixed, restricting applications even further.
By connecting to local data sources, applications gain greater flexibility due to the fact that the data sources reside within the same environment.
Given an application which requires the current temperature, a global server may use current GPS co-ordinates to determine the nearest weather station. 
Yet if the application were being used indoors, a value from a local temperature sensor might not only suffice, but be more appropriate. 
Perhaps this could be achieved otherwise, if the server had known about the temperature sensor, however why should the application need to connect to a global server in order to use a local connection? Perhaps there isn't even an Internet connection available. 
Furthermore, given that the user is mobile, the server would need to know about {\sl every} temperature sensor. 
Local data sources allow applications to use a dynamic set of data sources, more related to their environment. 

% context
Context refers to any event which the system or the user feels that it is reasonable to respond to, where any happening of interest that can be observed from within a computer is considered an event \cite[page 11]{muhl2006distributed}. 
Mobile devices constantly operate in different contexts, which dictate how devices can interact with the environment. 
Upon a change in context an application may no longer be able to communicate with a given data source, or the data it is providing may no longer be useful. 
In this situation, an alternative data source must be found.

Once this alternate data source has been found, the application must still determine whether it provides useful data. 
Some data may be fundamental to the operation of the application, while other parts may be desired, but not crucial.  
Even once a data source is deemed useful, the application may still face the difficulty of the data being presented in an alternate format to the expected format. 
Fields may be missing, there may be additional fields, and fields may have different names. 
The application may also wish to specify some subscription filter about the messages it should receive, yet without knowledge of the data format, it cannot specify filters on different fields.

% example
For example, a transport application may face differences in data available for coaches (Figure \ref{fig:coach-schema}) and for trains (Figure \ref{fig:train-schema}). 
While both offer similar data to a large extent, fields are named differently, and the train schema additionally offers a menu for an on-board shop.
Alternatively, the application might be used abroad, for example in a French coach station. The data will still be very similar, but fields names may be in French (Figure \ref{fig:french-coach-schema}).

While these differences may be important when the user is idly waiting in a station, they become less important when the user is running late and simply requires the departure time and location.
Given the amount of uncertainty about what data will be available, a developer creating this application cannot hope to account for all possibilities. 
However, the developer can specify how the application should behave in different contexts.

\section{Project}
% what I've done

Schilit et al \cite{schilit1994context}  define a context-aware system as one which can examine the computing environment and react to changes in the environment. 
This project creates such a system, which performs automatic contextual reconfiguration for applications whenever there is a change in the environment. 
Pascoe \cite{pascoe1998adding} lists four context-aware capabilites, two of which occur in this reconfiguration. 
Contextual resource discovery is used to discover data sources available within the context, and contextual adaptation allows applications to adapt to different data schemas in order to use the sources. 

% describe context, say only focusing on mobility (Wi-Fi) for now.
M\"{u}hl defines three kinds of events which can be used to detect a change in context: physical events, timer events, and a state change in the system \cite[page 11]{muhl2006distributed}, all of which may warrant some kind of reconfiguration. 
In a healthcare system, a rapid accelerometer change followed by no movement may represent a physical event (namely a fall), in which case the application may need to send data to an emergency response team.
A timer event indicates the progression of real time, which advertisers could use to connect to advertisements for nearby restaurants at lunch time. 
Finally, there are state changes in the system, such as a change in network connection or location. 
This project focuses on these state changes, particularly using a change in network connection to indicate a change in context. 

% policy engine
Applications control how their connections should be reconfigured through policies. 
The ``policy engine'' monitors the environment and upon detecting a change in context applies these policies, reconfiguring applications' connections to connect them to different data sources, where appropriate.
These policies may specify the name of data sources, or be more complex rules regarding the structure of data sources' schemas. 
Alternatively, applications may opt to be notified of a relevant change, rather than have their connection reconfigured, allowing them to perform callback methods upon a change.

% repackaging
Depending on the policy, an application may be connected to a data source which does not offer the expected data schema. 
The system handles this by repackaging messages to match the application's schema. Any extraneous fields are removed, any absent fields are filled with ``empty'' values, and the policy is used to infer relationships between fields with different names. 
Once the system has repackaged the message, it passes it to the application, allowing the application to continue as if it had received a message conforming to its schema.

% subscription
The system allows an application to specify subscription filters in terms of its own schema. 
As part of the reconfiguration process the system converts this subscription to match the peer's schema, connecting the application with a working subscription in place.

% pervasive computing
The system supports many of the goals of the pervasive computing vision \cite{weiser1991computer}.
Use of local data sources allows the system to be scalable. 
Connections are created on an ad-hoc basis, based entirely upon what data sources are available within the environment. 
This allows for a dynamic set of sources which can be modified at will. 
Repackaging messages achieves some level of heterogeneity within the system. 
Different environments may represent data in different formats, though application policies allow the system to interpret this data in a meaningful way. 
The policy engine can perceive context, and use these perceptions in an intelligent way, namely by reconfiguring applications' connections according to their policies. 
While all of this is invisible to the user, applications can be configured by the user through the use of policies \cite{saha2003pervasive}.

% summarise
The system automatically reconfigures applications' connections and repackages messages to fit their schemas, granting the applications flexibility in the data sources they use.
Local data sources can be used to provide data relevant to the environment, and by repackaging these messages an application is no longer bound to an exact schema. 
Applications control how they interact with an environment through policies, while still seamlessly communicating within that environment.

\cleardoublepage

 
\chapter{Preparation}

This chapter first outlines the requirements of the system. 
It goes on to summarise related work which the project will build upon, namely the SBUS middleware \cite{ingram2009reconfigurable}. 
Finally, it describes how the system will be implemented to meet these requirements.

%TODO: Why the recursive discovery through peer inspection hasn't been implemented, and how it could be (should be somewhere else?)

\section{Requirements Analysis}

The following use cases outline the functional requirements of the system, describing how the system will operate in different circumstances.

% use cases - functional requirements
John lives in a retirement village. He has a heart condition and as such he regularly suffers from falls. He uses a heart monitor connected to an application on his phone which displays the heart monitor readings. Upon a fall, John wishes for his heart monitor readings to be sent the retirement village nurse. The nurse can then determine if John is in any immediate danger, and take appropriate action.

\begin{itemize}

\item {\bf Detect and Act upon Change in Context}

\begin{enumerate}

\item John has his phone in his pocket when he falls over in his home.

\item The system detects that John has fallen over which represents a change in his context.

\item The system reacts to this change in context and connects John's heart monitor readings to the nurse's office over the local network.

\item The nurse checks John's heart monitor readings, and dispatches an ambulance if necessary.

\end{enumerate}

\item {\bf Connect to Relevant Peers}

\begin{enumerate}

\item John enjoys visiting the local library, however is concerned that he may suffer a fall which will go unnoticed, since the library is large and often empty.

\item John visits the library and connects his phone to the library's Wi-Fi network.

\item John finds the book he wants, but falls while reaching for it on a high shelf.

\item The system detects this fall but cannot connect John's heart monitor to the nurse's office. It searches for some application on the Wi-Fi network which can accept heart rate readings, and upon finding the librarian running such an application, connects John's heart monitor to the application.

\item The librarian sees that some kind of accident has happened, and can search for John or call an ambulance.

\end{enumerate}

\item {\bf Allow Communication Between Incompatible Schemas}

\begin{enumerate}

\item John uses a high end heart monitor which not only monitors his heart rate, but also blood pressure.

\item The retirement village supports this system so can use John's heart rate and blood pressure, however the librarian runs an outdated system which only supports heart rate.

\item The system converts John's sensor readings to match the library's system, making the two systems compatible.

\end{enumerate}

\end{itemize}

%TODO: more of this.
% non-functional requirements
Due to the ever changing nature of mobile environments, the system must also meet certain non-functional requirements to ensure that is is responsive and usable. 
The system must search for peers to connect to within a reasonable time, and any message conversion required for a given peer must incur negligible delay.
These constraints are necessary so that any communication between peers remains relevant to the user's context.

\section{SBUS}

SBUS \cite{ingram2009reconfigurable} is a messaging middleware which can be used for stream based communication. 
The basic entity in SBUS is a {\sl component}. 
Each component consists of one or more {\sl endpoints} which can be connected, or {\sl mapped} together allowing direct communication between the two endpoints. 
Every endpoint has an associated message type, and in order for two endpoints to be mapped together, their type must match. 
An endpoint may specify a subscription filter on events it wishes to receive. Any connected peer endpoints will only publish events to this endpoint if the event matches the subscription.

These connections and subscriptions are dynamically reconfigurable and may be modified by any other component, allowing thrid party remapping to occur, subject to security policies. 
A component may issue a message to some other component, instructing it to remap to another component's endpoints, or update its subscription filters.

The only central point of the system is a resource discovery component (RDC). 
The RDC itself is a component, and acts as a name service. 
Components may query the RDC for other components within the system, using parameters including component name, connected peers, or public key. 
The RDC will return IP addresses of matching components, allowing a direct connection to be established.

\section{System Design}

The system will be based upon the SBUS middleware, using and extending it to meet the requirements. 
Applications will use SBUS to send and receive messages.

% Android NDK
SBUS is written in C++ and will need to be ported for use on Android. 
The Android Native Development Kit\footnote{http://developer.android.com/tools/sdk/ndk/index.html} (NDK) will be used to compile SBUS for Android, and the Java Native Interface (JNI) framework will be used to allow applications to use this library. 
Since the NDK offers fairly obscure documentation with little more than trivial examples, some time will be required to understand how the build process works. This presents the largest unknown factor in the system, so will be the first work undertaken, to allow for modification to the project should more or less time be required than expected.

% How applications will find data sources relevant to them (constraints in terms of schema)
When components query the RDC they generally know something about the metadata of the component they want, such as its name or connected peers. 
In a mobile system, applications are unlikely to have this information available in every case, simply because there are too many possible environments. 
To allow applications to find relevant components in any environment, the system will be extended to allow queries centred around data.
Applications will be able to search by the format of the data they use, connecting to any component supporting that format without any other prior knowledge of that component.

% How applications will cope with unexpected data format - must be done at system level
SBUS endpoints can currently only communicate if they share a common message type. 
This ensures that endpoints only ever receive messages matching that type, so values can be extracted from the message through a known path. 
When searching for components by data, an application may be connected to some endpoint which contains the required message type nested somewhere within its own type.
% TODO: example
This means that an application may not know where a particular value is located within a message, so would have to perform some complex parsing to find the value. 
This is obviously undesirable since each environment might use a slightly different message type.
To avoid each application implementing such a parser, these differences will be handled at the system level, before the application receives the message. 
The system will reformat any incoming message to match the application message type, meaning an application will never face uncertainity in a message. 

% How change in context will be detected
A typical smartphone offers many sensors which can be used to determine a context change. 
Android allows applications to register to receive intents\footnote{http://developer.android.com/reference/android/content/Intent.html} which indicate some event such as a network change, or an incoming phone call. 
Furthermore, the AIRS application\footnote{https://play.google.com/store/apps/details?id=com.airs} can be used to obtain sensor readings.
The system can use these intents and sensor readings to monitor the environment, and determine context changes. 
Upon detecting a context change, it can then remap other applications on the phone connecting them to peers suited to the environmental context. 

% How applications' connections will be reconfigured (spoke map)
In order for the system to remap applications, it will first need to know about them. 
An RDC is well suited for this, because it maintains a list of registered components, which can then be searched when a component queries the RDC. 
This list can also serve as a list of components which the system can remap.

These two aspects can be combined into a Phone Manager (PM) application. The PM will act similarly to an RDC, allowing applications on the device to register with it. It will detect context changes and issue messages to registered components to instruct them to remap to other components upon these context changes.

% How the Phone Manager knows what to reconfigure to (policies)
The PM will remap applications upon a change in context, but must remap them to appropriate places. 
The PM cannot decide what constitutes ``appropriate'' for any given application, as such, the application must decide for itself. 
Applications will register {\sl policies} with the PM which dictate how they should be remapped upon a context change. 
This policy might be an IP address to map to, or a query which can be performed against an RDC in the environment to return matching components. 
The PM will simply remap applications as specified by their policies.

% Security
This project will not focus on security. 
A secure version of SBUS which uses OpenSSL to encrypt communications between components exists, and all of the work undertaken in this project could be applied to that version of SBUS.

\cleardoublepage


% ~ 4000 words
\chapter{Implementation}

This chapter consists of three main parts. 
The first shows how applications can find data sources which offer the data they require, and how they can use these sources despite differences in schemas. 
The second part explains how the SBUS library was ported to Android, and how it can be used by application developers. 
Finally, this is all brought together with the Phone Manager which is responsible for detecting changes in context and reconfiguring applications' connections upon a change.

\section{Flexible Schema Matching}

\begin{itemize}

\item Syntax: +H, +S, what they mean. Nested searches +S[+H]

\item How system handles these - hashes.

\item Constructing looking table from these.

\item Repacking messages from lookup table.

\item Updating subscription from lookup table.

\item How RDC searches for matching schemas. Alternative search methods, e.g. hash table of linked lists or search tree. What performance increases they may offer, and how the results would vary (i.e. a breadth first search would return the schema which matches the constraints at the highest level).

\item Optimisations - reordering constraints - also don't repeat constraints.

\end{itemize}

How accurate is the lookup table? 
Are any fields repackaged as ``empty'' when we could have filled them with something else? 
Fairly qualitative, perhaps a discussion about how the constraints are used to create this lookup table, and how multiple constraints may be fulfilled by a single field.

Should multiple constraints (e.g. must have two int fields) be allowed to match to the same field?

\section{Android Port}

\begin{itemize}

\item Android NDK (write how bad the documentation is)

\item Rewriting method to get IP address

\item Putting files (sbuswrapper) onto phone - issues using Android FS without rooting.

\item JNI

\end{itemize}

\section{Policy Engine/Phone Manager}

\begin{itemize}

\item Policy engine - mini-RDC for phone

\item AIRS-SBUS gateway - events as context change

\item Dynamically add/remove RDC (rdc\_update endpoint). Callback method for find out about RDC

\item Ideas for where to get this RDC from (currently set by user)

\item How applications set policies.

\item Applying policies.

\end{itemize}

\cleardoublepage


% ~ Eval + Conclusion ~ 2000 words
\chapter{Evaluation}

The project set out to create a system which would allow applications to automatically reconnect to different sources upon a change in context, where these sources may not necessarily present data in the format expected. 
The project has been successful in this respect. Sensor readings from AIRS and Android intents are used to detect a change in context. After detecting a change, applications will be connected to sources which satisfy any constraints they have specified and messages will automatically be converted to applications' expected format, allowing operation of the application to continue without interruption.

\section{System Overview}

% overview of system image
\begin{figure}[tbh]
%set the image x size to the width of the page
\epsfxsize=\hsize
\centerline{\epsfbox{figs/overview.eps}}
\caption[System Overview]{The light blue boxes represent components and the green boxes within them represent endpoints. The purple boxes are lists of components registered with the RDC and Phone Manager, and the orange box is a list of policies set on the Phone Manager}
\label{fig:system_overview}
\end{figure}

Figure \ref{fig:system_overview} shows the interactions that occur within the system, detailing how a transport application which uses the coach schema in Figure \ref{fig:coach-schema} might have its connection reconfigured upon a change in context.

\begin{enumerate}

\item After the transport application has registered with the Phone Manager, it can send policies to the Phone Manager which specify how its endpoints should be reconfigured upon a change in context. For instance, the application may wish to be connected to any components which offer a structure matching the type of {\tt coach}. In another case, it may simply give an IP address it should be connected to.

\item The AIRS-SBUS gateway acts as a server for sensor readings from AIRS, which then emits the readings as SBUS events, which are consumed by the Phone Manager. The Phone Manager also receives other intents from Android, signalling a change in Wi-Fi network.

\item The Phone Manager processes these events to determine whether a change in context has occurred. Once a change happens, the Phone Manager will inform the transport application about the change by sending it details of whether an RDC should be added or removed. 

\item The transport application receives this message about the RDC change and will automatically accept this RDC change, registering or deregistering with the RDC, or perform some callback method, perhaps prompting the user to make a decision. The transport application may have chosen to automatically accept the change, but display a message to the user to indicate that the context has changed.

\item The Phone Manager applies all of the policies within its policy table, sending messages to application endpoints, instructing them to connect to different components. The transport application will be sent whatever it has set in its policy.

\item If the transport application has specified an IP address in its policy, the endpoint will receive this IP address and can connect directly to it to start receiving messages. However, if the policy specified that a component must have a structure matching the type of {\tt coach}, the endpoint will query the RDC with this constraint to find matching components.

\item The RDC searches components registered with it for those matching the constraints. In this case, it might find only a component sending coach data, while in other cases it may find components offering train data. In both cases, these components match the constraints, so the RDC returns the list of matching components to the application.

\item The application endpoint then connects to one of the matching components, allowing the user to continue receiving fresh travel data without any interaction needed.

\end{enumerate}

% table of example policies
\begin{table}[tbh]
\centering

\begin{tabular}{c c c c}
\hline\hline
Component & Endpoint & Remote Address & Remote Endpoint \\
\hline

Healthcare & HeartRate & 128.232.0.20:44444 & HeartRate \\
Transport & Departures & +NStagecoach & Coach \\
Transport & Departures & +Scoach & NULL \\

\hline
\end{tabular}

\caption{Phone Manager Example Policies}
\label{tab:example_policies}
\end{table}

% about example policies
Table \ref{tab:example_policies} shows example policies which components might set with the Phone Manager. 
Both the local component and local endpoint are implicitly set as whichever endpoint sent the policy. 
The remote address may be an IP address or a query to be resolved by an RDC. 
An IP address may be useful when the application should always be mapped to the same IP address. 
In the first example, a healthcare system has registered to have heart rate readings mapped directly to an endpoint at a known IP address, perhaps a relative or emergency response team. 
However the real flexibility of the system lies in the queries. 
These will be resolved by some RDC, ensuring components are mapped only to other components satisfying their queries. 
In the second row, the transport application sets a policy for any component named ``Stagecoach'' which has an endpoint named ``Coach'', while in the third row it sets a policy for any endpoint which has a structure matching the type of {\tt coach}. 

% event-condition-action through specifying AIRS sensor + condition
The system is in line with an event-condition-action model, where events are the sensor readings and actions are mapping applications according to their policies. 
This could easily be extended to allow applications to specify which sensors to use as events, and conditions those events should fulfill in order to take an action.

% table of example policies with events and conditions
\begin{table}[tbh]
\centering

\begin{tabular}{c c c c}
\hline\hline
Event & Condition & Component & Remote Address \\
\hline

Accelerometer & \begin{math} val \ge50 \end{math} & Healthcare & 128.232.0.20:44444 \\
Time & \begin{math} 12 \le val.hour \le 14 \end{math} & Adverts & +IRestaurant \\
Wi-Fi & \begin{math} val = true \end{math} & Transport & +NStagecoach \\
GPS & \begin{math} location(val) = France \end{math} & Transport & +Scoach \\

\hline
\end{tabular}

\caption{Phone Manager Example Policies using Event-Condition-Action Model}
\label{tab:event_condition_action}
\end{table}

% about examples
As shown in Table \ref{tab:event_condition_action}, applications would not only specify the action they wish to take through their map constraints, but also the events and conditions in which they wish to take that action.
The healthcare system could specify that it only wishes for heart rate data to be connected to the IP address if a high accelerometer value was measured, because the reading may indicate that a fall has occurred. 
The transport application can set different policies for different events. If a Wi-Fi connection is established, the application wants to connect specifically to Stagecoach components, however if a GPS event indicates that the user were in France, the application is more lenient and willing to accept any component offering a structure matching the type of {\tt coach}. 
The application may also display adverts, where policies could be used to determine the source of the adverts, perhaps from nearby restaurants around lunchtime. 

% user policies
This could further be used to allow the user to set their own policies for components. 
The application developer cannot possibly account for every scenario, however by allowing the user to set their own policies, through which they are specifying context changes they are interested in, the system becomes fully customisable and personal. 
The user of the transport application might not want to wait for a bus when it is raining. 
Therefore, they could create a policy using a weather event from AIRS, with the condition ``rain''. 
The action might be to connect the application to the IP address of a local taxi company, which provides information about when the next taxi will be available using the schema in Figure \ref{fig:taxi-schema}.

%TODO: UI for user to set policy (system can also set policy).

%TODO: Screenshots of original message + repackaged?

%TODO: How long to go through policy list (all the if statements)

%TODO: How long to tell a component about context change - long list, might take a while for last one - prioritise?

\section{System Performance}

This section examines the performance of the system, specifically aspects related to the schema conversion part of the system. 
The schema conversion plays an important role in mobility because it allows applications to function despite some level of uncertainty, which in turn makes them more adaptable to different environments. 
This schema conversion can only be useful if it can be done in a reasonable time though, since the whole idea behind the system is to provide users with meaningful, {\sl real time} data. 
These tests show the kind of performance which can be expected from the system. 

% rdc search optimisation
\begin{figure}[tbh]
\epsfxsize=\hsize
\centerline{\epsfbox{figs/rdc_search_optimisations.eps}}
\caption{RDC Search Optimisations}
\label{fig:rdc_search_optimisations}
\end{figure}

Figure \ref{fig:rdc_search_optimisations} shows the time taken for an RDC to search through all registered components to find those matching schema constraints, for different numbers of registered components and with different optimisations applied. 
In a real environment, RDCs are likely to have many components registered, so one search cannot impact the performance of any other component too greatly.
As expected, the time taken to search increases linearly with the number of components registered. 
The structures first optimisation narrowly beats the exact first and exact structures optimisations, all of which peform more efficiently than searching without optimisation. 
These optimisations play a greater role as the number of components increases, becoming more important in order to prevent a bottleneck in the system. 
However, even in the worst case the RDC can perform the search with 100 other components registered in just over 40 milliseconds, so should not pose a problem given expected usage.

% construct lookup
\begin{figure}[tbh]
\epsfxsize=\hsize
\centerline{\epsfbox{figs/construct_lookup.eps}}
\caption[Construct Lookup Times]{Time taken to construct a lookup table between schemas of different sized producers and schemas of different sized consumers}
\label{fig:construct_lookup}
\end{figure}

Figure \ref{fig:construct_lookup} shows the time taken to construct a lookup table between schemas of different sized producers and schemas of different sized consumers. 
The lookup table is created before any messages can be exchanged between different schemas, therefore could cause messages to become stale if the process takes too long.
The general increase in time from left to right shows that more time is taken to construct a lookup table as the size of the producer schema increases, as expected since there are more fields. 
The greater values in the upper right half of the graph show a disparity with the smaller values in the lower left half of the graph. 
This shows that the time taken to construct a lookup table is greater when the producer's schema is larger than the consumer's schema, corresponding to searching in the larger producer schema for the smaller consumer schema to be matched, as opposed to identifying the missing outer structures if the consumer schema were larger.
In either case, the lookup table is constructed once per connection, and takes microseconds.

% repack message
\begin{figure}[tbh]
\epsfxsize=\hsize
\centerline{\epsfbox{figs/repack_message.eps}}
\caption[Repackage Message]{Time taken to repackage a message between schemas of different sized producers and schemas of different sized consumers}
\label{fig:repack_message}
\end{figure}

Figure \ref{fig:repack_message} shows the time taken to repackage a message between schemas of different sized producers and schemas of different sized consumers. 
This repackaging happens once per message, thus causes a delay on every single message received. 
The general increase in time from top to bottom shows that more time is taken when the consumer schema is larger, as expected because more fields are being repackaged under the appropriate names. 
As with constructing the lookup table, messages are repackaged within microseconds, meaning that any message will likely still be relevant after the time taken for repackaging.

Figures \ref{fig:construct_lookup} and \ref{fig:repack_message} both use a schema constraint which matches the type of the smallest schema.

\section{Android Port}

This section evaluates the performance of using SBUS on a phone in comparison to using SBUS on a laptop, investigating whether any significant gains or losses occur as a result of the port.

% table showing map times
\begin{table}[tbh]
\centering

\begin{tabular}{c c c c}
\hline\hline

Consumer & Producer & Mean \begin{math} (\mu s) \end{math} & Standard Deviation  \begin{math} (\mu s) \end{math} \\
\hline

Laptop	&	Laptop	& \begin{math} 0.7979\times10^4	\end{math} & \begin{math} 0.9787\times10^4 \end{math} \\
Laptop	&	Phone	& \begin{math} 2.0824\times10^4	\end{math} & \begin{math} 2.0064\times10^4 \end{math} \\
Phone	&	Laptop	& \begin{math} 2.1834\times10^4	\end{math} & \begin{math} 1.5705\times10^4 \end{math} \\
Phone	& Phone	& \begin{math} 1.0933\times10^4	\end{math} & \begin{math} 0.2810\times10^4 \end{math} \\
\hline
\end{tabular}

\caption{Connection Times Between Components on Different Devices}
\label{tab:map_times}
\end{table}

% figure showing map times
\begin{figure}[tbh]
\epsfxsize=\hsize
\centerline{\epsfbox{figs/map.eps}}
\caption{Connection Times Between Components on Different Devices}
\label{fig:map_times}
\end{figure}

Table \ref{tab:map_times} and Figure \ref{fig:map_times} show mean times and standard deviations for consumers to connect to producers, across phones and laptops.
Each consumer first queries an RDC on a remote machine by the producer name, to find the appropriate producer to connect to. 
Once the consumer receives the IP of the producer from the RDC, it then establishes the connection with it, hence two connections occur within each mapping. 
No schema conversion occurs as part of this map. 
These figures show that a greater time is required when the producer and consumer are on different devices than when they are on the same device. 
This bears no surprise since the local connections are faster than the network connections. 
The mean time taken is slightly greater when both the producer and consumer are on the phone rather than both on the laptop, though the standard deviation is less. 
Overall, an application written in Java running a phone would see very similar network delays to that of a program written in C++ running on any other machine.

% figure showing jni times
\begin{figure}[tbh]
\epsfxsize=\hsize
\centerline{\epsfbox{figs/jni.eps}}
\caption{Function Call via JNI Timing Breakdown}
\label{fig:jni_times}
\end{figure}

Figure \ref{fig:jni_times} shows the breakdown of time taken to perform an SBUS function call via Java compared to C++. 
In this case, the function call is to emit a message. 
This clearly shows that using JNI to call a native method from Java introduces a delay, almost five milliseconds here. 
Each call to an SBUS function via Java will incur some delay, meaning that applications written in Java for a phone will never operate as quickly as those written in C++.
Application developers can expect this delay, which is likely to show less variance than any network delay, and build their applications to take account of it. 
Provided an application is calling SBUS functions less frequently than this delay, such as a emitting a message, it is unlikely to cause any problems.

%TODO: Stress test? How many components on Android can communicate?

%TODO: Example code for Android SBUS component?

%TODO: Screenshots?

\cleardoublepage

\chapter{Conclusion}

% summarise
The project has created a context-aware system which enables developers to write applications without prior knowledge of the location of data sources or the format of their data, decoupling applications from their data sources. 
The system supports the goals of pervasive computing by allowing scalability in the data sources used, heterogeneity in the data applications use, all while being invisible to the user.

The ``policy engine'' monitors the environment for events which the system or user believe to represent a change in context.
Upon detection of an event, the policy engine applies applications' policies, reconfiguring their connections as appropriate. 
Through these policies, applications may control which data sources they should be connected to, without requiring applications to find data sources themselves.
Policies may be simple, such as the name of a data source, or more complex, specifying how fields in a data source's schema must match their own. 

The system repackages messages so that applications will only ever be presented with data conforming to their own schema, removing uncertainty about the format of the data.
Furthermore, applications may specify subscription filters in terms of their own schemas, placing restrictions on which messages they will be sent, without requiring knowledge of peers' schemas.
When reconfiguring a connection, the system translates these filters to match the peer's schema and applies them to the peer, filtering messages as dictated by the application.

% example
Referring back to the transport example from the Introduction, there may be a transport application based on the coach schema (Figure \ref{fig:coach-schema}) which requires data sources to offer a structure matching the type of the {\tt coach} structure. 
At a coach station, all data sources match this schema exactly, so the application can connect to any one of them. 
The user may then travel to a larger station, offering train and coach services. 
Although train data sources use a different schema (Figure \ref{fig:train-schema}) to the application, their schema contains a structure matching the required {\tt coach} structure.
Upon arrival at the station, the user connects to the Wi-Fi network. The policy engine detects this change in context and connects the application to a train data source. 
Messages will be repackaged to match the application schema, meaning that the same code handles messages for coaches or trains.
The user might have subscribed to only receive data for which the departure time is after midday. 
Since this subscription is specified in terms of the coach schema, it will not function correctly if applied directly to the train schema. 
In this case, the system reformats the subscription to match the train schema and applies it when connecting the application to the data source, ensuring the user's subscription filter is met.

The application might equally have used the train schema, yet only required the {\tt info} structure, allowing it to use coach data sources.

\section{Issues}
% remaining problems
The system currently faces issues when context is rapidly changing, for example, moving in and out of range of a Wi-Fi network. 
The network connection drops causing the system to reconfigure connections, which may disconnect applications from their peers.
However, the network connection is almost immediately re-established causing the system to reconfigure connections again, possibly connecting applications to these same peers. 
If this reconnection occurs before disconnection has completed, the application believes itself to still be connected to the peer, thus will not create a new connection.

A solution to this problem might be to implement constraints which must be satisfied before reconfiguration occurs. 
In a Wi-Fi network, there may be a constraint that the network signal must be greater than a given threshold, meaning no reconfiguration would occur until the user moves firmly into the network. 
Alternatively, conditions on multiple context changes might be required before performing reconfiguration.  
For an accelerometer context change, a high accelerometer message followed by a message indicating little movement might be used to detect a fall.

\section{Changes}
%TODO: what I'd have done differently
{\sl I think this section might be more obvious after writing in the Preparation chapter about how the project was planned. }


\section{Future Work}
% other stuff that could be done

% alternate events for change in context - AIRS
Applications cannot currently specify changes in context which they are interested in. 
The policy engine will trigger all policies upon detecting a change.
Applications could specify the type of context change within their policy, giving them the ability to specify different policies for different context changes. 
This moves the system to an event driven architecture, where context changes are represented as events. 
Upon receiving an event, the policy engine would check whether the event satisfies the conditions of a policy, only applying the policy when it does. 
This allows applications to specify policies for simple events, such as a Wi-Fi change event, to more complex events, such as a high accelerometer reading followed by no movement to represent a fall. 

As well as applications specifying policies, users could also specify policies for different applications. 
This moves the system further towards the pervasive vision. 
Applications would take sensible actions for different changes in context, however the user could refine how applications should react to changes in context, causing them to behave in a way more suited to the user's needs.

The user may have the ability to specify which types of events can be used. 
This offers some privacy to the user by allowing them to block applications from knowing if some part of their context has changed.

% qualify names if multiple fields in sensor have same name
The constraints applications specify about the schema of data sources they may be connected to are currently specified using the field names of their own schema.
While this works provided there are no repeated fields names within the schema, it would fail on a schema such as Figure \ref{fig:repeatednameschema}. 
A solution to this would be to qualify all names specified in the constraints in a similar way to subscriptions. 
We could specify a constraint for {\tt sensor}, meaning the entire structure, or {\tt sensor/sensor}, meaning the text field.

\begin{figure}[tbh]
\begin{lstlisting}
@sensor
{
	txt sensor
	int value
}
\end{lstlisting}
\caption[Schema with repeated name]{A schema with a structure named {\tt sensor} and a text field also named {\tt sensor}. A constraint specified on {\tt sensor} would be ambiguous without fully qualified names}
\label{fig:repeatednameschema}
\end{figure}

\cleardoublepage

%%%%%%%%%%%%%%%%%%%%%%%%%%%%%%%%%%%%%%%%%%%%%%%%%%%%%%%%%%%%%%%%%%%%%
% the bibliography

\addcontentsline{toc}{chapter}{Bibliography}
% nocite means add all references even if not cited.
\nocite{*}
\bibliography{refs}
\cleardoublepage

%%%%%%%%%%%%%%%%%%%%%%%%%%%%%%%%%%%%%%%%%%%%%%%%%%%%%%%%%%%%%%%%%%%%%
% the appendices
\begin{appendix}

\chapter{Possible Titles}

\begin{itemize}
\item Automatic Communication Configuration in Mobile Environments
\item Data Stream Connection and Message Negotiation
\item Flexibile Data Streams in Mobile Environments
\item Automatic Data Stream Configuration in Mobile Environments
\end{itemize}

\cleardoublepage

\chapter{Example Schemas}

\begin{figure}[h]
\begin{lstlisting}
@coach
{
    int bay
    date departure
    stops
    (
    	town
    	{
    		txt name
    		date arrival
    	}
    )
}
\end{lstlisting}

\caption[Example Schema for a Coach]{A coach schema, showing a stand number, a date and time of departure, and a list of stops, with town names and estimated arrival times}
\label{fig:coach-schema}
\end{figure}

\begin{figure}[h]
\begin{lstlisting}
@train
	info
	{
		int platform
		date departure
		stops
		(
			station
			{
				txt name
				date arrival
			}
		)
	}
	shop
	{
		items
		(
			txt name
			dbl price
		)
	}
}
\end{lstlisting}

\caption[Example Schema for a Train]{A train schema, showing a structure containing a stand number, a date and time of departure, and a list of stops, with names and estimated arrival times, as well as an additional structure containing a list of items for sale in the on-board shop}
\label{fig:train-schema}
\end{figure}

\begin{figure}[h]
\begin{lstlisting}
@autobus
{
    int aire-de-stationnement
    date depart
    arrets
    (
    	ville
    	{
    		txt nom-de-la-ville
    		date arrivee
    	}
    )
}
\end{lstlisting}

\caption[Example Schema for a French Coach]{A French coach schema, showing the same fields as the English coach schema, however the field names are in French}
\label{fig:french-coach-schema}
\end{figure}

\begin{figure}[h]
\begin{lstlisting}
@taxi
{
	date available
}
\end{lstlisting}

\caption[Example Schema for a Taxi]{A taxi schema, simply showing the next available taxi}
\label{fig:taxi-schema}
\end{figure}


\cleardoublepage

% should this be an appendix?
\chapter{Project Proposal}

\vfil

\centerline{\Large Computer Science Part II Project Proposal }
\vspace{0.4in}
\centerline{\Large Automated Data Stream Management on Mobile Devices }
\vspace{0.4in}
\centerline{\large Thomas Smith (tcs40), Sidney Sussex College}
\vspace{0.3in}
\centerline{\large Originator: Jatinder Singh}
\vspace{0.3in}
\centerline{\large 17$^{th}$ October 2012}

\vfil


\noindent
{\bf Project Supervisor:} Jatinder Singh
\vspace{0.2in}

\noindent
{\bf Director of Studies:} Chris Hadley
\vspace{0.2in}
\noindent

\noindent
{\bf Project Overseers:} David Greaves  \& John Daugman


% Main document

\section*{Introduction and Description of the Project}

A typical approach for developing a smartphone app currently includes connecting to some central server over the Internet, and requesting data from it. An alternative approach would be to use data streams, where a data source and data sink are connected together, and the source automatically pushes data to the sink. Connections can then be made on an ad-hoc basis, allowing us to connect purely within a local network, or across the Internet.

When considering mobile devices, the available data sources which can be discovered will largely depend upon the context and scope in which the device is operating. This context can be affected by many factors, including location, or 3G/WiFi networks which the device is connected to. Sources may have their scope limited to a local network, such that they are only visible on that network, meaning once we go outside that scope, we will have to find a new source.

The project will investigate how a messaging middleware which handles both stream-based and request/response based interactions can be adapted to allow automated discovery of components. If a change in context causes a connected data source to become out of scope or irrelevant, the middleware would automatically search for other sources, and if possible reconnect to one providing the relevant service, allowing use of the app to continue seamlessly.

An example of this is transportation. If we were outside Cambridge Railway Station, our location could be found using GPS, and the middleware could connect over 3G to a source providing details of upcoming departures. Upon boarding a train to London, and connecting to on-board WiFi, we no longer care about other departures - the context has changed, and we only want information relating to our train. The middleware would automatically find and connect to another source, perhaps giving arrival times for stations along the route. Upon arriving at King's Cross Station, we disconnect from the WiFi network which means the previous source is now out of scope. We could connect to King's Cross WiFi, and the middleware would find and connect to sources giving data about the London Underground, and/or National Rail trains. If we were then to get on an Underground train, even though we would not be able to establish a 3G connection, there could still be a source on board the train, providing data over Bluetooth, meaning use of the app can continue.

Component discovery can be done in two ways. The first essentially uses a directory, which we can query. The second is by inspection of other components - given a component, we can ask which other components in the network is it linked to. The project will investigate using this second method to build a graph of connected components, which could be used to find a specific component.

In order to connect to a component providing the correct function, it will be necessary to determine whether a service is relevant. This will involve more than comparing message types. As show in figure \ref{fig:differentSchemas}, both schemas show data offering the same service, but with different fields. The aim is to enable an app expecting data conforming to schema (a) to also handle data from sources serving schema (b).

\begin{figure}[h]
\begin{subfigure}[b]{0.5\textwidth}
\begin{verbatim}
weather {
    dbl temperature;
}
\end{verbatim}
\caption{A simple weather schema.}
\end{subfigure}
\begin{subfigure}[b]{0.5\textwidth}
\begin{verbatim}
weather {
    dbl temperature;
    dbl humidity;
    int rainfall;
}
\end{verbatim}
\caption{A more complicated weather schema.}
\end{subfigure}
\caption{Two different schemas, both serving weather data.}
\label{fig:differentSchemas}
\end{figure}

\section*{Starting Point}

SBUS\footnote{formerly PIRATES, http://www.cl.cam.ac.uk/research/time/pirates/docs/overview.pdf} is a messaging middleware which supports client/server and stream based communication. In using a data stream, sources and sinks can be connected in a peer-to-peer network, with optional filters on events the source wishes to receive.

All of this is dynamically reconfigurable - one component can tell another to open or close connections to another component (subject to security policies). Message filters can also be changed at run time.

A component can handle multiple different message types, where each type corresponds to a different endpoint. Network connections are made to the component, and the endpoint is then specified by name. For two endpoints to communicate, their message types must match. This type is currently specified in a component file.

Component discovery can be done though a resource discovery component (RDC). A component can query the RDC using a component or endpoint name, and the RDC will then link the component to the appropriate partner. All components can be queried for identity, schemas, status, connected peers, given that the component is known.


\section*{Work to be Undertaken}

The project breaks down into the following sub-projects:

\begin{enumerate}

\item Port the existing SBUS library to Android using the Android NDK. Write an example app - a device will act as a source and transmit sensor measurements recorded with AIRS\footnote{https://play.google.com/store/apps/details?id=com.airs}, which a computer can then connect to.

\item Implement a recursive method for discovery of components through inspection. We can currently inspect a known component to see what other components are connected to it - this would extend this functionality to inspect the entire network. This would allow discovery of all other components on a network through one known component, meaning that the network can be searched without using a central resource. This will require use of graph traversal algorithm to map out the network, and an efficient data structure to store the topology of the network.

\item Connection to a new component may mean that a previously unseen data schema is being used. This will require the device to ``learn'' the new schema. This unknown schema could simply be a more, or less detailed version of a previous schema. May require use of reflection, or implementation of some handshaking protocol to exchange schemas.

\item To decide whether a new component is relevant, we need to know what service it is providing. This will require investigating search criteria in order to discover components based on the data they handle. To allow partially compatible schemas (see figure \ref{fig:differentSchemas}), we will investigate the use of ontologies to classify message types into meaningful categories.

\item Detect and act upon a change in context - search for new components using a RDC or inspection and where appropriate connect the device to these components. This will require monitoring various sensors on the device, such as location and connected networks in order to detect a change in context.

\item Create apps demonstrating the use of this system, and record metrics from the device, which can be used as a comparison against other implementations.


\end{enumerate}

\section*{Success Criterion for the Project}


The project will be a success if I can demonstrate that a mobile device can automatically reconnect to a different component upon a change in context, similar to the example outlined in the Introduction.

We can compare the two methods of component discovery, to determine which is more appropriate in different scenarios. We can measure how effectively a component can accept partial schemas - whether performance deteriorates when a message contains far more fields than expected.

The project can further be evaluated by comparison of an app developed using this system, to one developed using other approaches. There are many metrics which can be measured for comparison. From a performance viewpoint, we can measure overheads such as power consumption, network usage, and time taken to reconnect to a service.
In development we can measure lines of code and the reusability of code - one app could even have multiple use cases! For example, one travel app, as described in the Introduction would replace the need for separate National Rail and London Underground apps.



\section*{Possible Extensions}

If the project were completed early, I could look at dynamic visibility of components. Components can specify security policies restricting access to themselves, which also prevents them being discovered. If this access control were to change after searching for the component, we would still have no knowledge of it unless searching again. This extension would look at how a component could be made aware, or even a connection automatically established to the previously ``hidden'' component, if the access control changes to allow us to discover it. This could be useful if we wanted to grant access to a larger group of users, from a previously prioritised group.


\section*{Timetable and Milestones}

Planned starting date is 19/10/2012.

\begin{enumerate}

\item {\bf Michaelmas weeks 3-4} Get SBUS in existing form running on Android, either by creating an interface to the native code, or writing an app entirely in native code. Write example sensor app.

\item {\bf Michaelmas weeks 5-6} Investigate algorithms to build a graph of the network of components, and implement an appropriate one for recursive component inspection to allow decentralised component discovery.

\item {\bf Michaelmas weeks 7-8} Implement a method to search for a new component when necessary. This will also include detecting when it is necessary - detect change of network, location change.

\item {\bf MILESTONE: } Automatic reconnection to component complete.

\item {\bf Michaelmas vacation} Write code to allow searching for components by type. This will be done in a way such that more general versions of a schema can be found, if a specific one cannot.

\item {\bf Lent weeks 0-2} Extend current connection method to allow components to connect to schemas providing the correct service, but with different fields.

\item {\bf MILESTONE: } Dynamic data typing complete.

\item {\bf Lent weeks 3-5} Create example apps demonstrating the project and apps using a client/server mechanism, and measure overheads of using these in order to evaluate the project.

\item {\bf Lent weeks 6-8} Finish any remaining programming.

\item {\bf MILESTONE: } Programming complete.

\item {\bf Easter vacation:} Write draft dissertation.

\item {\bf Easter weeks 0-2:} Write final copy of dissertation, and ensure all demos are working.

\item {\bf Easter week 3:} Early submission, allowing one week in case the project overruns.

\item {\bf MILESTONE: } Project complete.

\end{enumerate}

\section*{Resource Declaration}

For this project I will use my own computer running Ubuntu 12.04. I accept full responsibility for this machine and I have made contingency plans to protect myself against hardware and/or software failure. Backup will be to a repository hosted by GoogleCode, as well as copies to my Dropbox folder, and to USB drives. I will require use of Android devices for testing - either my own, or provided by the Computer Lab.

I will use the Android SDK\footnote{http://developer.android.com/sdk/index.html} and Android NDK\footnote{http://developer.android.com/tools/sdk/ndk/index.html} to develop Android applications with the ability to run native code.

I will be using the SBUS library as the data stream middleware, provided by Jat Singh.

Should my machine suddenly fail, I will use the PWF computers running Linux. 


\end{appendix}

\end{document}
