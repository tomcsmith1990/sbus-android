\documentclass[12pt,twoside,notitlepage]{report}

\usepackage{a4}
\usepackage{verbatim}
\usepackage{caption}
\usepackage{subcaption}
\usepackage{listings}

% set tab size for lstlisting
\lstset{tabsize=4}

\input{epsf}                            % to allow postscript inclusions
% On thor and CUS read top of file:
%     /opt/TeX/lib/texmf/tex/dvips/epsf.sty
% On CL machines read:
%     /usr/lib/tex/macros/dvips/epsf.tex



\raggedbottom                           % try to avoid widows and orphans
\sloppy
\clubpenalty1000%
\widowpenalty1000%

\addtolength{\oddsidemargin}{6mm}       % adjust margins
\addtolength{\evensidemargin}{-8mm}

\renewcommand{\baselinestretch}{1.1}    % adjust line spacing to make
                                        % more readable

\begin{document}

\bibliographystyle{plain}


%%%%%%%%%%%%%%%%%%%%%%%%%%%%%%%%%%%%%%%%%%%%%%%%%%%%%%%%%%%%%%%%%%%%%%%%
% Title


\pagestyle{empty}

\hfill{\LARGE \bf Thomas Smith}

\vspace*{60mm}
\begin{center}
\Huge
{\bf Automatic Configuration in Mobile Environments} \\
\vspace*{5mm}
Computer Science Tripos Part II \\
\vspace*{5mm}
Sidney Sussex College \\
\vspace*{5mm}
\today  % today's date
\end{center}

\cleardoublepage

%%%%%%%%%%%%%%%%%%%%%%%%%%%%%%%%%%%%%%%%%%%%%%%%%%%%%%%%%%%%%%%%%%%%%%%%%%%%%%
% Proforma, table of contents and list of figures

\setcounter{page}{1}
\pagenumbering{roman}
\pagestyle{plain}

\chapter*{Proforma}

{\large
\begin{tabular}{ll}
Name:               & \bf Thomas Smith	\\
College:            & \bf Sidney Sussex College	\\
Project Title:      & \bf Automatic Configuration \\ &\bf in Mobile Environments	\\
Examination:        & \bf Computer Science Tripos Part II, 2013 	\\
Word Count:         & \bf 2466\footnotemark[1] \\
Project Originator: & Dr Jatinder Singh		\\
Supervisor:         & Dr Jatinder Singh		\\ 
\end{tabular}
}

\footnotetext[1]{This word count was computed by {\tt detex diss.tex | tr -cd '0-9A-Za-z $\tt\backslash$n' | wc -w}}
\stepcounter{footnote}


\section*{Original Aims of the Project}

The original aim of the project was to create a system which could adapt to the changes in context which come with mobility, such as location or networks. 
This required the ability to detect these changes, in order to find and connect to new data sources. 
As a new data source may have different data schema to the expected schema, the system must offer a way to make meaningful use of this data in relation to the expected schema.

\section*{Work Completed}

A messaging middleware was modified to allow users to search for data sources by aspects of their schema. Users can search for an exact match on some part of a schema, or a looser match which only considers field types, and not names.
Once this middleware was ported for use on Android, an application was created to monitor changes in context, particularly those that arise as a result of mobility. Upon detection of a change in context, the application informs other applications of the change, or automatically reconfigures their connections to connect them to a new data source.

\section*{Special Difficulties}

The main difficulty in the project came from porting the messaging middleware to Android. The Android NDK\footnote{http://developer.android.com/tools/sdk/ndk/index.html} toolkit was used in order to compile C++ for use on Android, however missing libraries and poor documentation made this task more difficult than expected.
 
\newpage
\section*{Declaration}

I Thomas Smith of Sidney Sussex College, being a candidate for Part II of the Computer
Science Tripos, hereby declare that this dissertation and the work described in it are my own work,
unaided except as may be specified below, and that the dissertation
does not contain material that has already been used to any substantial
extent for a comparable purpose.

\bigskip
\leftline{Signed: }

\bigskip
\leftline{Date: }

\cleardoublepage

\tableofcontents

\listoffigures

\newpage

%%%%%%%%%%%%%%%%%%%%%%%%%%%%%%%%%%%%%%%%%%%%%%%%%%%%%%%%%%%%%%%%%%%%%%%
% now for the chapters

\cleardoublepage        % just to make sure before the page numbering
                        % is changed

\setcounter{page}{1}
\pagenumbering{arabic}
\pagestyle{headings}

% Intro + Preparation ~ 1400 words
\chapter{Introduction}

% summarise
While many applications now exist for mobile devices, few applications themselves are truly mobile. 
The user may happen to be mobile, however the application does not adapt to different environments. 
This project addresses the issues faced in creating applications which can change to suit different environments. 
I show how relevant data sources in an environment can be found for applications, and how flexibility in the data they use can be achieved. 

\section{Problem}
% what's the problem
% describe current approach
Different environments mean that devices operate in different contexts and scopes, which dictate how a device can interact with the environment. 
% describe context more, then say you're mainly focusing on mobility
Context may consist of both physical attributes, such as available Wi-Fi networks and location, as well as psychological aspects such as the user's behaviour or mood.
For example, the device may deduce from the accelerometer that the user is running, in which case the user may be in a rush and require different data than if they were sat calmly.
Scope is also an important factor, since data sources may only be available on a particular network, or within a given time period. 
Changes in context and scope mean that an application may no longer be able to communicate with a given data source, or the data it is providing may no longer be useful. 
In this situation, an alternative data source must be found.

Once this alternate data source has been found, the application must still determine whether it provides useful data. 
Some data may be absolutely necessary for operation of the application, while other parts may be desired, but operation can continue without them. 
However, even if a data source is deemed useful, the application may still have the difficulty of the data being presented in an alternate format to the expected format. 
Some fields may be missing, there may be additional fields, and some fields may have different names. 
The application may also wish to specify some subscription filter about the messages it should receive, yet without knowledge of the data format, it cannot specify constraints on different fields.

% example
For example, a transport application may face differences in data available for coaches (Figure \ref{fig:coach-schema}) and for trains (Figure \ref{fig:train-schema}). 
While both offer similar data to a large extent, fields are named differently, and the train schema also offers a menu for an on-board shop.
Furthermore, if the application were to be used abroad, for example in a French coach station, the data will still be very similar, but fields names may be in French (Figure \ref{fig:french-coach-schema}).

While these differences may be important when the user is sat waiting in a station, they become less important if the user is running to catch a coach or a train and simply requires the departure time and location.
Given the amount of uncertainty about what data will be available, a developer creating this application cannot hope to account for all possibilities. 
However, the developer can specify how the application should behave in different contexts.

\section{Project}

% what I've done
Schilit et al \cite{Schilit:1994:CCA:1439278.1440041}  define a context-aware system as one which can examine the computing environment and react to changes in the environment. 
This project creates such a system, which performs automatic contextual reconfiguration for applications whenever there is a change in the environment. 
Pascoe \cite{Pascoe:1998:AGC:857199.858020} lists four context-aware capabilites, two of which occur in this reconfiguration. 
Contextual resource discovery is used to discover data sources available within the context, and contextual adaptation allows applications to adapt to these data sources' schemas in order to use the sources.

A ``mobile policy engine'' is the part of the system which performs the automatic reconfiguration. 
Applications control how their connections should be reconfigured through policies. 
These policies may specify the name of a data sources, or be more complex rules regarding the structure of data sources' schemas. 
The policy engine monitors the environment, and upon detecting a change applies these policies, connecting applications to different data sources where appropriate.
Alternatively, applications may opt to be notified of a relevant change, rather than have their connection reconfigured, allowing them to perform callback methods upon a change.

Depending on the policy, an application may be connected to a data source which does not offer the expected data schema. 
The system handles this by repackaging messages to match the application's schema. Any extraneous fields are removed, while any absent fields are filled with ``empty'' values. The policy is used to infer relationships between fields, even if they have different names. 
Once the system has repackaged the message, it passes it to the application, allowing the application to continue as if it had received a message conforming to its schema.

The system allows an application to specify subscription filters in terms of its own schema. 
As part of the reconfiguration process the system converts this subscription to match the data source's schema, connecting the application with a working subscription in place.

\cleardoublepage

 
\chapter{Preparation}


\cleardoublepage


% ~ 4000 words
\chapter{Implementation}


\cleardoublepage


% ~ Eval + Conclusion ~ 2000 words
\chapter{Evaluation}

\begin{itemize}
\item
{\bf TEST: } How long does the RDC take to find a matching schema (at the end of the list of components) when there are 0, 10, 50, 100 other components? How much faster is it with optimisations to put structure constraints or exact constraints (or both) first?\\
{\bf REASON: } In reality, will have some other components, don't want too big a delay, especially if it affects other components. Could optimise searches with a hash table?

\item
{\bf TEST: } How long does it take for components to have connections reconfigured after connecting to a new Wi-Fi network?\\
{\bf REASON: } If a lot of components are registered on the phone, how long does it take to get to the last one? Perhaps some priority is needed?

\item
{\bf TEST: } How long does it take to repack messages for: similar size schemas (e.g. just different names), much larger schemas (removing fields), much smaller fields (adding fields)?\\
{\bf REASON: } Some applications may be time sensitive (e.g. stock market). Don't want to wait ages for each message to be repacked.

\item
{\bf TEST: } How long does it take to construct a lookup table for: similar size schemas, much larger schemas, much smaller fields?\\
{\bf REASON: } Again, time sensitive.

\item How accurate is the lookup table? Are any fields repackaged as ``empty'' when we could have filled them with something else? Fairly qualitative, perhaps a discussion about how the constraints are used to create this lookup table, and how multiple constraints may be fulfilled by a single field. Possibly more suited to Implementation or Conclusion?

\item Should multiple constraints (e.g. must have two int fields) be allowed to match to the same field?

\item Some screenshots? Perhaps just showing a repackaged message being passed to a consumer and the original sensor message?

\item Example code for Android consumer, showing how easy flexibility is?

\item Why the recursive discovery through peer inspection hasn't been implemented, and how it could be (should be somewhere else?)

\item Alternative search algorithms for schema comparison in RDC. What performance increases they may offer, and how the results would vary (i.e. a breadth first search would return the schema which matches the constraints at the highest level).

\item Table of example policies, how reconfiguration works.
\end{itemize}


\cleardoublepage

\chapter{Conclusion}

% summarise
The project has created a context-aware system which enables developers to write applications without prior knowledge of the location of data sources, or the format of their data. 

Upon detecting a change in context, the ``mobile policy engine'' applies applications' policies, reconfiguring their connections as appropriate. 
These policies may be simple, such as the name of a data source, or more complex, specifying how fields in a data source's schema must match the application's own, in terms of exact matches, type-matches, or uninterested. 
Through these policies, applications can place restrictions on which data sources they can be connected to, without requiring applications to find data sources themselves.

Applications will only ever be presented with data conforming to their own schema, removing uncertainty about the format of the data.
Furthermore, applications may specify subscription filters in terms of their own schemas, placing restrictions on which messages they will be sent, without requiring knowledge of peers' schemas.
When reconfiguring a connection, the system translates these filters to match peer's schema and applies them to the peer, filtering messages as dictated by the application.

% example
Referring back to the transport example from the Introduction, there may be an application using the coach schema (Figure \ref{fig:coach-schema}) which requires data sources to offer a structure matching the type of the {\tt coach} structure. 
At a coach station, all data sources will match this schema exactly, so the application can connect to any one of them. 
Upon arriving at a larger station and connecting to a Wi-Fi network, the system detects a change in context, and consequently connects the application to a new data source. 
The user may be changing to a train here, however information about trains use a different schema (Figure \ref{fig:train-schema}). 
This schema does not match the application schema exactly, however it does contain a structure matching the necessary {\tt coach} structure. 
Therefore, the system can connect the application to this data source and repackage messages to match the application schema, allowing the same application code to handle the message. 
The user could have subscribed to only receive data for which the departure time is after midday. 
Since this subscription is specified in terms of the coach schema, it will not function correctly if applied directly to the train schema. 
In this case, the system reformats the subscription to match the train schema and applies it when connecting the application to the data source, ensuring the user's subscription filter is met.

The application might equally have used the train schema, yet only required the {\tt info} structure, allowing it to use coach data sources.

\section{Issues}
% remaining problems
The system currently faces issues when context is rapidly changing, for example, moving in and out of range of a Wi-Fi network. 
The network connection drops causing the system to reconfigure connections, which may disconnect applications from their peers.
However, the network connection is almost immediately re-established causing the system to reconfigure connections again, possibly connecting applications to these same peers. 
If this reconnection occurs before disconnection has completed, the application believes itself to still be connected to the peer, thus will not create a new connection.

A solution to this problem might be to implement constraints which must be satisfied before reconfiguration occurs. 
In a Wi-Fi network, there may be a constraint that the network signal must be greater than a given threshold, meaning no reconfiguration would occur until the user moves firmly into the network. 
If a context change depended on the accelerometer, it might be that the accelerometer must detect some pattern for several seconds before reconfiguration.

\section{Changes}
% what I'd have done differently
{\sl I think this section might be more obvious after writing in the Preparation chapter about how the project was planned. }


\section{Future Work}
% other stuff that could be done

The system currently depends solely on a change in network status to detect a change in context. 
The Android platform offers a whole host of different events\footnote{http://developer.android.com/reference/android/content/Intent.html} and sensors\footnote{http://developer.android.com/guide/topics/sensors/sensors\_overview.html} which could be used to determine a change in context, such as the battery level being low, the timezone changing, or accelerometer readings. 
The system could be extended to use more of these features to determine a change in context, allowing applications to specify how they want to react to different types of change in context. 
In addition to applications specifying the types of change in context they are interested in, the user may have the ability to specify which types of context change are allowed to be used. 
This offers some privacy to the user by allowing them to block applications from knowing if some part of their context has changed.

The constraints an application specifies about the schema of data sources it may be connected to are currently specified using the field names of the application schema.
While this works provided there are no repeated fields names within a schema, it would fail on a schema such as Figure \ref{fig:repeatednameschema}. 
A solution to this would be to qualify all names specified in the constraints in a similar way to subscriptions. 
We could specify a constraint for {\tt sensor}, meaning the entire structure, or {\tt sensor/sensor}, meaning the text field.

\begin{figure}[h]
\begin{lstlisting}
@sensor
{
	txt sensor
	int value
}
\end{lstlisting}
\caption[Schema with repeated name]{A schema with a structure named {\tt sensor} and a text field also named {\tt sensor}. A constraint specified on {\tt sensor} would be ambiguous without fully qualified names.}
\label{fig:repeatednameschema}
\end{figure}

\cleardoublepage

%%%%%%%%%%%%%%%%%%%%%%%%%%%%%%%%%%%%%%%%%%%%%%%%%%%%%%%%%%%%%%%%%%%%%
% the bibliography

\addcontentsline{toc}{chapter}{Bibliography}
% nocite means add all references even if not cited - TODO: remove in final version
\nocite{*}
\bibliography{refs}
\cleardoublepage

%%%%%%%%%%%%%%%%%%%%%%%%%%%%%%%%%%%%%%%%%%%%%%%%%%%%%%%%%%%%%%%%%%%%%
% the appendices
\appendix

\chapter{Possible Titles}

\begin{itemize}
\item Automatic Communication Configuration in Mobile Environments
\item Data Stream Connection and Message Negotiation
\item Flexibile Data Streams in Mobile Environments
\item Automatic Data Stream Configuration in Mobile Environments
\end{itemize}

\cleardoublepage

\chapter{Example Schemas}

\begin{figure}[h]
\begin{lstlisting}
@coach
{
    int bay
    date departure
    stops
    (
    	town
    	{
    		txt name
    		date arrival
    	}
    )
}
\end{lstlisting}

\caption[Example Schema for a Coach]{A coach schema, showing a stand number, a date and time of departure, and a list of stops, with town names and estimated arrival times}
\label{fig:coach-schema}
\end{figure}

\begin{figure}[h]
\begin{lstlisting}
@train
	info
	{
		int platform
		date departure
		stops
		(
			station
			{
				txt name
				date arrival
			}
		)
	}
	shop
	{
		items
		(
			txt name
			dbl price
		)
	}
}
\end{lstlisting}

\caption[Example Schema for a Train]{A train schema, showing a structure containing a stand number, a date and time of departure, and a list of stops, with names and estimated arrival times, as well as an additional structure containing a list of items for sale in the on-board shop}
\label{fig:train-schema}
\end{figure}

\begin{figure}[h]
\begin{lstlisting}
@autobus
{
    int aire-de-stationnement
    date depart
    arrets
    (
    	ville
    	{
    		txt nom-de-la-ville
    		date arrivee
    	}
    )
}
\end{lstlisting}

\caption[Example Schema for a French Coach]{A French coach schema, showing the same fields as the English coach schema, however the field names are in French}
\label{fig:french-coach-schema}
\end{figure}

\begin{figure}[h]
\begin{lstlisting}
@taxi
{
	date available
}
\end{lstlisting}

\caption[Example Schema for a Taxi]{A taxi schema, simply showing the next available taxi}
\label{fig:taxi-schema}
\end{figure}


\cleardoublepage

% should this be an appendix?
\chapter{Project Proposal}

\vfil

\centerline{\Large Computer Science Part II Project Proposal }
\vspace{0.4in}
\centerline{\Large Automated Data Stream Management on Mobile Devices }
\vspace{0.4in}
\centerline{\large Thomas Smith (tcs40), Sidney Sussex College}
\vspace{0.3in}
\centerline{\large Originator: Jatinder Singh}
\vspace{0.3in}
\centerline{\large 17$^{th}$ October 2012}

\vfil


\noindent
{\bf Project Supervisor:} Jatinder Singh
\vspace{0.2in}

\noindent
{\bf Director of Studies:} Chris Hadley
\vspace{0.2in}
\noindent

\noindent
{\bf Project Overseers:} David Greaves  \& John Daugman


% Main document

\section*{Introduction and Description of the Project}

A typical approach for developing a smartphone app currently includes connecting to some central server over the Internet, and requesting data from it. An alternative approach would be to use data streams, where a data source and data sink are connected together, and the source automatically pushes data to the sink. Connections can then be made on an ad-hoc basis, allowing us to connect purely within a local network, or across the Internet.

When considering mobile devices, the available data sources which can be discovered will largely depend upon the context and scope in which the device is operating. This context can be affected by many factors, including location, or 3G/WiFi networks which the device is connected to. Sources may have their scope limited to a local network, such that they are only visible on that network, meaning once we go outside that scope, we will have to find a new source.

The project will investigate how a messaging middleware which handles both stream-based and request/response based interactions can be adapted to allow automated discovery of components. If a change in context causes a connected data source to become out of scope or irrelevant, the middleware would automatically search for other sources, and if possible reconnect to one providing the relevant service, allowing use of the app to continue seamlessly.

An example of this is transportation. If we were outside Cambridge Railway Station, our location could be found using GPS, and the middleware could connect over 3G to a source providing details of upcoming departures. Upon boarding a train to London, and connecting to on-board WiFi, we no longer care about other departures - the context has changed, and we only want information relating to our train. The middleware would automatically find and connect to another source, perhaps giving arrival times for stations along the route. Upon arriving at King's Cross Station, we disconnect from the WiFi network which means the previous source is now out of scope. We could connect to King's Cross WiFi, and the middleware would find and connect to sources giving data about the London Underground, and/or National Rail trains. If we were then to get on an Underground train, even though we would not be able to establish a 3G connection, there could still be a source on board the train, providing data over Bluetooth, meaning use of the app can continue.

Component discovery can be done in two ways. The first essentially uses a directory, which we can query. The second is by inspection of other components - given a component, we can ask which other components in the network is it linked to. The project will investigate using this second method to build a graph of connected components, which could be used to find a specific component.

In order to connect to a component providing the correct function, it will be necessary to determine whether a service is relevant. This will involve more than comparing message types. As show in figure \ref{fig:differentSchemas}, both schemas show data offering the same service, but with different fields. The aim is to enable an app expecting data conforming to schema (a) to also handle data from sources serving schema (b).

\begin{figure}[h]
\begin{subfigure}[b]{0.5\textwidth}
\begin{verbatim}
weather {
    dbl temperature;
}
\end{verbatim}
\caption{A simple weather schema.}
\end{subfigure}
\begin{subfigure}[b]{0.5\textwidth}
\begin{verbatim}
weather {
    dbl temperature;
    dbl humidity;
    int rainfall;
}
\end{verbatim}
\caption{A more complicated weather schema.}
\end{subfigure}
\caption{Two different schemas, both serving weather data.}
\label{fig:differentSchemas}
\end{figure}

\section*{Starting Point}

SBUS\footnote{formerly PIRATES, http://www.cl.cam.ac.uk/research/time/pirates/docs/overview.pdf} is a messaging middleware which supports client/server and stream based communication. In using a data stream, sources and sinks can be connected in a peer-to-peer network, with optional filters on events the source wishes to receive.

All of this is dynamically reconfigurable - one component can tell another to open or close connections to another component (subject to security policies). Message filters can also be changed at run time.

A component can handle multiple different message types, where each type corresponds to a different endpoint. Network connections are made to the component, and the endpoint is then specified by name. For two endpoints to communicate, their message types must match. This type is currently specified in a component file.

Component discovery can be done though a resource discovery component (RDC). A component can query the RDC using a component or endpoint name, and the RDC will then link the component to the appropriate partner. All components can be queried for identity, schemas, status, connected peers, given that the component is known.


\section*{Work to be Undertaken}

The project breaks down into the following sub-projects:

\begin{enumerate}

\item Port the existing SBUS library to Android using the Android NDK. Write an example app - a device will act as a source and transmit sensor measurements recorded with AIRS\footnote{https://play.google.com/store/apps/details?id=com.airs}, which a computer can then connect to.

\item Implement a recursive method for discovery of components through inspection. We can currently inspect a known component to see what other components are connected to it - this would extend this functionality to inspect the entire network. This would allow discovery of all other components on a network through one known component, meaning that the network can be searched without using a central resource. This will require use of graph traversal algorithm to map out the network, and an efficient data structure to store the topology of the network.

\item Connection to a new component may mean that a previously unseen data schema is being used. This will require the device to ``learn'' the new schema. This unknown schema could simply be a more, or less detailed version of a previous schema. May require use of reflection, or implementation of some handshaking protocol to exchange schemas.

\item To decide whether a new component is relevant, we need to know what service it is providing. This will require investigating search criteria in order to discover components based on the data they handle. To allow partially compatible schemas (see figure \ref{fig:differentSchemas}), we will investigate the use of ontologies to classify message types into meaningful categories.

\item Detect and act upon a change in context - search for new components using a RDC or inspection and where appropriate connect the device to these components. This will require monitoring various sensors on the device, such as location and connected networks in order to detect a change in context.

\item Create apps demonstrating the use of this system, and record metrics from the device, which can be used as a comparison against other implementations.


\end{enumerate}

\section*{Success Criterion for the Project}


The project will be a success if I can demonstrate that a mobile device can automatically reconnect to a different component upon a change in context, similar to the example outlined in the Introduction.

We can compare the two methods of component discovery, to determine which is more appropriate in different scenarios. We can measure how effectively a component can accept partial schemas - whether performance deteriorates when a message contains far more fields than expected.

The project can further be evaluated by comparison of an app developed using this system, to one developed using other approaches. There are many metrics which can be measured for comparison. From a performance viewpoint, we can measure overheads such as power consumption, network usage, and time taken to reconnect to a service.
In development we can measure lines of code and the reusability of code - one app could even have multiple use cases! For example, one travel app, as described in the Introduction would replace the need for separate National Rail and London Underground apps.



\section*{Possible Extensions}

If the project were completed early, I could look at dynamic visibility of components. Components can specify security policies restricting access to themselves, which also prevents them being discovered. If this access control were to change after searching for the component, we would still have no knowledge of it unless searching again. This extension would look at how a component could be made aware, or even a connection automatically established to the previously ``hidden'' component, if the access control changes to allow us to discover it. This could be useful if we wanted to grant access to a larger group of users, from a previously prioritised group.


\section*{Timetable and Milestones}

Planned starting date is 19/10/2012.

\begin{enumerate}

\item {\bf Michaelmas weeks 3-4} Get SBUS in existing form running on Android, either by creating an interface to the native code, or writing an app entirely in native code. Write example sensor app.

\item {\bf Michaelmas weeks 5-6} Investigate algorithms to build a graph of the network of components, and implement an appropriate one for recursive component inspection to allow decentralised component discovery.

\item {\bf Michaelmas weeks 7-8} Implement a method to search for a new component when necessary. This will also include detecting when it is necessary - detect change of network, location change.

\item {\bf MILESTONE: } Automatic reconnection to component complete.

\item {\bf Michaelmas vacation} Write code to allow searching for components by type. This will be done in a way such that more general versions of a schema can be found, if a specific one cannot.

\item {\bf Lent weeks 0-2} Extend current connection method to allow components to connect to schemas providing the correct service, but with different fields.

\item {\bf MILESTONE: } Dynamic data typing complete.

\item {\bf Lent weeks 3-5} Create example apps demonstrating the project and apps using a client/server mechanism, and measure overheads of using these in order to evaluate the project.

\item {\bf Lent weeks 6-8} Finish any remaining programming.

\item {\bf MILESTONE: } Programming complete.

\item {\bf Easter vacation:} Write draft dissertation.

\item {\bf Easter weeks 0-2:} Write final copy of dissertation, and ensure all demos are working.

\item {\bf Easter week 3:} Early submission, allowing one week in case the project overruns.

\item {\bf MILESTONE: } Project complete.

\end{enumerate}

\section*{Resource Declaration}

For this project I will use my own computer running Ubuntu 12.04. I accept full responsibility for this machine and I have made contingency plans to protect myself against hardware and/or software failure. Backup will be to a repository hosted by GoogleCode, as well as copies to my Dropbox folder, and to USB drives. I will require use of Android devices for testing - either my own, or provided by the Computer Lab.

I will use the Android SDK\footnote{http://developer.android.com/sdk/index.html} and Android NDK\footnote{http://developer.android.com/tools/sdk/ndk/index.html} to develop Android applications with the ability to run native code.

I will be using the SBUS library as the data stream middleware, provided by Jat Singh.

Should my machine suddenly fail, I will use the PWF computers running Linux. 


\end{document}
